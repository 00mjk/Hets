\documentclass{entcs} \usepackage{entcsmacro}
\usepackage{graphicx}
\usepackage{mathpartir}
\usepackage{amsmath,amssymb}
\newcommand{\hearts}{\heartsuit}
\newcommand{\prems}{\mathit{prems}}
\newcommand{\eval}{\mathit{eval}}

\newcommand{\mi}[1]{\mathit{#1}}

% defs.tex ---
% for WADT99
% copied from Tarlecki's definitions for frocos'98
% modified from moving defs.tex

\newcommand{\ASP}{\mathit{ASP}}
\newcommand{\SP}{\mathit{SP}}
\newcommand{\UDD}{\mathit{UDD}}
\newcommand{\UT}{\mathit{T}}
\newcommand{\UN}{\mathit{U}}
\newcommand{\UDecl}{\mathit{Dcl}}
\newcommand{\UDefn}{\mathit{Dfn}}

\newcommand{\parspec}[3]{#1{\stackrel{#2}{\longrightarrow}}#3}

\newcommand{\archspec}[2]{\textbf{arch spec }#1\textbf{ result }#2}
\newcommand{\decl}[2]{#1\colon#2}
\newcommand{\dfn}[2]{#1=#2}
\newcommand{\amlg}[4]{#1\textbf{ with }#2\textbf{ and }#3\textbf{ with }#4}
\newcommand{\appl}[3]{#1\lbrack#2\textbf{ fit }#3\rbrack}
\newcommand{\gris}{\mathrel{::=}}


\newcommand{\Mset}{\mathcal{M}}
\newcommand{\Fset}{\mathcal{F}}

\newcommand{\Dgm}{\mathit{Diag}}
\newcommand{\Shape}[1]{\mathit{Shape}(#1)}
\newcommand{\Nodes}[1]{\mathit{Nodes}(#1)}
\newcommand{\Edges}[1]{\mathit{Edges}(#1)}
%\newcommand{\restrict}[2]{\reduct{#1}{#2}}

\newcommand{\fm}[2]{\langle#1\rangle_{#2}}

\newcommand{\incl}[2]{\iota_{{#1}\subseteq{#2}}}

\newcommand{\st}{\mathit{st}}
\newcommand{\stC}{\mathit{C}_{\st}}
\newcommand{\stB}{\mathit{B}_{\st}}
\newcommand{\stP}{\mathit{P}_{\st}}
\newcommand{\emptystC}{\stC^\emptyset}
\newcommand{\emptyUC}{\UC^\emptyset}
\newcommand{\UC}{\mathcal{C}}
\newcommand{\UEv}{\mathit{UEv}}
\newcommand{\Uset}{\mathcal{V}}
\newcommand{\UU}{\mathit{V}}

\newcommand{\EstC}{\mathcal{C}_{\st}}
\newcommand{\emptyEstC}{\EstC^\emptyset}
\newcommand{\EstB}{\mathcal{B}_{\st}}
\newcommand{\SD}{D}

\newcommand{\CstC}{\mathfrak{C}_{\st}}

\newcommand{\getstctx}{\mathit{ctx}}

\newcommand{\env}{\mathit{E}}

\newcommand{\lambdaexp}[2]{\lambda#1\cdot#2}
\newcommand{\reduct}[2]{#1|_{#2}}

\newlength{\croutw}
\newlength{\crouth}
\newcommand{\crossout}[1]%
        {\settowidth{\croutw}{$#1$}\settoheight{\crouth}{$#1$}#1%
        \hspace{-1.0\croutw}\raisebox{0.3\crouth}{\rule{\croutw}{0.1ex}}}

\newcommand{\commentout}[1]{\ignorespaces}
\newcommand{\underscore}{\rule{3mm}{0.005in}}

\newcommand{\extaticsem}[3]{#1 \vdash #2 \extsmb #3}
\newcommand{\extsmb}{\mathrel{\rhd\hspace{-0.5em}\rhd}}

\newcommand{\rulesection}[1]{\[\fbox{$#1$}\]}
\newcommand{\infrule}[2]{\frac{#1}{#2}}
\newcommand{\staticsem}[3]{#1 \vdash #2 \rhd #3}
\newcommand{\modelsem}[3]{#1 \vdash #2 \Rightarrow #3}
\newcommand{\staticsembreak}[3]{\begin{array}{r}#1 \vdash #2 \qquad\\ 
                                                \rhd #3 \end{array}}
\newcommand{\modelsembreak}[3]{\begin{array}{r}#1 \vdash #2 \qquad\\
                                                 \Rightarrow #3 \end{array}}



% Defs from Till 


%\newenvironment{proof}{\begin{trivlist}\item[\textbf{Proof. 
%}]}{\end{trivlist}}

%\newcommand{\qed} {\hfill{$\Box$}}

\newcommand{\ttsize}{\footnotesize }
\newcommand{\NT}[1]{{\ttsize\texttt{#1}}}

%\newcommand{\gram}[1]{{\texttt{#1}}}
%\newenvironment{grammar}
% {\texonly{\footnotesize}
%  \begin{example}}{\end{example}\normalsize}
%\newenvironment{slgrammar}
% {\texonly{\footnotesize}
%  \begin{example}\texorhtml{\slshape}{\it}}{\end{example}\normalsize}


\newcommand{\cofidirectory}{}


%symbols etc
\newcommand{\parrightarrow}{\p\rightarrow}
\newcommand{\p}[1]{\mathrel{\ooalign{\hfil$\mapstochar\mkern 
5mu$\hfil\cr$#1$}}}
%\newsavebox{\pararrbox}
%\savebox{\pararrbox}{$\ \rightarrow\!\!\!\!\!\!{\raisebox{.15cm}{\tiny
%p}}\ \ $}
%\newcommand{\classicalpararrow}{{\usebox{\pararrbox}}}
\newcommand{\totrightarrow}{\rightarrow}


\newcommand{\X}{CASL}
%\newcommand{\congr}{\equiv}
%\newcommand{\TSX}{{T_{\Sigma}(X)}}
\newcommand{\LSX}{{W_{\Sigma}(X)}}

\newcommand{\Powfin}{\mathcal{P}_\omega}
\newcommand{\gen}[1]{\mid_{#1}}
\newcommand{\cogen}[1]{\mid^{#1}}
\newcommand{\induce}[2]{\mid^{#1}_{#2}}
\newcommand{\inducefrom}[1]{\!\mid_{#1}}
\newcommand{\induceto}[1]{\mid^{#1}}

%\makeindex

%\newcommand{\Mod}{\mathbf{Mod}}
\newcommand{\sen}{\mathbf{Sen}}

%\newcommand{\Sen}{\mathbf{Sen}}

%\newcommand{\Sign}{\mathbf{Sign}}
%\newcommand{\Set}{\mathbf{Set}}
%\newcommand{\ttsize}{\footnotesize }
%\newcommand{\NT}[1]{{\ttsize\texttt{#1}}}
\newcommand{\PF}{\mathit{PF}}
\newcommand{\TF}{\mathit{TF}}

\newcommand{\SY}{\mathit SY}
\newcommand{\SYs}{\mathit SYs}
\newcommand{\RSY}{\mathit RSY}
\newcommand{\RSYs}{\mathit RSYs}
\newcommand{\Sym}{\mathit{Sym}}
\newcommand{\RawSym}{\mathit{RawSym}}
\newcommand{\RawSymMap}{\mathit{RawSymMap}}
\newcommand{\IDAsRawSym}{\mathit{IDAsRawSym}}
\newcommand{\SymAsRawSym}{\mathit{SymAsRawSym}}

%\newcommand{\dom}{\mathit{dom}}
%\newcommand{\graph}{\mathit{graph}}
%% incompatible with xypic -- dirk


%\newenvironment{grammar}
% {\ttsize\begin{alltt}}{\end{alltt}}



\newcommand{\TSX}{{T_{\Sigma^\#}(X)}}

\newcommand{\bit}{\begin{itemize}}
\newcommand{\eit}{\end{itemize}}
\newcommand{\Equivalent}{\Leftrightarrow}
%\newcommand{\Implies}{\Rightarrow}
\newcommand{\logand}{\wedge}
\newcommand{\logor}{\vee}
%\newcommand{\union}{\cup}
\newcommand{\intersection}{\cap}
\newcommand{\comp}{\circ}
\newcommand{\suchthat}{\mid}
\newcommand{\congr}{\equiv}
\newcommand{\disjoint}{\uplus}
\newcommand{\buildset}[1]{\{#1\}}
\newcommand{\Si}{\Sigma}
\newcommand{\al}{\alpha}
\newcommand{\pp}[2]{#1_1\commadots #1_{#2}}
%\newcommand{\map}[2]{\colon#1\!\longrightarrow\!#2}
\newcommand{\mmap}[2]{#1\!\longrightarrow\!#2}
\newcommand{\anddots}{\logand\,\cdots\,\logand}
\newcommand{\lldots}{\,\ldots\,}
\newcommand{\ccdots}{\,\cdots\,}
\newcommand{\commadots}{,\lldots,}
\newcommand{\sigspec}[1]{\langle #1 \rangle}
\newcommand{\eq}{\stackrel{\mbox{\scriptsize e}}{=}}
\newcommand{\weq}{\stackrel{\mbox{\scriptsize w}}{=}}

\newcommand{\mBox}[1]{\, \mbox{#1} \,}

\newcommand{\engtwocase}[3]{
\left\{
\begin{array}{ll}
  #1,&\mBox{if }#2\\
  #3,&\mBox{otherwise}
\end{array}\right.}

\newcommand{\threecase}[5]{
\left\{
\begin{array}{ll}
  #1,&\mbox{if }#2\\
  #3,&\mbox{if }#4\\
  #5,&\mbox{otherwise}
\end{array}\right.}

\newcommand{\threefullcase}[6]{
\left\{
\begin{array}{ll}
  #1,&\mbox{if }#2\\
  #3,&\mbox{if }#4\\
  #5,&\mbox{if }#6
\end{array}\right.}

\newcommand{\delete}[1]{$\overline{\mbox{#1}}$}
\newcommand{\new}[1]{{\it\tiny #1}}

\newcommand{\mbs}[1]{\mbox{$#1$}}




\newcommand{\ASL}[1]{[\![#1]\!]}

\newcommand{\sym}{\mathit{sym}}
\newcommand{\totqual}{\mathsf{t}}
\newcommand{\parqual}{\mathsf{p}}
\newcommand{\ws}{\mathit{ws}}
\newcommand{\FinSet}[1]{\mathit{FinSet}(#1)}
\newcommand{\RawSymOf}{\mathit{RawSymOf}}
\newcommand{\Gram}[1]{{\textup{\texttt{#1}}}}

\newcommand{\implicitqualkind}{\mathit{implicit}}
\newcommand{\sortqualkind}{\mathit{sort}}
\newcommand{\funqualkind}{\mathit{fun}}
\newcommand{\predqualkind}{\mathit{pred}}
\newcommand{\SymKind}{\mathit{SymKind}}

%\newcommand{\Sort}{\mathit{Sort}}
\newcommand{\QualFunName}{\mathit{QualFunName}}
\newcommand{\QualPredName}{\mathit{QualPredName}}



% Lutz's definitions


\hfuzz1pc
\sloppy
%\frenchspacing


%\spnewtheorem{thm}{Theorem}[section]{\bfseries}{\itshape}
%\spnewtheorem{cor}[theorem]{Corollary}{\bfseries}{\itshape}
%\spnewtheorem{lem}[theorem]{Lemma}{\bfseries}{\itshape}
%\spnewtheorem{prop}[theorem]{Proposition}{\bfseries}{\itshape}
%\spnewtheorem{defn}[theorem]{Definition}{\bfseries}{\upshape}
%\spnewtheorem{rem}[theorem]{Remark}{\bfseries}{\upshape}
%\spnewtheorem{expl}[theorem]{Example}{\bfseries}{\upshape}
%\spnewtheorem{thmdefn}[theorem]{Theorem and Definition}{\bfseries}{\itshape}
%\spnewtheorem{propdefn}[theorem]{Proposition and Definition}{\bfseries}{\itshape}
%\spnewtheorem{assumption}[theorem]{Assumption}{\bfseries}{\upshape}
%\spnewtheorem{algorithm}[theorem]{Algorithm}{\bfseries}{\upshape}

%% Other environments
%\newenvironment{mycd}
%       {\begin{displaymath}\renewcommand{\arraystretch}{0.5}\begin{CD}}
%       {\end{CD}\end{displaymath}}
\newenvironment{rmenumerate}%
        {\begin{enumerate}\renewcommand{\labelenumi}{(\roman{enumi})}}%
        {\end{enumerate}}
\newenvironment{alenumerate}%
        {\begin{enumerate}\renewcommand{\labelenumi}{(\alph{enumi})}}%
        {\end{enumerate}}
\newenvironment{Alenumerate}%
        {\begin{enumerate}\renewcommand{\labelenumi}{\Alph{enumi})}}%
        {\end{enumerate}}
\newenvironment{numenumerate}%
        {\begin{enumerate}\renewcommand{\labelenumi}{\arabic{enumi})}}%
        {\end{enumerate}}
%\newenvironment{defn}%
%       {\begin{definition}\em}%
%       {\end{definition}}
%\newenvironment{rem}%
%       {\begin{remark}\em}%
%       {\end{remark}}
%\renewenvironment{example}%
%       {\begin{expl}\em}%
%       {\end{expl}}
%\newenvironment{texteqnarray}%
%       {\setlength{\itemsep}{0pt}%
%       \setlength{\topsep}{0pt}%
%       \begin{trivlist}\item[]\begin{displaymath}%
%       \renewcommand{\arraystretch}{1.2}\begin{array}{rcll}}%
%       {\end{array}\end{displaymath}\end{trivlist}}
%\newenvironment{acknowledge}%
%       {\begin{list}{}{\setlength{\leftmargin}{0pt}%
%                       \setlength{\topsep}{1cm}}% 
%       \item{\bf Acknowledgements:}}%
%       {\end{list}}

%\newcommand{\qed}{\hfill\openbox}
%\newenvironment{proof}
%       {\begin{trivlist}\item{\sc Proof: }}
%       {\qed\end{trivlist}}



%% Math definitions
\newcommand{\Cat}{\mathbf}
\newcommand{\Cls}{\mathcal}
\newcommand{\Opname}{\mathrm}

\newcommand{\Ob}{{\Opname{Ob}\,}}
\newcommand{\Mor}{{\Opname{Mor}\,}}
\newcommand{\Epi}{{\Opname{Epi}\,}}
\newcommand{\Mono}{{\Opname{Mono}\,}}
\newcommand{\Iso}{{\Opname{Iso}\,}}
\newcommand{\Ident}{{\Opname{Ident}\,}}
\newcommand{\Op}{{op}}
\newcommand{\colim}{{\Opname{colim}\,}}

\newcommand{\lrule}[3]{(#1)\;\;\infrule{#2}{#3}}
\newcommand{\lrrule}[3]{(#1)\;\;\infrule{\underline{#2}}{#3}}
\newcommand{\laxiom}[2]{(#1)\;\;#2}

\newcommand{\BA}{{\Cat A}}
%\newcommand{\BB}{{\Cat B}}
\newcommand{\BC}{{\Cat C}}
\newcommand{\BD}{{\Cat D}}
\newcommand{\BI}{{\Cat I}}
\newcommand{\BK}{{\Cat K}}
\newcommand{\BL}{{\Cat L}}
\newcommand{\BM}{{\Cat M}}
\newcommand{\BN}{{\Cat N}}
\newcommand{\BS}{{\Cat S}}
\newcommand{\BW}{{\Cat W}}
\newcommand{\Btwo}{{\Cat 2}}

\newcommand{\CC}{{\Cls C}}
\newcommand{\CE}{{\Cls E}}
%\newcommand{\CK}{{\Cls K}}
% \newcommand{\CL}{{\Cls L}}
\newcommand{\CM}{{\Cls M}}
\newcommand{\CO}{{\Cls O}}
\newcommand{\CP}{{\Cls P}}
\newcommand{\CS}{{\Cls S}}
\newcommand{\CT}{{\Cls T}}
\newcommand{\eps}{\varepsilon}

\newcommand{\integers}{{\mathbb Z}}
\newcommand{\tensor}{\otimes}

%\newcommand{\openbox}{\leavevmode
%  \hbox to.77778em{%
%  \hfil\vrule
%  \vbox to.675em{\hrule width.6em\vfil\hrule}%
%  \vrule\hfil}}
\newcommand{\mystrut}[1]{\rule[#1]{0cm}{0.1cm}}



\newcommand{\red}{\vdash}
\newcommand{\idred}{\red}
\newcommand{\step}{\red_c}
% \newcommand{\stepeq}{\succeq}
\newcommand{\downto}{\succeq}
\newcommand{\upto}{\preceq}
\newcommand{\moremonicthan}{\Rightarrow}
\newcommand{\into}{\hookrightarrow}
\newcommand{\impl}{\Rightarrow}
\newcommand{\id}{{id}}
\newcommand{\ev}{{ev}}
\newcommand{\Hom}{{hom}}
\newcommand{\Dom}{{dom\, }}
\newcommand{\Cod}{{cod}}
\newcommand{\adj}{\dashv}
\newcommand{\powerset}{{\mathcal P}}
\newcommand{\finpowerset}{{\powerset_\omega}}
\newcommand{\kVec}{\mbox{$k$-$\Cat{Vec}$}}
\newcommand{\termObj}{1}

\newcommand{\map}[2]{:#1\to #2}
\newcommand{\restr}[2]{{#1}|_{#2}}
\newcommand{\wordbrace}[1]{\underbrace{\hspace{1.5cm}}_{\displaystyle{#1}}}
\newcommand{\forget}[1]{|_{#1}}
\newcommand{\argument}{\_\!\_}%{\underline{\;\;}}
%\newcommand{\substack}[2]{{{#1} \atop {#2}}}

%% Category Names
\newcommand{\CASLsign}{\Cat{CASLsign}}
\newcommand{\enrCASLsign}{\Cat{enrCASLsign}}
\newcommand{\refCASLsign}{\Cat{refCASLsign}}
%\newcommand{\Mod}{\Cat{Mod}\,}
% the \, caused bad effect when copmbined with indices etc.
\newcommand{\EnrMod}{\Cat{Mod}_e}
\newcommand{\CAT}{\Cat{CAT}}
\newcommand{\Sen}{\Cat{Sen}}
\newcommand{\Sign}{\Cat{Sign}}
\newcommand{\SignVar}{\Cat{Var}}
\newcommand{\EnrSign}{\Cat{EnrSign}}
%\newcommand{\Set}{\Cat{Set}}
\newcommand{\CLS}{\Cat{CLS}}

\newcommand{\SmallSpecName}[1]{\textsc{\small #1}}
%% Institution Names
\newcommand{\EnrInst}{EnrPCFOL}
\newcommand{\CASLInst}{SubPCFOL}
\newcommand{\PCFOLInst}{PCFOL}

%% Words
\newcommand{\Word}{\mathbf}
\newcommand{\Ba}{{\Word{a}}}
\newcommand{\Bb}{{\Word{b}}}
\newcommand{\Bc}{{\Word{c}}}
\newcommand{\Bf}{{\Word{f}}}
\newcommand{\Bg}{{\Word{g}}}
\newcommand{\Bh}{{\Word{h}}}
\newcommand{\Bl}{{\Word{l}}}
\newcommand{\Br}{{\Word{r}}}



%% For HasCASL:

\newcommand{\HasCASL}{{\sc HasCasl}\xspace}
\newcommand{\ML}{{\textsf{ML}}\xspace}
\newcommand{\BaseTypes}{B}
\newcommand{\SimpleTypes}{T}
\newcommand{\TotalFunType}[2]{#1\to #2}
\newcommand{\sbullet}{\mbox{ $\scriptstyle\bullet$ }}
\newcommand{\lambdaTotal}[2]{\lambda\, #1\sbullet\!\! !\, #2}
\newcommand{\PartialFunType}[2]{#1\to ?#2}
\newcommand{\lambdaPartial}[2]{\lambda\, #1\sbullet #2}
\newcommand{\lambdaType}[2]{\lambda\, #1\sbullet #2}
\newcommand{\NonStrictFunType}[2]{\PartialFunType{\Maybe{#1}}{#2}}
\newcommand{\lambdaNonStrict}[2]{\lambda\, #1\sbullet #2}
\newcommand{\substTerm}[3]{{#1[#3/#2]}}
\newcommand{\PredType}[1]{{\textsf{pred}(#1)}}
\newcommand{\ProdType}[2]{#1 * #2}
\newcommand{\pMonType}[2]{#1 -\!\!\mu\!\!\to\hspace{-1pt}? #2}
\newcommand{\tMonType}[2]{#1 -\!\!\mu\!\!\to #2}
\newcommand{\pContType}{Pcont}
\newcommand{\tContType}{Tcont}
\newcommand{\projOp}[1]{{pr_#1}}
\newcommand{\fst}{{\textsf{fst}}}
\newcommand{\snd}{{\textsf{snd}}}
\newcommand{\UnitType}{{\textsf{unit}}}
\newcommand{\UnitOp}{{()}}
\newcommand{\Maybe}[1]{?#1}
\newcommand{\TypeSchemes}{{TS}}
\newcommand{\TypeContext}{\mathcal{C}}
\newcommand{\UniversalType}[2]{{\forall #1\sbullet #2}}
\newcommand{\applyPoly}[2]{{#1[#2]}}
%\newcommand{\lambdaPoly}[2]{\Lambda #1:\TypeKind.\, #2}
\newcommand{\ConstrainedType}[2]{{#1 \Rightarrow #2}}
\newcommand{\Constraint}{C}
\newcommand{\TypeKind}{\textsf{type}}
\newcommand{\TypeVars}{{TV}}
\newcommand{\TypeOps}{{TO}}
\newcommand{\PseudoTypes}{{PT}}
\newcommand{\SubstType}[3]{{#1[#3/#2]}}
\newcommand{\varco}[1]{{v^+(#1)}}
\newcommand{\varcontra}[1]{{v^-(#1)}}
\newcommand{\Covar}{\textsf{covariant}}
\newcommand{\Contravar}{\textsf{contravariant}}
\newcommand{\sigSorts}{S}
\newcommand{\sigClasses}{C}
\newcommand{\sigConstrs}{T}
\newcommand{\sigAlias}{{A}}
\newcommand{\sigExpand}{{\varepsilon}}
\newcommand{\unalias}{{\bar\sigExpand}}
\newcommand{\sigSub}{\leq}
\newcommand{\extSub}{\leq_*}
\newcommand{\elPred}[2]{{#1\in #2}}
\newcommand{\downCast}[2]{{#1\textsf{ as }#2}}
\newcommand{\upCast}[2]{{#1:#2}}
\newcommand{\sigOps}{O}
\newcommand{\IndVars}{V}
\newcommand{\ifthenelse}[3]{{\textsf{if }#1\textsf{ then }#2
                        \textsf{  else }#3}}
\newcommand{\BT}{B}
\newcommand{\emb}{{emb}}

\newcommand{\Context}{\Gamma}
\newcommand{\ContextTerm}[3]{{#1 \rhd #2: #3}}
\newcommand{\ContextEq}[2]{{#1 \rhd #2}}
\newcommand{\ContextEntails}[3]{{#2 \entails_{#1} #3}}
\newcommand{\ContextImpl}[3]{{#2 \impl_{#1} #3}}
\newcommand{\exeq}{\stackrel{e}{=}}
\newcommand{\IsDef}{\operatorname{def}}

\newcommand{\isFormula}[1]{{#1\in\textsf{Prop}}}

\newcommand{\Model}{M}
\newcommand{\Sem}[1]{{[\![#1]\!]}}
\newcommand{\extendsTo}{\preceq}
\newcommand{\True}{\top}
\newcommand{\False}{\bot}
\newcommand{\uexists}{\exists !}
\newcommand{\iotaTerm}[2]{\iota #1\sbullet #2}
%\newcommand{\conj}{\wedge}
\newcommand{\disj}{\vee}
\newcommand{\modimpl}{\to}
\newcommand{\modiff}{\leftrightarrow}
\newcommand{\intEq}{eq}
\newcommand{\dOrder}[3]{#2 \le\![#1]\,#3}
\newcommand{\resTerm}[2]{#1\ res\ #2}
\newcommand{\mono}[2]{\operatorname{mono}_{#1}#2}
%\newcommand{\entails}{\vdash}

%% CASL constructs

%% HasCASL constructs
\newcommand{\Class}{\textbf{class}}
\newcommand{\Instance}{\textbf{instance}}
\newcommand{\Program}{\textbf{program}}
\newcommand{\Internal}{\textbf{internal}}
\providecommand{\pfun}{\mathrel{\rightarrow?}}
\newcommand{\iimpl}{\Rightarrow}
\newcommand{\iconj}{\wedge}

%% For Monads:
\newcommand{\BBT}{\mathbb{T}}
\newcommand{\DO}{\operatorname{do}}
\newcommand{\Let}{\operatorname{let}}
\newcommand{\retOp}{\operatorname{ret}}
\newcommand{\retOpHC}{\mathit{ret}}
\newcommand{\ret}[1]{\retOp #1}
\newcommand{\letTerm}[2]{\DO\, #1; \ #2}
\newcommand{\leteq}{\leftarrow}
\newcommand{\se}[2]{\mathsf{se}(#1,#2)}
\newcommand{\Pfin}{\mathcal P_{\mathit{fin}}}

%% Program logic
\newcommand{\nec}{\square\,}
\newcommand{\gbox}{\square\hspace{-6.5pt}\raisebox{2pt}{\tiny{G}}\;}
\newcommand{\pbox}[1]{[#1]\,}
\newcommand{\pdiamond}[1]{<\hspace{-3pt}#1\hspace{-3pt}>\!}
%\newcommand{\lbox}{\square\,}
\newcommand{\lbox}{\Box}
\newcommand{\ldiamond}{\Diamond}
\newcommand{\pmodal}[2]{[#1]_G\,#2}
\newcommand{\Timpl}{\Rightarrow_T}
\newcommand{\Tconj}{\wedge}
\newcommand{\HTriple}[3]{\{#1\}~#2~\{#3\}}
\newcommand{\ContextHTriple}[4]{#1\rhd\HTriple{#2}{#3}{#4}}
\newcommand{\DummyCHT}[4]{\HTriple{#2}{#3}{#4}}
\newcommand{\sef}{\mathit{sef}}
\newcommand{\dsef}{\mathit{dsef}}

%% For polymorphism
\newcommand{\extInst}[1]{\mathrm{Ext}(#1)}
%\newcommand{\gextInst}[1]{\mathrm{gExt}(#1)}
\newcommand{\Poly}[1]{\mathrm{Poly}(#1)}
%% For institutions


\newcommand{\ModFunctor}{\Cat{Mod}\,}


\newcommand{\la}{\langle}
\newcommand{\ra}{\rangle}


\newcommand{\si}{\sigma}

\newcommand{\ISig}{\mathit{Sig}}
\newcommand{\IMod}{\mathit{Mod}}
\newcommand{\IAx}{\mathit{Ax}}

\newcounter{blubber}

\newenvironment{algenumerate}
{\begin{list}
  {\arabic{blubber}.}
  {\usecounter{blubber}
   \setlength{\leftmargin}{3ex}
    \setlength{\parsep}{0pt}
    \setlength{\itemindent}{1ex}
    \setlength{\itemsep}{2pt}   
    \setlength{\listparindent}{0ex}
  }
}
{\end{list}}

\newenvironment{specsem}%
        {\smallskip\par\noindent\qquad$\begin{array}{@{}l@{{}={}}l}}%
        {\end{array}$\par\noindent}


%% Width of parboxes for figures
\newlength{\myboxwidth}
\setlength{\myboxwidth}{\textwidth}
\addtolength{\myboxwidth}{-20pt}
\newenvironment{myfigure}{\begin{figure}\begin{center}
	\setlength{\fboxsep}{10pt}}% 
	{\end{center}\end{figure}}

%% Coalgebraic Modal Logic
\newcommand{\Lang}{\mathcal{L}}	
\newcommand{\FLang}{\mathcal{F}}	
\newcommand{\pls}{\Lambda}
\newcommand{\pl}[3]{#1\in #2(#3)}
\newcommand{\polypl}[3]{#1\in #2 #3}
\newcommand{\negpl}[3]{#1\notin #2(#3)}
\newcommand{\plbox}[1]{[#1]}
%\newcommand{\pldiamond}[1]{<\hspace{-3pt}#1\hspace{-3pt}>\!}
\newcommand{\pldiamond}[1]{\langle#1\rangle}
\newcommand{\gldiamond}[1]{\Diamond_{#1}}
\newcommand{\glbox}[1]{\square_{#1}}

\newcommand{\HMLBox}[1]{\square\hspace{-5.5pt}\raisebox{2pt}{$\scriptscriptstyle{#1}$}\;}
\newcommand{\HMLDiamondb}[1]{\Diamond\hspace{-5pt}\raisebox{1.5pt}{$\scriptscriptstyle{#1}$}\;}
\newcommand{\HMLDiamonda}[1]{\Diamond\hspace{-5.3pt}\raisebox{1.9pt}{$\scriptscriptstyle{#1}$}\;}
\newcommand{\pldbox}[2]{[#1]_{#2}}
\newcommand{\plempty}{\plbox{\emptyset}}
\newcommand{\plcan}{\square}
\newcommand{\FC}{{\mathfrak C}}
\newcommand{\negcl}[1]{\mathit{cl}(#1)}
\newcommand{\Nat}{{\mathbb{N}}}
\newcommand{\Int}{{\mathbb{Z}}}
\newcommand{\Rat}{{\mathbb{Q}}}
\newcommand{\Real}{{\mathbb{R}}}
\newcommand{\List}{\mathsf{list}}
\newcommand{\PDist}{D_\omega}%% Provisionary!!
%\newcommand{\Id}{\mathit{Id}}
\newcommand{\SFun}{\mathcal{SP}}
\newcommand{\plcomp}{\circledast}
\newcommand{\contrapower}{2^{\argument}}
\newcommand{\trans}[1]{#1^\flat}
\newcommand{\boolcl}[2]{\mathrm{bcl}_{#1}(#2)}
\newcommand{\Bag}{\mathcal{B}}
\newcommand{\Baginfty}{\mathcal{B}_\infty}
%\newcommand{\Ab}{\mathcal{B}_{\mathbb{Z}}}
\newcommand{\Prop}{\mathsf{Prop}}
\newcommand{\Up}{\mathsf{Up}}
\newcommand{\Lit}{\mathsf{Lit}}
\newcommand{\nneg}{\sim}
\newcommand{\satisfies}{\vDash}
\newcommand{\gsatisfies}{\satisfies_g}
\newcommand{\nsatisfies}{\nvDash}
\newcommand{\PSPACE}{\mathit{PSPACE}}
\newcommand{\Rules}{\mathcal{R}}
\newcommand{\RulesC}{\Rules_C}
\newcommand{\Frame}[1]{2^{2^{#1}}}
\newcommand{\UpP}{\mathsf{Up}\mathcal{P}}
\newcommand{\one}{\mathbb{1}}
%\newcommand{\smaller}[2]{#1_{<#2}}
\newcommand{\total}{\nu}
\newcommand{\sgn}{\mathit{sgn}}
%\newcommand{\RedClause}{\Clause^r}
\newcommand{\size}{\mathit{size}}
\newcommand{\arity}{\mathit{ar}}

\newcommand{\Sorts}{\mathcal{S}}
\newcommand{\MSorts}{\Sorts_0}
\newcommand{\ModOp}{L}
\newcommand{\contrapow}{\mathcal{Q}}
\newcommand{\profto}{\stackrel{\bullet}{\to}}
\newcommand{\Struct}{\mathcal{M}}
\newcommand{\Id}{\mathit{Id}}
\newcommand{\Det}{\mathsf{Det}}
\newcommand{\hDet}{\mathsf{hDet}}
\newcommand{\FF}{\mathfrak{F}}
\newcommand{\idOp}{\plbox{\iota}}
\newcommand{\Sig}{\mathit{Sig}}

\newcommand{\NEXP}{\mi{NEXPTIME}}
\newcommand{\EXP}{\mi{EXPTIME}}
\newcommand{\NP}{\mi{NP}}
\newcommand{\at}{\mi{at}}
\newcommand{\Max}{\mi{max}}
\newcommand{\muprod}{\textstyle\prod^\mu}
\newcommand{\tsum}{\textstyle\sum}
\DeclareMathOperator{\mge}{\ge}
\DeclareMathOperator{\meq}{=}
\DeclareMathOperator{\mle}{\le}
\DeclareMathOperator{\fixarrow}{\uparrow}
\newcommand{\Stable}{S}
\newcommand{\Zero}{Z}
\newcommand{\Regular}{U}
\newcommand{\last}{\lambda}
\newcommand{\fix}[2]{#1\!\fixarrow #2}
\newcommand{\msf}{\mathsf}

\newcommand{\CKCMi}{\CK\!+\!\CMi\xspace}
\newcommand{\CK}{\mathit{CK}}
\newcommand{\CMi}{\mathit{CMi}}
\newcommand{\Ax}{\mathcal{A}}

%\newcommand{\PLentails}{\entails_{\mi{PL}}}
%\newcommand{\ModSig}{\Lambda}

\newcommand{\modelsCA}{\models}
\newcommand{\Th}{\mathsf{Th}}
\newcommand{\rank}{\operatorname{rank}}
\newcommand{\Space}{\mathcal{S}}
\newcommand{\osder}[2]{\entails_{#1}^{#2}}
\newcommand{\gentails}[1]{\entails^{#1}_g}

\sloppy
% The following is enclosed to allow easy detection of differences in
% ascii coding.
% Upper-case    A B C D E F G H I J K L M N O P Q R S T U V W X Y Z
% Lower-case    a b c d e f g h i j k l m n o p q r s t u v w x y z
% Digits        0 1 2 3 4 5 6 7 8 9
% Exclamation   !           Double quote "          Hash (number) #
% Dollar        $           Percent      %          Ampersand     &
% Acute accent  '           Left paren   (          Right paren   )
% Asterisk      *           Plus         +          Comma         ,
% Minus         -           Point        .          Solidus       /
% Colon         :           Semicolon    ;          Less than     <
% Equals        =3D           Greater than >          Question mark ?
% At            @           Left bracket [          Backslash     \
% Right bracket ]           Circumflex   ^          Underscore    _
% Grave accent  `           Left brace   {          Vertical bar  |
% Right brace   }           Tilde        ~

% A couple of exemplary definitions:

%\newcommand{\Nat}{{\mathbb N}}
%\newcommand{\Real}{{\mathbb R}}
\newcommand{\COLOSS}{{\textrm CoLoSS}}
\def\lastname{Hausmann and Schr\"oder}
\begin{document}
\begin{frontmatter}
  \title{Optimizing Conditional Logic Reasoning within \COLOSS}
  \author[DFKI]{Daniel Hausmann\thanksref{myemail}}
  \author[DFKI,UBremen]{Lutz Schr\"oder\thanksref{coemail}}
  \address[DFKI]{DFKI Bremen, SKS}
  \address[UBremen]{Department of Mathematics and Computer Science, Universit\"at Bremen, Germany}
%  \thanks[ALL]{Work forms part of DFG-project \emph{Generic Algorithms and Complexity
%      Bounds in Coalgebraic Modal Logic} (SCHR 1118/5-1)}
  \thanks[myemail]{Email: \href{mailto:Daniel.Hausmann@dfki.de} {\texttt{\normalshape Daniel.Hausmann@dfki.de}}}
  \thanks[coemail]{Email: \href{mailto:Lutz.Schroeder@dfki.de} {\texttt{\normalshape Lutz.Schroeder@dfki.de}}}
\begin{abstract} 
  The generic modal reasoner CoLoSS covers a wide variety of logics
  ranging from graded and probabilistic modal logic to coalition logic
  and conditional logics, being based on a broadly applicable
  coalgebraic semantics and an ensuing general treatment of modal
  sequent and tableau calculi. Here, we present research into
  optimisation of the reasoning strategies employed in
  CoLoSS. Specifically, we discuss strategies of memoisation and
  dynamic programming that are based on the observation that short
  sequents play a central role in many of the logics under
  study. These optimisations seem to be particularly useful for the
  case of conditional logics, for some of which dynamic programming
  even improves the theoretical complexity of the algorithm. These
  strategies have been implemented in CoLoSS; we give a detailed
  comparison of the different heuristics, observing that in the
  targeted domain of conditional logics, a substantial speed-up can be
  achieved.
\end{abstract}
\begin{keyword}
  Coalgebraic modal logic, conditional logic, automated reasoning,
  optimisation, heuristics, memoizing, dynamic programming
\end{keyword}
\end{frontmatter}
\section{Introduction}\label{intro}

In recent decades, modal logic has seen a development towards semantic
heterogeneity, witnessed by an emergence of numerous logics that,
while still of manifestly modal character, are not amenable to
standard Kripke semantics. Examples include probabilistic modal
logic~\cite{FaginHalpern94}, coalition logic~\cite{Pauly02}, and
conditional logic~\cite{Chellas80}, to name just a few. The move
beyond Kripke semantics, mirrored on the syntactical side by the
failure of normality, entails additional challenges for tableau and
sequent systems, as the correspondence between tableaus and models
becomes looser, and in particular demands created by modal formulas
can no longer be discharged by the creation of single successor nodes.

This problem is tackled on the theoretical side by introducing the
semantic framework of coalgebraic modal
logic~\cite{Pattinson03,Schroder05}, which covers all logics mentioned
above and many more. It turns out that coalgebraic modal logic does
allow the design of generic reasoning algorithms, including a generic
tableau method originating from~\cite{SchroderPattinson09}; this
generic method may in fact be separated from the semantics and
developed purely syntactically, as carried out
in~\cite{PattinsonSchroder08b,PattinsonSchroder09a}. 

Generic tableau algorithms for coalgebraic modal logics, in particular
the algorithm described in~\cite{SchroderPattinson09}, have been
implemented in the reasoning tool
\COLOSS~\cite{CalinEA09}\footnote{available under
  \url{http://www.informatik.uni-bremen.de/cofi/CoLoSS/}}. As
indicated above, it is a necessary feature of the generic tableau
systems that they potentially generate multiple successor nodes for a
given modal demand, so that in addition to the typical depth problem,
proof search faces a rather noticeable problem of breadth. The search
for optimisation strategies to increase the efficiency of reasoning
thus becomes all the more urgent. Here we present one such strategy,
which is generally applicable, but particularly efficient in reducing
both depth and branching for the class of conditional logics. We
exploit a notable feature of this class, namely that many of the
relevant rules rely rather heavily on premises stating equivalence
between formulas; thus, conditional logics are a good candidate for
memoising strategies, applied judiciously to short sequents. We
describe the implementation of memoising and dynamic programming
strategies within \COLOSS, and discuss the outcome of various
comparative experiments.
 
\section{Generic Sequent Calculi for Coalgebraic Modal Logic}


Coalgebraic modal logic, originally introduced as a specification
language for coalgebras, seen as generic reactive
systems~\cite{Pattinson03}, has since evolved into a generic framework
for modal logic beyond Kripke semantics~\cite{CirsteaEA09}. The basic
idea is to encapsulate the branching type of the systems relevant for
the semantics of a particular modal logic, say probabilistic or
game-theoretic branching, in the choice of a set functor, the
signature functor (e.g.\ the distribution functor and the games
functor in the mentioned examples), and to capture the semantics of
modal operators in terms of so-called predicate liftings. For the
purposes of the present work, details of the semantics are less
relevant than proof-theoretic aspects, which we shall recall
presently. The range of logics covered by the coalgebraic approach is
extremely broad, including, besides standard Kripke and neighbourhood
semantics, e.g.\ graded modal logic~\cite{Fine72}, probabilistic modal
logic~\cite{FaginHalpern94}, coalition logic~\cite{Pauly02}, various
conditional logics equipped with selection function
semantics~\cite{Chellas80}, and many more.

Syntactically, logics are parametrised by the choice of a \emph{modal
  similarity type} $\Lambda$, i.e.\ a set of modal operators with
associated finite arities. This choice determines the set of formulas
$\phi,\psi$ via the grammar
\begin{equation*}
  \phi, \psi ::=  p \mid
  \phi \land \psi \mid \lnot \phi \mid \hearts(\phi_1, \dots, \phi_n)
\end{equation*}
where $\hearts$ is an $n$-ary operator in $\Lambda$. Examples are
$\Lambda=\{L_p\mid p\in[0,1]\cap\mathbb{Q}\}$, the unary operators
$L_p$ of probabilistic modal logic read `with probability at least
$p$'; $\Lambda=\{\gldiamond{k}\mid k\in\Nat\}$, the operators
$\gldiamond{k}$ of graded modal logic read `in more than $k$
successors'; $\Lambda=\{[C]\mid C\subseteq N\}$, the operators $[C]$
of coalition logic read `coalition $C$ (a subset of the set $N$ of
agents) can jointly enforce that'; and, for our main example here,
$\Lambda=\{\Rightarrow\}$, the \emph{binary} modal operator
$\Rightarrow$ of conditional logic read e.g.\ `if \dots then normally
\dots'.

Coalgebraic modal logic was originally limited to so-called
\emph{rank-$1$ logics} axiomatised by formulas with nesting depth of
modal operators uniformly equal to $1$~\cite{Schroder05}. It has since
been extended to the more general non-iterative
logics~\cite{SchroderPattinson08d} and to some degree to iterative
logics, axiomatised by formulas with nested
modalities~\cite{SchroderPattinson08PHQ}. The examples considered here
all happen to be rank-$1$, so we focus on this case. In the rank-$1$
setting, it has been shown~\cite{Schroder05} that all logics can be
axiomatised by \emph{one-step rules} $\phi/\psi$, where $\phi$ is
purely propositional and $\psi$ is a clause over formulas of the form
$\hearts(a_1,\dots,a_n)$, where the $a_i$ are propositional
variables. In the context of a sequent calculus, this takes the
following form~\cite{PattinsonSchroder08b}.
\begin{definition}
  If $S$ is a set (of formulas or variables) then $\Lambda(S)$ denotes
  the set $\lbrace \hearts(s_1, \dots, s_n) \mid \hearts \in \Lambda
  \mbox{ is $n$-ary, } s_1, \dots, s_n \in S \rbrace$ of formulas
  comprising exactly one application of a modality to elements of
  $S$. An \emph{$S$-sequent}, or just a \emph{sequent} in case
  $S=(\Lambda)$, is a finite subset of $S \cup \lbrace \neg A \mid A
  \in S \rbrace$. Then, a \emph{one-step rule}
  $\Gamma_1,\dots,\Gamma_n/\Gamma_0$ over a set $V$ of variables
  consists of $V$-sequents $\Gamma_1,\dots,\Gamma_n$, the
  \emph{premises}, and a $\Lambda(S)$-sequent $\Gamma_0$, the
  \emph{conclusion}. A \emph{goal} is a set of sequents, typically
  arising as the set of instantiated premises of a rule application.
\end{definition}
\noindent A given set of one-step rules then induces an instantiation
of the \emph{generic sequent calculus}~\cite{PattinsonSchroder08b}
which is given by a set of rules $\mathcal{R}_{sc}$ consisting of the
finishing and the branching rules $\mathcal{R}^b_{sc}$ (i.e.\ rules
with no premise or more than one premise), the linear rules
$\mathcal{R}^l_{sc}$ (i.e.\ rules with exactly one premise) and the
modal rules $\mathcal R^m_{sc}$, i.e.\ the given one-step rules. The
finishing and the branching rules are presented in
Figure~\ref{fig:branching} (where $\top=\neg\bot$ and $p$ is an atom),
the linear rules are shown in Figure~\ref{fig:linear}. So far, all
these rules are purely propositional. As an example for a set of modal
one-step rules, consider the modal rules $\mathcal R^m_{sc}$ of the
standard modal logic \textbf{K} as given by Figure~\ref{fig:modalK}.
\begin{figure}[!h]
  \begin{center}
    \begin{tabular}{| c c c |}
    \hline
      & & \\[-5pt]
      (\textsc {$\neg$F}) \inferrule{ }{\Gamma, \neg\bot} &
      (\textsc {Ax}) \inferrule{ }{\Gamma, p, \neg p} &
      (\textsc {$\wedge$}) \inferrule{\Gamma, A \\ \Gamma, B}{\Gamma, A\wedge B} \\[-5pt]
      & & \\
    \hline
    \end{tabular}
  \end{center}
  \caption{The finishing and the branching sequent rules $\mathcal{R}^b_{sc}$}
  \label{fig:branching}
\end{figure}
\begin{figure}[!h]
  \begin{center}
    \begin{tabular}{| c c |}
    \hline
      & \\[-5pt]
      (\textsc {$\neg\neg$})\inferrule{\Gamma, A}{\Gamma, \neg\neg A} &
      (\textsc {$\neg\wedge$}) \inferrule{\Gamma, \neg A, \neg B}{\Gamma, \neg(A\wedge B)} \\[-5pt]
      & \\
    \hline
    \end{tabular}
  \end{center}
  \caption{The linear sequent rules $\mathcal{R}^l_{sc}$}
  \label{fig:linear}
\end{figure}

\begin{figure}[!h]
  \begin{center}
    \begin{tabular}{| c |}
    \hline
      \\[-5pt]
      (\textsc {\textbf{K}})\inferrule{\neg A_1, \ldots , \neg A_n, A_0}
                      {\Gamma, \neg \Box A_1,\ldots,\neg \Box A_n, \Box A_0 } \\[-5pt]
      \\
    \hline
    \end{tabular}
  \end{center}
  \caption{The modal rule of \textbf{K}}
  \label{fig:modalK}
\end{figure}
\noindent This calculus has been shown to be complete under suitable
coherence assumptions on the rule set and the coalgebraic semantics,
provided that the set of rules \emph{absorbs cut and contraction} in a
precise sense~\cite{PattinsonSchroder08b}. We say that a formula is
\emph{provable} if it can be derived in the relevant instance of the
sequent calculus; the algorithms presented below are concerned with
the \emph{provability problem}, i.e.\ to decide whether a given
sequent is provable. This is made possible by the fact that the
calculus does not include a cut rule, and hence enables automatic
proof search. For rank-1 logics, proof search can be performed in
$\mathit{PSPACE}$ under suitable assumptions on the representation of
the rule set~\cite{SchroderPattinson09,PattinsonSchroder08b}. The
corresponding algorithm has been implemented in the system
CoLoSS~\cite{CalinEA09} which remains under continuous
development. Particular attention is being paid to finding heuristic
optimisations to enable practically efficient proof search, as
described here.

Our running example used in the presentation of our optimisation
strategies below is conditional logic as mentioned above. The most
basic conditional logic is $\textbf{CK}$~\cite{Chellas80} (we shall
consider a slightly extended logic below), which is characterised by
assuming normality of the right-hand argument of the non-monotonic
conditional $\Rightarrow$, but only replacement of equivalents in the
left-hand arguments. Absorption of cut and contraction requires a
unified rule set consisting of the rules
\begin{equation*}
  \infrule{A_0=A_1;\;\dots;\; A_n=A_0\quad \neg B_1,\dots,\neg B_1,B_0}
  {\neg(A_1\Rightarrow B_1),\dots,\neg(A_n\Rightarrow B_n),(A_0\Rightarrow B_0)}
\end{equation*}
with $A=C$ abbreviating the pair of sequents $\neg A,C$ and $\neg
C,A$.  This illustrates two important points that play a role in the
optimisation strategies described below. First, the rule has a large
branching degree both existentially and universally -- we have to
check that there \emph{exists} a rule proving some sequent such that
\emph{all} its premises are provable, and hence we have to guess one
of exponentially many subsequents to match the rule and then attempt
to prove linearly many premises; compare this to linearly many rules
and a constant number of premises per rule (namely, $1$) in the case
of $\textbf{K}$ shown above. Secondly, many of the premises are short
sequents. This will be exploited in our memoisation strategy. We note
that \emph{labelled} sequent calculi for conditional logics have been
designed previously~\cite{OlivettiEA07} and have been implemented in
the CondLean prover~\cite{OlivettiPozzato03}; contrastingly, our
calculus is unlabelled, and turns out to be conceptually simpler. The
comparison of the relative efficiency of labelled and unlabelled
calculi remains an open issue for the time being.

\section{The Algorithm}

According to the framework introduced above, we now devise a generic
algorithm to decide the provability of formulae, which is easily
instantiated to specific logic by just implementing the relevant modal
rules.


Algorithm~\ref{alg:seq} makes use of the sequent rules in the
following manner: In order to show the \emph{provability} of a formula
$\phi$, the algorithm starts with the sequent $\{\phi\}$ and just
keeps trying to apply all of the sequent rules in $R_{sc}$ to it,
giving precedence to the linear rules.  Below, we refer to a sequent
as \emph{open} if it has not yet been checked for provability. Under
suitable tractability assumptions on the rule set as
in~\cite{SchroderPattinson09,PattinsonSchroder08b}, this algorithm
realises the theoretical upper bound $\mathit{PSPACE}$. It is the
starting point of the proof search algorithms employed in CoLoSS,
being essentially a sequent reformulation of the algorithm described
in~\cite{CalinEA09}; optimisations of this algorithm are the subject
of the present work. 

%\begin{algorithm}[h]
%\fbox{\parbox{\myboxwidth}{
\begin{alg}(Decide provability of a sequent $\Gamma$)
\begin{upshape}
 \begin{algenumerate}
   \item Try all possible applications of rules from $R_{sc}$ to $\Gamma$, 
     giving precedence to linear rules. For every such rule application,
     perform the following steps, and answer `provable' in case these steps
     succeed for one of the rule applications.
   \item Let $\Lambda$ denote the  set of premises  arising from the rule 
     application.
   \item Check recursively that every sequent in $\Lambda$ is provable.
  \end{algenumerate}
\end{upshape}
\label{alg:seq}
\end{alg}
%}}
%\end{algorithm}

%\begin{proposition}
%\begin{upshape}
%Let $\mathcal{R}_{sc}$ be strictly one-step complete, closed under contraction,
%and PSPACE-tractable. Then Algorithm~\ref{alg:seq} is sound and complete w.r.t. provability
%and it is in PSPACE.
%\end{upshape}
%\end{proposition}%

%\begin{proof}
%We just note, that Algorithm~\ref{alg:seq} is an equivalent implementation of the algorithm proposed
%in~\cite{SchroderPattinson09}. For more details, refer to ~\cite{SchroderPattinson09}, Theorem 6.13.
%\end{proof}

\subsection{The conditional logic instance}

The genericity of the introduced sequent calculus allows us to easiliy
create instantiations of Algorithm~\ref{alg:seq} for a large variety
of modal logics: By setting $\mathcal R^m_{sc}$ for instance to the
rule shown in Figure~\ref{fig:modalCKCEM},
% (where $A_a = A_b$ is
%shorthand for $A_a\rightarrow A_b\wedge A_b\rightarrow A_a$), 
we obtain an algorithm for deciding provability (and satisfiability)
of conditional logic. We restrict ourselves to the examplary
conditional logic \textbf{CKCEM} for the remainder of this section;
slightly adapted versions of the optimisation will work for other
conditional logics. \textbf{CKCEM} is characterised by the additional
axioms of \emph{conditional excluded middle} $(A\Rightarrow
B)\lor(A\Rightarrow\neg B)$, which to absorb cut and contraction is
integrated in the rule for \textbf{CK} as follows.

\begin{figure}[h!]
  \begin{center}
    \begin{tabular}{| c |}
    \hline
      \\[-5pt]
      (\textsc {\textbf{CKCEM}})\inferrule{A_0 = \ldots = A_n \\ B_0,\ldots, B_j,\neg B_{j+1},\ldots,\neg B_n}
                      {\Gamma, (A_0\Rightarrow B_0),\ldots,(A_j\Rightarrow B_j),
                      \neg(A_{j+1}\Rightarrow B_{j+1}),\ldots,\neg(A_n\Rightarrow B_n) } \\[-5pt]
      \\
    \hline
    \end{tabular}
  \end{center}
  \caption{The modal rule $\textbf{CKCEM}$ of conditional logic}
  \label{fig:modalCKCEM}
\end{figure}
\noindent It has been shown in~\cite{PattinsonSchroder09a} that the
complexity of \textbf{CKCEM} is $\mathit{coNP}$, using a dynamic
programming approach in the spirit of~\cite{Vardi89}; in fact, this
was the original motivation for exploring the optimisation strategies
pursued here.

In the following, we use the notions of \emph{conditional antecedent} and
\emph{conditional consequent} to refer to the parameters of the modal operator of
conditional logic. 

In order to decide using Algorithm~\ref{alg:seq} whether there is an
instance of the modal rule of conditional logic which can be applied
to the actual current sequent, it is necessary to create a preliminary
premise for each possible combination of equalities of all the
premises of the modal operators in this sequent. This results in
$2^n-1$ new premises for a sequent with $n$ top-level modal
operators. 
\begin{example}\label{ex:cond}
For the sequent $\Gamma=\{(A_0\Rightarrow B_0),
(A_1\Rightarrow B_1), (A_2\Rightarrow B_2)\}$, there are $2^3=8$
possible instances of the rule to be tried, corresponding to the
non-empty subsequents of the goal; the premise to be checked for
$I\subseteq\{1,2,3\}$ consists of $A_i=A_j$ for $i,j\in I$, and
$\{B_i\mid i\in I\}$.
\end{example}
\noindent It seems to be a more intelligent approach to first
partition the set of all antecedents of the top-level modal operators
in the current sequent into equivalence classes with respect to
logical equality.  This partition allows for a significant reduction
both of the number of rules to be tried and of the number of premises
to be actually proved for each rule.

\begin{example}
Consider again the sequent from Example~\ref{ex:cond}. By using the examplary
knowledge that $A_0=A_1$, $A_1\neq A_2$ and $A_0\neq A_2$, it is immediate
that there are just two reasonable instations of the modal rule, leading
to the two premises $\{\{B_0,B_1\}\}$ and $\{\{B_2\}\}$. For the
first of these two premises, note that it is not necessary to show the
equivalence of $A_0$ and $A_1$ again.
\end{example}


\noindent In the case of conditional logic, observe the following:
Since the modal antecedents that appear in a formula are not being
changed by any rule of the sequent calculus, it is possible to extract
all possibly relevant antecedents of a formula even \emph{before} the
actual sequent calculus is applied. This allows us to first compute
the equivalence classes of all the relevant antecedents and then feed
this knowledge into the actual sequent calculus, as illustrated next.

\section{The Optimisation}

\begin{definition}
A conditional antecedent of \emph{modal nesting depth} $i$ is a
conditional antecedent which contains at least one antecedent of
modal nesting depth $i-1$ and which contains no antecedent 
of modal nesting depth greater than $i-1$. A
conditional antecedent of nesting depth 0 is an antecedent
that does not contain any further modal operators.
Let $p_i$ denote the set of all conditional antecedents of modal
nesting depth $i$. Further, let $\prems(n)$ denote the set of all
conditional antecedents of modal nesting depth at most $n$ (i.e.
$\prems(n)=\bigcup_{j=1..n}^{} p_j$).
Finally, let $depth(\phi)$ denote the maximal modal nesting in
the formula $\phi$.
\end{definition}

%\begin{example}
%In the formula $\phi=((p_0\Rightarrow p_1) \wedge ((p_2\Rightarrow p_3)\Rightarrow p_4))
%\Rightarrow (p_5\Rightarrow p_6)$,
%the premise $((p_0\Rightarrow p_1) \wedge ((p_2\Rightarrow p_3)\Rightarrow p_4))$ has a
%nesting depth of 2 (and not 1), $(p_0\Rightarrow p_1)$ and $(p_2\Rightarrow p_3)$ both
%have a nesting depth of 1, and finally $p_0$, $p_2$ and $p_4$ all have a nesting depth of 0.
%Furthermore, $\prems(1)=\{p_0,p_2,p_4,(p_0\Rightarrow p_1),(p_2\Rightarrow p_3)\}$
%and $depth(\phi)=3$.
%\end{example}

\begin{definition}
  A set $\mathcal{K}$ of sequents together with a function
  $\eval:\mathcal{K}\rightarrow \{\top,\bot\}$ is called
  a \emph{knowledge base}.
\end{definition}

\noindent We may now construct an optimized algorithm which allows us
to decide provability (and satisfiability) of formulae more
efficiently in some cases. The optimized algorithm is constructed from
two functions (namely from the actual proving function and from the
so-called \emph{pre-proving} function):

%\begin{algorithm}[h] 
\begin{alg}\label{alg:opt}
  (Decide provability of $\phi$ using the knowledge
  base $(\mathcal{K},\eval)$)
{\upshape
 \begin{algenumerate}
 \item\label{step:look-up} If $\Gamma\in\mathcal{K}$, answer `provable' if
   $\eval(\Gamma)=\top$, else `unprovable'. Otherwise:
 \item\label{step:rule} Try all possible applications of rules from
   $\mathcal{RO}_{sc}$ to $\Gamma$, giving precedence to linear rules,
   where $\mathcal{RO}_{sc}$ is an optimised set of rules taking into
   account the knowledge base, as explained below. For every such rule
   application, perform the following steps, and answer `provable' in
   case these steps succeed for one of the rule applications.
   \item Let $\Lambda$ denote the  set of premises  arising from the rule 
     application.
   \item Check recursively that every sequent in $\Lambda$ is provable.
  \end{algenumerate}
}
\label{alg:optSeq}
\end{alg}
%\end{algorithm}

\noindent Algorithm~\ref{alg:optSeq} is very similar to
Algorithm~\ref{alg:seq} but relies on the knowledge base passed to it
and moreover uses a modified set of rules $\mathcal{RO}_{sc}$. The set
of rules $\mathcal{RO}_{sc}$ makes appropriate use of the knowledge
base. It is obtained from $\mathcal{R}_{sc}$ by replacing the modal
rule from Figure~\ref{fig:modalCKCEM} with the modified modal rule
from Figure~\ref{fig:modalCKCEMm}. The point is that the premises
$A_0=\dots=A_n$ are replaced by side conditions representing lookup in
the knowledge base. This improves on standard memoising as embodied by
the lookup operation in step~\ref{step:look-up} of the above algorithm
in that existential branching over potentially applicable rules is
reduced: the rule does not even match the target sequent unless the
equivalence premises are already in the knowledge base. This is still
a complete system due to the way memoising is organised, as explained
below.

 \begin{figure}[h!]
  \begin{center}
    \begin{tabular}{| c |}
    \hline
      \\[-5pt]
     (\textsc {\textbf{CKCEM}$^m$})\inferrule{B_0,\ldots, B_j,\neg B_{j+1},\ldots,\neg B_n}
                     {\Gamma, (A_0\Rightarrow B_0),\ldots,(A_j\Rightarrow B_j),
                     \neg(A_{j+1}\Rightarrow B_{j+1}),\ldots,\neg(A_n\Rightarrow B_n) } \\[-5pt]
     \\
     \hfill ($\bigwedge{}_{i,j\in\{1..n\}}{\eval(A_i=A_j)=\top}$)\\
   \hline
   \end{tabular}
 \end{center}
 \caption{The modified modal rule \textbf{CKCEM}$^m$ of conditional logic}
 \label{fig:modalCKCEMm}
\end{figure}

%\begin{algorithm}[h]

\noindent The knowledge base used in Algorithm~\ref{alg:opt} is
computed in Algorithm~\ref{alg:preprove}. The algorithm proceeds by
dynamic programming with stages corresponding to modal nesting depth,
in the spirit of~\cite{Vardi89}.  Thus, in order to show the
equivalence of two conditional antecedents of nesting depth at most
$i$, we assume that the equivalences $\mathcal{K}_{i}$ between modal
antecedents of nesting depth less than $i$ have already been computed
and the result is stored in $\eval_i$; hence, two antecedents are
equal, if their equivalence is provable by Algorithm~\ref{alg:optSeq}
using only the knowledge base $(\mathcal{K}_{i},\eval_i)$.
\begin{alg}
\begin{upshape}
  Step 1: Take a formula $\phi$ as input. Set $i=0$, $\mathcal{K}_0=\emptyset$, $\eval_0=\emptyset$.\\
  Step 2: Generate the set $\prems_i$ of all conditional antecedents of $\phi$
  of nesting depth at most $i$. If $i<depth(\phi)$ continue
  with Step 3, else set $\mathcal{K}=\mathcal{K}_{i-1}, \eval=\eval_{i-1}$ and continue with Step 4.\\
  Step 3: Let $eq_i$ denote the set of all equalities $A_a = A_b$ for different
  formulae $A_a,A_b\in \prems_i$. Compute
  Algorithm~\ref{alg:optSeq} ($\psi$, $(\mathcal{K}_i,\eval_i)$) for all $\psi\in eq_i$.
  Set $\mathcal{K}_{i+1} = eq_i$, set $i = i + 1$. For each equality $\psi\in eq_i$,
  set $\eval_{i+1}(\psi)=\top$ if the result of Algorithm~\ref{alg:optSeq} was `provable'
  and $\eval_{i+1}(\psi)=\bot$ otherwise. Continue with Step 2.\\
  Step 4: Call Algorithm~\ref{alg:optSeq} ($\phi$, $(\mathcal{K},\eval)$) and return its return
  value as result.
\label{alg:preprove}
\end{upshape}
\end{alg}
%\end{algorithm}


\subsection{Treatment of Requisite Equivalences Only}

Since Algorithm~\ref{alg:preprove} tries to show the logical equivalence of any combination
of two conditional antecedents that appear in $\phi$, it will have worse completion
time than Algorithm~\ref{alg:seq} on many formulae:

\begin{example}
Consider the formula 
\begin{quote}
$\phi=(((p_0\Rightarrow p_1)\Rightarrow p_2)\Rightarrow p_4)\vee
(((p_5\Rightarrow p_6)\Rightarrow p_7)\Rightarrow p_8)$.
\end{quote}
Algorithm~\ref{alg:preprove} will not only
try to show the necessary equivalences between the pairs
$(((p_0\Rightarrow p_1)\Rightarrow p_2), ((p_5\Rightarrow p_6)\Rightarrow p_7))$,
$((p_0\Rightarrow p_1), (p_5\Rightarrow p_6))$ and $(p_0,p_5)$, but it will
also try to show equivalences between any two conditional antecedents (e.g. $(p_0,
(p_5\Rightarrow p_6))$), even though these equivalences will not be needed
during the execution of Algorithm~\ref{alg:optSeq}.
\end{example}

Based on this observation it is possible to assign a category to each pair of
antecedents that appear in it:

\begin{definition}
The \emph{paths (in $\phi$)} to a conditional antecedent $\psi$ describe the orders
of modal arguments through which $\psi$ is reached, when starting from the root
of $\phi$:
The path to a top-level antecedent is just $\{1\}$. If $\psi$ does not appear as
antecedent on the topmost level of $\phi$, the path to it is $\{1\}$ prepended to the set
of paths to $\psi$ in any top-level conditional antecedent of $\phi$ together with $\{0\}$
prepended to the set of paths to $\psi$ in any top-level conditional consequent of $\phi$.
\end{definition}

\begin{example}
Consider the formula $\phi=(p_0\Rightarrow p_2)\Rightarrow ((p_0\Rightarrow p_1)\Rightarrow p_3)$. Then the path to
$(p_0\Rightarrow p_2)$ is $\{1\}$, whereas the path to $(p_0\Rightarrow p_1)$ is $\{01\}$. The paths
to $p_0$ are $\{11,011\}$.
\end{example}

\begin{definition}
Let $A$ and $B$ be two conditional antecedents. $A$ and $B$ are called \emph{connected (in $\phi$)} if
at least one path to $A$ is also a path to $B$ (and hence vice-versa). If no path to $A$ is a path to $B$,
the two antecedents are said to be \emph{independent}.
\end{definition}

Since two independent conditional antecedents will never appear in the scope of the
same application of the modal rule, it is in no case necessary to show (or
refute) the logical equivalence of independent conditional antecedents. Hence it suffices to focus
our attention to the connected conditional antecedents. It is then obvious that
any possibly requisite equivalence and its truth-value are allready included in $(K,\eval)$
when the main proving is induced. On the other hand, we have to be aware that it
may be the case, that we show equivalences of antecedents which are in fact not needed
(since antecedents may indeed be connected and still it is possible that they never appear together in an application
of the modal rule - this is the case whenever two preceeding antecedents are not logically equivalent).

As result of these considerations, we devise Algorithm~\ref{alg:optPreprove},
an improved version of Algorithm~\ref{alg:preprove}. The only difference is
that before proving any equivalence, Algorithm~\ref{alg:optPreprove} checks
whether the current pair of conditional antecedents is actually connected;
only then does it treat the equivalence. Hence independent pairs of antecedents
remain untreated
%\begin{algorithm}[h]
\begin{alg}
\begin{upshape}
  Step 1: Take a formula $\phi$ as input. Set $i=0$, $\mathcal{K}_0=\emptyset$, $\eval_0=\emptyset$.\\
  Step 2: Generate the set $\prems_i$ of all conditional antecedents of $\phi$
  of nesting depth at most $i$. If $i<depth(\phi)$ continue
  with Step 3, else set $\mathcal{K}=\mathcal{K}_{i-1}, \eval=\eval_{i-1}$ and continue with Step 4.\\
  Step 3: Let $eq_i$ denote the set of all equalities $A_a = A_b$ for different and not independent
  pairs of formulae $A_a,A_b\in \prems_i$. Compute
  Algorithm~\ref{alg:optSeq} ($\psi$, $(\mathcal{K}_i,\eval_i)$) for all $\psi\in eq_i$.
  Set $\mathcal{K}_{i+1} = eq_i$, set $i = i + 1$. For each equality $\psi\in eq_i$,
  set $\eval_{i+1}(\psi)=\top$ if the result of Algorithm~\ref{alg:optSeq} was `provable'
  and $\eval_{i+1}(\psi)=\bot$ otherwise. Continue with Step 2.\\
  Step 4: Call Algorithm~\ref{alg:optSeq} ($\phi$, $(\mathcal{K},\eval)$) and return its 
  return value as result.
\label{alg:optPreprove}
\end{upshape}
\end{alg}
%\end{algorithm}

\section{Implementation}

The proposed optimized algorithms have been implemented (using the programming
language Haskell) as part of the generic coalgebraic modal logic satisfiability
solver (\COLOSS\footnote{As already mentioned above, more information about \COLOSS,
a web-interface to the tool and 
the tested benchmarking formulae can be found at \url{http://www.informatik.uni-bremen.de/cofi/CoLoSS/}}).
\COLOSS~provides the general coalgebraic framework in which the generic
sequent calculus is embedded. It is easily possible to instantiate this generic sequent
calculus to specific modal logics, one particular example being conditional logic.
The matching function for conditional logic in \COLOSS~was hence adapted in order to realize
the different optimisations (closely following Algorithms~\ref{alg:seq},~\ref{alg:preprove} and
~\ref{alg:optPreprove}), so that \COLOSS~now provides an efficient algorithm for
deciding the provability (and satisfiability) of conditional logic formulae.

\subsection{Comparing the Proposed Algorithms}
\label{sec:bench}

In order to show the relevance of the proposed optimisations, we devise several classes
of conditional formulae. Each class has a characteristic general shape, defining its
complexity w.r.t. different parts of the algorithms and thus exhibiting specific
advantages or disadvantages of each algorithm:

\begin{itemize}
\item The formula \verb|bloat(|$i$\verb|)| is a full binary tree of depth $i$ (containing $2^i$ pairwise logically
inequivalent atoms and $2^i-1$ modal antecedents):
\begin{quote}
\verb|bloat(|$i$\verb|)| = $($\verb|bloat(|$i-1$\verb|)|$)\Rightarrow($\verb|bloat(|$i-1$\verb|)|)\\
\verb|bloat(|$0$\verb|)| = $p_{rand}$
\end{quote}
Formulae from this class should show the problematic performance of Algorithm~\ref{alg:preprove} whenever
a formula contains many modal antecedents which appear at different depths. A comparison of the different
algorithms w.r.t. formulae \verb|bloat(|$i$\verb|)| is depicted in Figure~\ref{fig:benchBloat}.
Since Algorithm~\ref{alg:preprove} does not check whether pairs of modal antecedents are independent or connected,
it performs considerably worse than Algorithm~\ref{alg:optPreprove} which only attempts to prove the logical
equivalence of formulae which are not independent. Algorithm~\ref{alg:seq} has the best performance in this
extreme case, as it only has to consider pairs of modal antecedents which actually appear during the course
of a proof. This is the price to pay for the optimisation by dynamic programming.
\end{itemize}

\begin{figure}[!h]
  \begin{center}
\begin{tabular}{| l | r | r | r |}
\hline
$i$ & Algorithm~\ref{alg:seq} & Algorithm~\ref{alg:preprove} & Algorithm~\ref{alg:optPreprove}  \\
\hline
 1 & 0.008s & 0.009s & 0.010s\\
 2 & 0.008s & 0.011s & 0.010s\\
 3 & 0.009s & 0.028s & 0.014s\\
 4 & 0.010s & 0.233s & 0.022s\\
 5 & 0.011s & 2.840s & 0.087s\\
 6 & 0.013s & 33.476s & 0.590s\\
 7 & 0.019s & 402.239s & 4.989s\\
 \hline
 \end{tabular}
  \end{center}
  \caption{Results for bloat($i$)}
  \label{fig:benchBloat}
\end{figure}

\begin{itemize}
\item The formula \verb|conjunct(|$i$\verb|)| is just an $i$-fold conjunction of a specific formula $A$:
\begin{quote}
\verb|conjunct(|$i$\verb|)| = $A_1\wedge\ldots\wedge A_i$\\
$A=(((p_1\vee p_0)\Rightarrow p_2)\vee((p_0\vee p_1)\Rightarrow p_2))\vee\neg(((p_0\vee p_1)\Rightarrow p_2)\vee((p_1\vee p_0)\Rightarrow p_2))$)
\end{quote}
This class consists of formulae which contain logically (but not sytactically) equivalent antecedents.
As $i$ increases, so does the amount of appearances of identical modal antecedents in different positions
of the considered formula. A comparison of the different algorithms w.r.t. formulae \verb|conjunct(|$i$\verb|)| is depicted in
Figure~\ref{fig:benchConjunct}. It is obvious that the optimized algorithms perform considerably better than the unoptimized
Algorithm~\ref{alg:seq}. The reason for this is, that Algorithm~\ref{alg:seq} repeatedly proves equivalences between the same
pairs of modal antecedents. The optimized algorithms on the other hand are equipped with knowledge about the modal antecedents,
so that these equivalences have to be proved only once. However, even the runtime of the optimized algorithms is exponential in $i$,
due to the exponentially increasing complexity of the underlying propositional formula. Note that the use of propositional taulogies (such as
$A \leftrightarrow (A\wedge A) $ in this case) would help to greatly reduce the computing time for \verb|conjunct(|$i$\verb|)|.
Optimisation of propositional reasoning is not the scope of this paper though, thus we devise the following examplary class of formulae
(for which propositional tautologies would not help).
\end{itemize}

\begin{figure}[!h]
  \begin{center}
\begin{tabular}{| l | r | r | r |}
\hline
$i$ & Algorithm~\ref{alg:seq} & Algorithm~\ref{alg:preprove} & Algorithm~\ref{alg:optPreprove}  \\
\hline
 1 & 0.012s & 0.013s & 0.012s\\
 2 & 0.021s & 0.015s & 0.014s\\
 3 & 0.681s & 0.021s & 0.021s\\
 4 & 131.496s & 0.048s & 0.048s\\
 5 & $>$600.000s & 0.199s & 0.201s\\
 6 & $\gg$600.000s & 1.152s & 1.161s\\
 7 & $\gg$600.000s & 8.746s & 8.667s\\
 8 & $\gg$600.000s & 74.805s & 75.595s\\
 9 & $\gg$600.000s & $>$600.000s & $>$600.000s\\
 \hline
 \end{tabular}
  \end{center}
  \caption{Results for conjunct($i$)}
  \label{fig:benchConjunct}
\end{figure}

\begin{itemize}
\item The formula \verb|explode(|$i$\verb|)| contains equivalent but not syntactically
equal and interchangingly nested modal antecedents of depth at most $i$:
\begin{quote}
\verb|explode(|$i$\verb|)| = $X^i_1\vee\ldots\vee X^i_i$\\
$X^i_1=(A^i_1\Rightarrow(\ldots(A^i_i\Rightarrow (c_1\wedge\ldots\wedge c_i))\ldots))$\\
$X^i_j=(A^i_{j\bmod i}\Rightarrow(\ldots(A^i_{(j+(i-1))\bmod i}\Rightarrow \neg c_j)\ldots))$\\
$A^i_j=p_{j \bmod i}\wedge\ldots\wedge p_{(j+(i-1)) \bmod i}$
\end{quote}
This class contains complex formulae for which the unoptimized
algorithm should not be efficient any more: Only the combined
knowledge about all appearing modal antecedents $A^i_j$ allows the
proving algorithm to reach all modal consequents $c_n$, and only the
combined sequent $\{(c_1\wedge\ldots\wedge c_i),\neg c_1,\ldots,\neg
c_i\}$ (containing every appearing consequent) is provable. For
formulae from this class (specifically designed to show the advantages
of optimization by dynamic programming), the optimized algorithms
vastly outperform the unoptimized algorithm (see
Figure~\ref{fig:benchExplode}).
\end{itemize}

\begin{figure}[!h]
  \begin{center}
\begin{tabular}{| l | r | r | r |}
\hline
$i$ & Algorithm~\ref{alg:seq} & Algorithm~\ref{alg:preprove} & Algorithm~\ref{alg:optPreprove}  \\
\hline
 1 & 0.009s & 0.008s & 0.009s\\
 2 & 0.011s & 0.010s & 0.010s\\
 3 & 0.028s & 0.012s & 0.014s\\
 4 & 0.268s & 0.018s & 0.018s\\
 5 & 4.555s & 0.025s & 0.027s\\
 6 & 10.785s & 0.035s & 0.039s\\
 7 & $\gg$600.000s & 0.054s & 0.060s\\
 8 & $\gg$600.000s & 0.079s & 0.089s\\
 9 & $\gg$600.000s & 0.122s & 0.140s\\
 \hline
 \end{tabular}
  \end{center}
  \caption{Results for explode($i$)}
  \label{fig:benchExplode}
\end{figure}

The tests were conducted on a Linux PC (Dual Core AMD Opteron 2220S
(2800MHZ), 16GB RAM).  It is obvious that a significant increase of
performance may be obtained through the proposed optimisations. In
general, the implementation of the proposed algorithms in the generic
reasoner \COLOSS has at least comparable performance as dedicated
conditional logic provers such as CondLean; a direct comparison is
presently made difficult by the fact that the benchmarking formulas
used to evaluate CondLean are not listed explicitly
in~\cite{OlivettiPozzato03}.

\section{Generalized Optimisation}

As previously mentioned, the demonstrated optimisation is not restricted to the
case of conditional
modal logics. 

\begin{definition}
If $\Gamma$ is a sequent, we denote the set of all arguments of
top-level modalities from $\Gamma$ by $arg(\Gamma)$.
A \emph{short sequent} is a sequent which consists of just one formula which
itself is a propositional formula over a fixed maximal number of modal arguments
from $arg(\Gamma)$. In the following, we fix the maximal number of modal arguments
in short sequents to be 2.
\end{definition}

The general method of the optimisation then takes the following form: 
Let $S_1,\ldots S_n$ be short sequents and assume that there is
a (w.r.t the considered modal logic) sound instance of the generic rule which
is depicted in Figure~\ref{fig:modalOpt} (where $\mathcal{S}$ is a set of any
sequents).

\begin{figure}[!h]
  \begin{center}
    \begin{tabular}{| c |}
    \hline
      \\[-5pt]
(\textsc {\textbf{Opt}}) \inferrule{ S_1 \\ \ldots \\ S_n \\ \mathcal{S} }
                      { \Gamma } \\[-5pt]
      \\
    \hline
    \end{tabular}
  \end{center}
  \caption{The general rule-scheme to which the optimisation may be applied}
  \label{fig:modalOpt}
\end{figure}

Then we devise a final version (Algorithm~\ref{alg:genOptPreprove}) of the
optimized algorithm: Instead of considering only equivalences of conditional antecedents
for pre-proving, we now extend our attention to any short sequents over any modal arguments.

\begin{algorithm}[h]
\begin{alg}
\begin{upshape}
  Step 1: Take a formula $\phi$ as input. Set $i=0$, $\mathcal{K}_0=\emptyset$, $\eval_0=\emptyset$.\\
  Step 2: Generate the set $args_i$ of all modal arguments of $\phi$
  which have nesting depth at most $i$. If $i<depth(\phi)$ continue
  with Step 3, else set $\mathcal{K}=\mathcal{K}_{i-1}, \eval=\eval_{i-1}$ and continue with Step 4.\\
  Step 3: Let $seq_i$ denote the set of all short sequents of form $S_i$ (where $S_i$ is a sequent
  from the premise of rule (\textbf{Opt})) over at most two formulae
  $A_a,A_b\in args_i$. Compute Algorithm~\ref{alg:optSeq} ($\psi$, $(\mathcal{K}_i,\eval_i)$) for all
  $\psi\in seq_i$. Set $\mathcal{K}_{i+1} = seq_i$, set $i = i + 1$. For each short sequent
  $\psi\in seq_i$, set $\eval_{i+1}(\psi)=\top$ if the result of Algorithm~\ref{alg:optSeq} was
  `provable' and $\eval_{i+1}(\psi)=\bot$ otherwise. Continue with Step 2.\\
  Step 4: Call Algorithm~\ref{alg:optSeq} ($\phi$, $(\mathcal{K},\eval)$) and return its return value
  as result.
\end{upshape}
\label{alg:genOptPreprove}
\end{alg}
\end{algorithm}

This new Algorithm~\ref{alg:genOptPreprove} may then be used to decide provability of formulae,
where the employed ruleset has to be extended by the generic modified rule given by Figure~\ref{fig:modModalOpt}.

\begin{figure}[!h]
  \begin{center}
    \begin{tabular}{| c |}
    \hline
      \\[-5pt]
      (\textsc {\textbf{Opt}$^m$}) \inferrule{ \eval(S_1)=\top \\ \ldots \\ \eval(S_n)=\top \\ \mathcal{S} }
                      { \Gamma } \\[-5pt]
      \\
    \hline
    \end{tabular}
  \end{center}
  \caption{The general optimized rule}
  \label{fig:modModalOpt}
\end{figure}

\begin{example}
The following two cases are instances of the generic optimisation:
\begin{enumerate}
\item (Classical modal Logics / Neighbourhood Semantics) Let $\Gamma = \{\Box A = \Box B\}$,
$n=1$, $S_1=\{A=B\}$ and $\mathcal{S}=\emptyset$. Algorithm~\ref{alg:genOptPreprove}
may be then applied whenever the following congruence rule is sound in the considered
logic:
\begin{quote}
\begin{center}
      (\textsc {\textbf{Opt}$_{Cong}$}) \inferrule{ A=B }
                      { \Box A = \Box B }
  \end{center}
\end{quote}
The according modified version of this rule is as follows:
\begin{quote}
\begin{center}
      (\textsc {\textbf{Opt}$^m_{Cong}$}) \inferrule{ {\eval(A=B)=\top} }
                      { \Box A = \Box B }
  \end{center}
\end{quote}
\item (Monotone modal logics) By setting $\Gamma = \{\Box A \rightarrow \Box B\}$,
$n=1$, $S_1=\{A\rightarrow B\}$ and $\mathcal{S}=\emptyset$, we may instantiate
the generic algorithm to the case of modal logics which are monotone w.r.t. their
modal operator. So assume the following rule to be sound in  the considered modal
logic:
\begin{quote}
\begin{center}
      (\textsc {\textbf{Opt}$_{Mon}$}) \inferrule{ A\rightarrow B }
                      { \Box A \rightarrow \Box B }
  \end{center}
\end{quote}
The according modified version of this rule is as follows:
\begin{quote}
\begin{center}
      (\textsc {\textbf{Opt}$^m_{Mon}$}) \inferrule{ {\eval(A\rightarrow B)=\top} }
                      { \Box A \rightarrow \Box B }
  \end{center}
\end{quote}

In the case that (\textbf{Opt}$_{Mon}$) is the only modal rule in the
considered logic (i.e. the case of plain monotone modal logic), all the
prove-work which is connected to the modal operator is shifted to the
pre-proving process. Especially matching with the modal rules
$\mathcal{RO}^m_{sc}$ becomes a mere lookup of the value of $\eval$.
This means, that all calls of the sequent algorithm Algorithm~\ref{alg:optSeq}
correspond in complexity just to ordinary sat-solving of propositional logic.
Furthermore, Algorithm~\ref{alg:optSeq} will be called $|\phi|$ times. This
observation may be generalized:
\end{enumerate}
\label{ex:neighMon}
\end{example}

\begin{remark}
In the case that all modal rules of the considered logic are instances of
the generic rule (\textbf{Opt}) with $P=\emptyset$ (as seen in Example~\ref{ex:neighMon}),
the optimisation does not only allow for a reduction of computing time, but
it also allows us to effectively reduce the sequent calculus to a sat-solving
algorithm.
Furthermore, the optimized algorithm will always be as efficient as the
original one in this case (since every occurence of short sequents over $arg(\Gamma)$ 
which accord to the current instantiation of the rule (\textbf{Opt}) will
have to be shown or refuted even during the course of the original algorithm).
\end{remark}

\section{Conclusion}  

We presented (from a practical point of view) two optimisations for reasoning
in conditional logic:
\begin{itemize}
\item The first optimisation makes use of the concept of dynamic programming
in order to seperate the two tasks that showing validity of formulas in conditional
logic consists of: The first task of proving equivalences of antecedents and the second task
of ordinary sequent proving. This substantially decreases the branching breadth
of the resulting sequent calculus.
\item The second proposed optimisation introduces a strategy to reduce the amount
of pairs of antecedents whose equivalence has to be considered. This is achieved
by distinguishing between connected and independent pairs of modal arguments.
\end{itemize}
When both optimisations are applied at the same time, a significant increase in
performance of the sequent algorithm for conditional logic can be observed.
This was shown in 
Section~\ref{sec:bench} by considering the results of benchmarking a Haskell
implementation (which forms part of the generic proving tool \COLOSS) of the
optimised algorithms.

It remains as an open question whether the gain in perfomance which is obtained
by optimising the algorithm for conditional logic may be transferred to other
logics by making use of the generic optimisation strategy as described in the
last section.

\bibliographystyle{myabbrv}
\bibliography{coalgml}


\end{document}
