%!TEX root = main.tex

In this section we show how the abstract syntax for Maude specifications
is defined in Haskell. This abstract syntax is based in the Maude grammar presented
in \cite[Chapter 24]{maude-book}.

The main datatype of this abstract syntax is \verb"Spec", that
distinguishes between the different specifications available in Maude:
modules, theories, and views. Although both modules and theories contain
the same information, their semantics are different and need different
constructors:

{\codesize
\begin{verbatim}
data Spec = SpecMod Module
          | SpecTh Module
          | SpecView View
          deriving (Show, Read, Ord, Eq)
\end{verbatim}
}

A \verb"Module" is composed of the identifier of the module, a list
of parameters, and a list of statements:

{\codesize
\begin{verbatim}
data Module = Module ModId [Parameter] [Statement]
            deriving (Show, Read, Ord, Eq)
\end{verbatim}
}

\noindent while a \verb"View" is composed of a module identifier, the
source and target module expressions, and a list of renamings:

{\codesize
\begin{verbatim}
data View = View ModId ModExp ModExp [Renaming]
            deriving (Show, Read, Ord, Eq)
\end{verbatim}
}

The \verb"Parameter" type contains the identifier of the parameter,
a sort (used as the parameter identifier), and its type (which is a
module expression):

{\codesize
\begin{verbatim}
data Parameter = Parameter Sort ModExp
               deriving (Show, Read, Ord, Eq)
\end{verbatim}
}

A \verb"Statement" can be any of the Maude statements:
importation, sort, subsort, and operator declarations, and
equation, membership axiom, and rule statements:

{\codesize
\begin{verbatim}
data Statement = ImportStmnt Import
               | SortStmnt Sort
               | SubsortStmnt SubsortDecl
               | OpStmnt Operator
               | EqStmnt Equation
               | MbStmnt Membership
               | RlStmnt Rule
               deriving (Show, Read, Ord, Eq)
\end{verbatim}
}

Importations consist of a module expression qualified by the type
of import:

{\codesize
\begin{verbatim}
data Import = Including ModExp
            | Extending ModExp
            | Protecting ModExp
            deriving (Show, Read, Ord, Eq)
\end{verbatim}
}

A subsort declaration keeps single relations between sorts, being
the first one the subsort and the second one the supersort:

{\codesize
\begin{verbatim}
data SubsortDecl = Subsort Sort Sort
                 deriving (Show, Read, Ord, Eq)
\end{verbatim}
}

Operator declarations are composed of the identifier of the
operator, a list of types giving the arity of the operator,
a type for its coarity, and a list of attributes:

{\codesize
\begin{verbatim}
data Operator = Op OpId [Type] Type [Attr]
              deriving (Show, Read, Ord, Eq)
\end{verbatim}
}

Membership statements consist of a term, its sort, a list
of conditions, and a list of statement attributes:

{\codesize
\begin{verbatim}
data Membership = Mb Term Sort [Condition] [StmntAttr]
                deriving (Show, Read, Ord, Eq)
\end{verbatim}
}

Equations and rules share the same elements: the lefthand
and righthand terms of the statement, a list of conditions,
and a list of statement attributes:

{\codesize
\begin{verbatim}
data Equation = Eq Term Term [Condition] [StmntAttr]
              deriving (Show, Read, Ord, Eq)

data Rule = Rl Term Term [Condition] [StmntAttr]
          deriving (Show, Read, Ord, Eq)
\end{verbatim}
}

We distinguish between the following module expressions:

\begin{itemize}
\item A single identifier:

{\codesize
\begin{verbatim}
data ModExp = ModExp ModId
\end{verbatim}
}

\item A summation, that keeps the two module
expressions involved:

{\codesize
\begin{verbatim}
            | SummationModExp ModExp ModExp
\end{verbatim}
}

\item A renaming, that contains the module expression renamed
and the list of renamings:

{\codesize
\begin{verbatim}
            | RenamingModExp ModExp [Renaming]
\end{verbatim}
}

\item An instantiation, composed of the module instantiated
and the list of view identifiers applied:

{\codesize
\begin{verbatim}
            | InstantiationModExp ModExp [ViewId]
            deriving (Show, Read, Ord, Eq)
\end{verbatim}
}

The \verb"Renaming" type distinguishes the different renamings
available in Maude:

\end{itemize}

\begin{itemize}

\item Renaming of sorts, that indicates that the first sort identifier
is changed to the second one:

{\codesize
\begin{verbatim}
data Renaming = SortRenaming Sort Sort
\end{verbatim}
}

\item Renaming of labels, where the first label is renamed to the
second one:

{\codesize
\begin{verbatim}
              | LabelRenaming LabelId LabelId
\end{verbatim}
}

\item Renaming of operators, that can be of three kinds: renaming
of operators without profile, with profile, or a map between terms,
as explained in Section \ref{subsec:views}:

{\codesize
\begin{verbatim}
              | OpRenaming1 OpId ToPartRenaming
              | OpRenaming2 OpId [Type] Type ToPartRenaming
              | TermMap Term Term
              deriving (Show, Read, Ord, Eq)
\end{verbatim}
}

\noindent where \verb"ToPartRenaming" specifies the new operator identifier
and the new attributes:

{\codesize
\begin{verbatim}
data ToPartRenaming = To OpId [Attr]
                    deriving (Show, Read, Ord, Eq)
\end{verbatim}
}

The \verb"Condition" type distinguishes between the different conditions
available in Maude, namely equational conditions, membership conditions,
matching conditions, and rewriting conditions:

{\codesize
\begin{verbatim}
data Condition = EqCond Term Term
               | MbCond Term Sort
               | MatchCond Term Term
               | RwCond Term Term
               deriving (Show, Read, Ord, Eq)
\end{verbatim}
}

We define the type \verb"Qid", a synonym of \verb"Token" that
will be used for identifiers:

{\codesize
\begin{verbatim}
type Qid = Token
\end{verbatim}
}

Terms are always represented in prefix notation. Notice that
the case of an operator applied to a list of terms is slightly different
to the Maude grammar because it also includes the type of the term.
It will be used later in the implementation to rename operators whose
profile has been specified:

{\codesize
\begin{verbatim}
data Term = Const Qid Type
          | Var Qid Type
          | Apply Qid [Term] Type
          deriving (Show, Read, Ord, Eq)
\end{verbatim}
}

Finally, the \verb"Type" distinguishes between sorts and kinds:

{\codesize
\begin{verbatim}
data Type = TypeSort Sort
          | TypeKind Kind
          deriving (Show, Read, Ord, Eq)
\end{verbatim}
}

\end{itemize}
























