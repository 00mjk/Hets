%!TEX root = main.tex

Images???

We show in this section the details of the generation of development
graphs for Maude specifications.

\subsection{Parameterized modules}

As we have seen in Section \ref{subsec:pmod}, parameterized modules 
receive a list of parameters $P_i$ each one associated to a module
expression $M_i$. Since these module expressions are theories, they
can have sorts that are not instantiated yet, so to avoid clash of
names these sorts are qualified with the parameter name. These
qualifications induce an implicit morphism in the definition links
of the development graph between the theory and the parameterized
module.

\subsection{Module expressions}\label{subsec:me}

Maude module expressions allow to combine and modify the information
contained in Maude modules.

\begin{itemize}
\item The summation expression of the module expressions $\mathit{ME}_1$ and
$\mathit{ME}_2$ generates a new node in the development graph
$(\mathit{ME}_1 + \mathit{ME}_2)$ with
the union of the information in both expressions. A definition link
is also created between the original expressions and the resulting one.
\item The renaming expression $\mathit{ME} * (R)$ creates a morphism with
the information given in $R$ that will be used in the link 
\item The instantiation module expression\footnote{This module expression
cannot be used in Core Maude in all places where the rest of expressions
can be used, for more information see \cite{maude-book}} assigns a view
or a parameter to each of the

Note that an instantiation generates some implicit morphisms and modifies
the ones stated in the views:
\begin{itemize}
\item When the parameter is instantiated with a view
\item As explained in the previous section, parameterized modules can
define sorts of the form \verb"name{X1, ..., Xn}", where \verb"X1, ..., Xn"
are parameter names that can be instantiated or bound by new parameters.
Thus, in addition to the morphism
\end{itemize}

\item When the module expression is a simple identifier the development
graph remains unchanged.
\end{itemize}

\subsection{Importations}

Maude allow three different kinds of module importation:

\begin{itemize}
\item Protecting, that means that \emph{no junk} and \emph{no confusion}
are added to the imported module expression.
\item Extending, that means that allows ``junk'' to be added to imported
module expression, while ``confusion'' is forbidden.
\item Including, that allows both ``junk'' and ``confusion'' to be
added to the imported module expression.
\end{itemize}

Each of these importation modes generates links with different proofs
obligations in the development graph:

\begin{itemize}
\item When a module expression $M_1$ is imported in protecting mode
by a module $M_2$ a new node $M_1'$ with the same signature that $M_1$
is generated and a free link between $M_1$ and $M_1'$
placed, while a definition link between $M_1'$ and $M_2$.
\item Extending mode generates a theorem link with the annotation
\verb"PCons", standing for proof-theoretic conservativity. This
constraint cannot be discharged in the current version of the system.
\item Including mode generates a definition link from the imported module
expression to the module that imports it.
\end{itemize}

The morphism in all these links depends of the type of module expression.
It will be an inclusion if the module expression does not contains a
renaming, while it be a morphism in other case.

\subsection{Views}

Maude views have a theory as source and either a module or a theory
as target. All the sorts and operators declared in the source theory
have to be mapped to sorts and operators in the target

A particular case of mapping between operators is the mapping between
terms, that has the general form \emph{op $e$ to term $t$}
where $e$ is a term consisting of a 
single operator applied to variables�declared either on-the-�y or with
variable declarations in the same view and the target term is any term
with variables, those in the source $e$ in the corresponding sorts
resulting from the mapping.
Since this shortcut allows to map operators with different profiles,

































