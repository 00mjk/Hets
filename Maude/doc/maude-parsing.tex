%!TEX root = main.tex

In this section we explain how the Maude specifications introduced in
Hets are parsed in order to obtain its abstract syntax, described in
Section \ref{subsec:abs-syntax}.

The function \verb"haskellify" receives a module (the first parameter
stands for the original module, while the second one contains the
flattened one) and returns a list of quoted identifiers identifying an
object of type \verb"Spec" as shown in the previous section.

{\codesize
\begin{verbatim}
  op haskellify : Module Module -> QidList .
  ceq haskellify(M, M') = 
      'SpecMod '`( 'Module haskellifyHeader(H) ' ' 
      '`[ haskellifyImports(IL) comma(IL, SS)
          haskellifySorts(SS) comma(IL, SS, SSDS)
          haskellifySubsorts(SSDS) comma(IL, SS, SSDS, ODS)
          haskellifyOpDeclSet(M', ODS) comma(IL, SS, SSDS, ODS, MAS)
          haskellifyMembAxSet(M', MAS) comma(IL, SS, SSDS, ODS, MAS, EqS)
          haskellifyEqSet(M', EqS) '`] '`) '\n '@#$endHetsSpec$#@ '\n
    if fmod H is IL sorts SS . SSDS ODS MAS EqS endfm := M .
\end{verbatim}
}

This function prints the keyword \verb"SpecMod" and uses the \verb"haskellify"
auxiliary functions to print the different parts of the module.
The functions \verb"comma" introduce a comma whenever it is necessary.
Since all the ``haskellify'' functions are very similar, we describe
them by using \verb"haskellifyImports" as example. This function traverses
all the imports in the list and applies the auxiliary function
\verb"haskellifyImport" to each of them:

{\codesize
\begin{verbatim}
  op haskellifyImports : ImportList -> QidList .
  eq haskellifyImports(nil) = nil .
  eq haskellifyImports(I IL) = 'ImportStmnt ' '`( haskellifyImport(I) '`)
                               comma(IL) haskellifyImports(IL) .
\end{verbatim}
}

This auxiliary function distinguishes between the importation modes,
using the appropriate keyword for each of them:

{\codesize
\begin{verbatim}
  op haskellifyImport : Import -> QidList .
  eq haskellifyImport(protecting ME .) = 'Protecting haskellifyME(ME) .
  eq haskellifyImport(including ME .) = 'Including haskellifyME(ME) .
  eq haskellifyImport(extending ME .) = 'Extending haskellifyME(ME) .
\end{verbatim}
}

\noindent where \verb"haskellifyME" is in charge of printing the 
module expression.

