\documentclass[10pt,a4paper]{article}
\usepackage{geometry}
\geometry{a4paper}
\geometry{margin=2.5cm,nohead}
\usepackage{amsfonts}
\usepackage[leqno]{amsmath}
\usepackage{rotating}
\usepackage[english]{babel}
\usepackage{url}
\usepackage{proof}
\usepackage{alltt}
\usepackage{hyperref}
\usepackage{verbatim}
\usepackage{amssymb}
\usepackage{xspace}
\usepackage{xypic}
\usepackage{hetcasl}
\usepackage{wrapfig}

\usepackage{graphicx}
\usepackage{proof}
\usepackage{hetcasl}

% Images
%\renewcommand{\topfraction}{0.95} 
%\renewcommand{\textfraction}{0.1} 
%\renewcommand{\floatpagefraction}{0.75} 

% Texttt curly brackets
\newcommand{\ttlcb}{\texttt{\char'173}}
\newcommand{\ttrcb}{\texttt{\char'175}}

\def\squareforqed{\hbox{\rlap{$\sqcap$}$\sqcup$}}
\def\qed{\ifmmode\squareforqed\else{\unskip\nobreak\hfil
\penalty50\hskip1em\null\nobreak\hfil\squareforqed
\parfillskip=0pt\finalhyphendemerits=0\endgraf}\fi}

\newtheorem{definition}{Definition} %[chapter]
\newtheorem{proposition}{Proposition} %[chapter]
\newtheorem{property}{Property} %[chapter]
\newtheorem{lemma}{Lemma} %[chapter]
\newtheorem{theorem}{Theorem} %[chapter]
\newenvironment{proof}{\par\addvspace{\bigskipamount}%
\noindent\textit{Proof.}\ }{\qed\par\addvspace{\bigskipamount}}

\newcommand{\stroke}{|}
\newcommand{\forget}[1]{|_{#1}}
\newcommand{\Mod}{\mathbf{Mod}}
\newcommand{\Sen}{\mathbf{Sen}}
\newcommand{\red}{\upharpoonright}

\newcommand{\codesize}{\small}
\newcommand{\field}[1]{\mathbb{#1}}
\newcommand{\vertex}[1]{\|\,#1\,\|}
\newcommand{\mt}{{\small$\blacktriangle$}}
\newcommand{\leaf}[1]{{\small$\blacktriangle\;$}#1{\small$\;\blacktriangle$}}

\newcommand{\nodo}[1]{\rule[-.5ex]{0pt}{2.4ex}\,#1\,}
%\newcommand{\lab}[1]{\!\raisebox{0.5ex}{\scriptsize\textsf{#1}}}
\newcommand{\lab}[1]{\!{\scriptsize\textsf{#1}}}
\newcommand{\labb}[2]{\!{\scriptsize\textsf{#1}_\textsf{#2}}}
\newcommand{\labapt}[2]{{\scriptstyle\;\textsf{#1}_\textsf{#2}}}

\newcommand{\model}{\mathcal{A}} % use: $\model$
\newcommand{\stepRule}[1]{\scriptstyle{ #1}}
\newcommand{\den}[1]{[\![#1]\!]_\model}  % use: $\den{e}$

\newcommand{\bigunion}{\mathop{ \mathgroup\symoperators \bigcup}}
\newcommand{\fracc}[2]{\begin{array}{c}{#1}\\ \hline {#2} \end{array}}

\newcommand{\Hets}{\textmd{\textsc{Hets}}\xspace}
\newcommand{\CASL}{\textmd{\textsc{Casl}}\xspace}
\newcommand{\CoFI}{\textmd{\textsc{CoFI}}\xspace}

\newcommand{\free}[2]{\mbox{\textbf{free} $#1$ \textbf{along} $#2$}}

\newcommand{\rewrites}{\Rightarrow}

%% macros for development graphs

\newcommand{\glinka}[3]{\!\xymatrix{{#1} \ar@{=>}[r]^{#2} & {#3}}\!\!}
\newcommand{\longglinka}[3]{\!\xymatrix{#1 \ar@{=>}[rr]^{#2} && #3}\!\!}
\newcommand{\llinka}[3]{\!\xymatrix{#1 \ar[r]^{#2} & #3}\!\!}
\newcommand{\hlinka}[3]{\!\xymatrix{#1 \ar@{=>}[r]^{#2}_{\mathit{hide}} & #3}\!\!}
\newcommand{\longhlinka}[3]{\!\xymatrix{#1 \ar@{=>}[rr]^{#2}_{\mathit{hide}} && #3}\!\!}
\newcommand{\flinka}[3]{\!\xymatrix{#1 \ar@{=>}[r]^{#2}_{\mathit{free}} & #3}\!\!}
\newcommand{\longflinka}[3]{\!\xymatrix{#1 \ar@{=>}[rr]^{#2}_{\mathit{free}} && #3}\!\!}

\newcommand{\tglinka}[3]{\!\xymatrix{#1 \ar@{==>}[r]^{#2} & #3}\!\!}
\newcommand{\longtglinka}[3]{\!\xymatrix{#1 \ar@{==>}[rr]^{#2} && #3}\!\!}
\newcommand{\thlinka}[4]{\!\xymatrix{#1 \ar@{==>}[r]^{#2}_{{\mathit{hide}}\ #3} & #4}\!\!}
\newcommand{\longthlinka}[4]{\!\xymatrix{#1 \ar@{==>}[rr]^{#2}_{{\mathit{hide}}\ #3} && #4}\!\!}
\newcommand{\tflinka}[4]{\!\xymatrix{#1 \ar@{==>}[r]^{#2}_{\mathit{free}\ #3} & #4}\!\!}
\newcommand{\tllinka}[3]{\!\xymatrix{#1 \ar@{-->}[r]^{#2} & #3}\!\!}

\newcommand{\gclinka}[3]{\!\xymatrix{#1 \ar@{=>}[r]^{#2}_{\mathit{cons}} & #3}\!\!}
\newcommand{\gdlinka}[3]{\!\xymatrix{#1 \ar@{=>}[r]^{#2}_{\mathit{def}} & #3}\!\!}
\newcommand{\gmlinka}[3]{\!\xymatrix{#1 \ar@{=>}[r]^{#2}_{\mathit{mono}} & #3}\!\!}

\newcommand{\tclinka}[3]{\!\xymatrix{#1 \ar@{==>}[r]^{#2}_{\mathit{cons}} & #3}\!\!}
\newcommand{\tdlinka}[3]{\!\xymatrix{#1 \ar@{==>}[r]^{#2}_{\mathit{def}} & #3}\!\!}
\newcommand{\tmlinka}[3]{\!\xymatrix{#1 \ar@{==>}[r]^{#2}_{\mathit{mono}} & #3}\!\!}

\newcommand{\longtclinka}[3]{\!\xymatrix{#1 \ar@{==>}[rr]^{#2}_{\mathit{cons}} && #3}\!\!}
\newcommand{\longtdlinka}[3]{\!\xymatrix{#1 \ar@{==>}[rr]^{#2}_{\mathit{def}} && #3}\!\!}
\newcommand{\longtmlinka}[3]{\!\xymatrix{#1 \ar@{==>}[rr]^{#2}_{\mathit{mono}} && #3}\!\!}

\newcommand{\greaches}[1]{\!\xymatrix{\ar@{=>>}[r]^{#1} & }\!\!}
\newcommand{\lreaches}[1]{\!\xymatrix{\ar@{>=>>}[r]^{#1} & }\!\!}

\newtheorem{fact}{Fact}


\graphicspath{{images/}}

\title{Integrating Maude into Hets%
%\thanks{Research supported by MEC Spanish project
%\emph{DESAFIOS} (TIN2006-15660-C02-01) and Comunidad de
%Madrid program \emph{PROMESAS} (S�0505/TIC/0407).}
}

\author{Mihai Codescu, Till Mossakowski,
Adri\'an Riesco, and Christian Maeder\\[.7cm]
\normalsize Technical Report 07/10\\[1ex]
  \normalsize\textit{Departamento de Sistemas Inform\'aticos y Computaci\'on}\\
  \normalsize\textit{Universidad Complutense de Madrid}\\[.4cm]
  September 2010
  }


\date{}

%\institute{}

%%%%%%%%%%%%%%%%%%%%%%%%%%%%%%%%%%%%%%%%%%%%%%%%%%%%%%%%%%
\begin{document}
%
\maketitle

\thispagestyle{empty}
\newpage
\thispagestyle{empty}
\mbox{}\vfill

\begin{abstract}
Maude modules can be understood as models that can be formally analyzed
and verified with respect to different properties expressing various
formal requirements.
However, Maude lacks the formal tools to perform some of these analyses
and thus they can only be done by hand.
The Heterogeneous Tool Set \Hets is an institution-based
combination of different logics and corresponding rewriting, model
checking, and proof tools.
%
We present in this paper an integration of Maude into \Hets that allows
to use the logics and tools already integrated in \Hets with
Maude specifications. To achieve such integration we have defined an
institution for Maude based on preordered algebras and a comorphism
between Maude and \CASL, the central logic in \Hets.

\smallskip

\noindent\textbf{Keywords:} rewriting logic, heterogeneous specifications,
Maude, \CASL
\end{abstract}

\vfill

\begin{small}
\tableofcontents
\end{small}

\vfill
\mbox{}
\newpage
%\setcounter{page}{1}

\section{Introduction}\label{sec:intro}
\input{intro}

\section{Rewriting logic and Maude}\label{sec:maude}
%!TEX root = main.tex

As mentioned in the introduction, Maude modules are executable rewriting logic especifications.
Rewriting logic \cite{Meseguer92-tcs} is a logic of change very suitable
for the specification of concurrent systems that is parameterized
by an underlying equational logic, for which Maude uses membership
equational logic \cite{BouhoulaJouannaudMeseguer00,Meseguer97},
which, in addition to equations, allows the statement of membership
axioms characterizing the elements of a sort. In the following sections we
present both logics and how their specifications are represented as Maude modules.

\subsection{Membership equational logic} \label{mel-section}

A \emph{signature} in membership equational logic is a triple $(K,\Sigma, S)$
(just $\Sigma$ in the following),\
with $K$ a set of {\em kinds},
$\Sigma = \{\Sigma_{k_1\ldots k_n,k}\}_{(k_1\ldots k_n,k)\in K^{*}\times K}$ a
many-kinded signature, and $S =
\{S_{k}\}_{k\in K}$ a pairwise disjoint $K$-kinded family of sets of
\emph{sorts}.
The kind of a sort $s$ is denoted by $[s]$.
We write $T_{\Sigma,k}$ and $T_{\Sigma,k}(X)$ to denote respectively the set
of ground
$\Sigma$-terms with kind $k$ and of $\Sigma$-terms with kind $k$ over variables
in $X$, where $X = \{ x_1:k_1, \dots, x_n:k_n\}$ is a set of $K$-kinded
variables.
Intuitively, terms with a kind but without a sort represent undefined or error
elements.

The atomic formulas of membership equational logic are either \emph{equations}
$t = t'$, where $t$ and $t'$ are $\Sigma$-terms of the same kind, or
\emph{membership axioms} of the form $t : s$, where the term $t$
has kind $k$ and $s \in S_k$.
\emph{Sentences} are universally-quantified Horn clauses of the
form $(\forall X)\, A_0 \Leftarrow A_1 \wedge \ldots \wedge A_n$,
where each $A_i$ is  either an equation or a membership axiom, and $X$ is a
set of $K$-kinded variables containing all the variables in the $A_i$.
A \emph{specification} is a pair $(\Sigma,E)$, where $E$ is a set
of sentences in membership equational logic over the signature $\Sigma$.

Models of membership equational logic specifications are
\emph{$\Sigma$-algebras} $\model$ consisting of a set $A_k$ for each kind $k \in K$,
a function $A_f : A_{k_1}\times \dots \times A_{k_n} \longrightarrow A_k$ for each 
operator $f \in \Sigma_{k_1 \dots k_n, k}$, and
a subset $A_s \subseteq A_k$ for each sort $s\in S_k$. The meaning
$\den{t}$ of a term $t$ in an algebra $\model$ is inductively defined as usual.
Then, an algebra $\model$ satisfies an equation $t = t'$  (or the equation holds
in the algebra), denoted $\model \models t = t'$, when both terms have the same meaning:
$\den{t} = \den{t'}$. In the same way, satisfaction of a membership is defined as:
$\model \models t : s$ when $\den{t} \in A_s$.

A membership equational logic specification $(\Sigma,E)$ has an initial model 
$\mathcal{T}_{\Sigma/E}$ whose elements
are $E$-equivalence classes of terms $[t]$.
We refer to \cite{BouhoulaJouannaudMeseguer00,Meseguer97} for a detailed presentation of
$(\Sigma,E)$-algebras,
sound and complete deduction rules, as well as the construction of
initial and free algebras.

Since the membership equational logic specifications that we consider are assumed
to satisfy the executability requirements of confluence, termination, and 
sort-decreasingness, their equations $t=t'$ can be oriented from left to right,
$t \rightarrow t'$. Such a statement holds in an algebra, denoted 
$\model \models t \rightarrow t'$, exactly when $\model \models t = t'$, i.e., when
$\den{t} = \den{t'}$. Moreover, under those assumptions an equational condition $u = v$
in a conditional equation can be checked by finding a common term $t$ such
that $u \rightarrow t$ and $v \rightarrow t$.

\subsection{Maude functional modules} \label{maudefmod}

Maude functional modules \cite[Chapter 4]{maude-book}, introduced 
with syntax \texttt{fmod ...\ endfm}, are executable membership
equational specifications and their semantics is given by the corresponding
initial membership algebra in the class of algebras satisfying the specification.

In a functional module we can declare sorts (by means of keyword
\texttt{sort}(\texttt{s})); subsort relations between sorts
(\texttt{subsort}); operators (\texttt{op}) for building values of these
sorts, giving the sorts of their arguments and result, and which may have
attributes such as being associative (\texttt{assoc}) or commutative
(\texttt{comm}), for example; memberships (\texttt{mb}) asserting that a term
has a sort; and equations (\texttt{eq}) identifying terms.  
Both memberships and equations can be conditional (\texttt{cmb} and \texttt{ceq}).

Maude does automatic kind inference from the sorts declared by the user and
their subsort relations.  Kinds are \emph{not} declared explicitly, and
correspond to the connected components of the subsort relation.
The kind corresponding to a sort \texttt{s} is denoted \texttt{[s]}.
For example, if we have sorts \texttt{Nat} for natural numbers and \texttt{NzNat} 
for nonzero natural numbers with a subsort \texttt{NzNat < Nat}, then 
\texttt{[NzNat]} = \texttt{[Nat]}.

An operator declaration like

{\codesize
\begin{verbatim}
 op _div_ : Nat NzNat -> Nat .
\end{verbatim}
}

\noindent
is logically understood as a declaration at the kind level

{\codesize
\begin{verbatim}
 op _div_ : [Nat] [Nat] -> [Nat] .
\end{verbatim}
}

\noindent
together with the conditional membership axiom

{\codesize
\begin{verbatim}
 cmb N div M : Nat if N : Nat and M : NzNat .
\end{verbatim}
}

A subsort declaration \texttt{NzNat < Nat} is logically understood as
the conditional membership axiom

{\codesize
\begin{verbatim}
 cmb N : Nat if N : NzNat .
\end{verbatim}
}

\subsection{Rewriting logic}

Rewriting logic extends equational logic by introducing the notion of \emph{rewrites} 
corresponding to
transitions between states; that is, while equations are interpreted as equalities and therefore
they are symmetric, rewrites denote changes which can be irreversible. 

A rewriting logic specification, or \emph{rewrite theory}, has the form
$\mathcal{R} = (\Sigma,E,R)$, where $(\Sigma,E)$ is an equational specification
and $R$ is a set of \emph{rules} as described below. From this definition,
one can see that rewriting logic is built on top of equational logic, so
that rewriting logic is parameterized
with respect to the version of the underlying equational logic; in our
case, Maude uses membership equational logic, as described in the
previous sections. A rule in $R$ has the general conditional
form\footnote{There is no need for the condition listing first equations,
then  memberships, and then rewrites: this is just a notational
abbreviation, they can be listed in any order.} 
\[
(\forall X) \; e \Rightarrow e' \Leftarrow \bigwedge_{i=1}^{n} u_i = u'_i \wedge
                      \bigwedge_{j=1}^{m}  v_j : s_j \wedge
                      \bigwedge_{k=1}^{l} w_k \Rightarrow w'_k
\]
where the head is a rewrite and the conditions can be equations,
memberships, and rewrites; both sides of a rewrite must have the same kind. 
From these rewrite rules, one can deduce rewrites of the form
$t \Rightarrow t'$ by means of general deduction rules introduced
in \cite{Meseguer92-tcs} (see also \cite{BruniMeseguer06}).

Models of rewrite theories are called \emph{$\mathcal{R}$-systems}
in \cite{Meseguer92-tcs}.
Such systems are defined as categories that possess a
$(\Sigma,E)$-algebra structure, together with a natural transformation
for each rule in the set $R$. More intuitively, the idea is that we have a
$(\Sigma,E)$-algebra, as described in Section~\ref{mel-section}, with
transitions between the elements in each set $A_k$; moreover, these
transitions must satisfy several additional requirements, including that
there are identity transitions for each element, that transitions can
be sequentially composed, that the operations in the signature $\Sigma$
are also appropriately defined for the transitions, and that we have
enough transitions corresponding to the rules in $R$. Then, if we keep in
this context the notation $\model$ to denote an $\mathcal{R}$-system, a
rewrite $t \Rightarrow t'$ is satisfied by $\model$, 
denoted $\model \models t \Rightarrow t'$, when there is a transition
$\den{t} \rightarrow_\model \den{t'}$ in the system between the
corresponding meanings of both sides of the rewrite, where $\rightarrow_\model$
will be our notation for such transitions. 

The rewriting logic deduction rules introduced in \cite{Meseguer92-tcs}
are sound and complete with respect to this notion of model. Moreover,
they can be used to build initial and free models; see \cite{Meseguer92-tcs}
for details.

\subsection{Maude system modules}

Maude system modules \cite[Chapter 6]{maude-book}, introduced with
syntax \texttt{mod ...\ endm}, are executable rewrite 
theories and their semantics is given by the initial system in the class of 
systems corresponding to the rewrite theory.  A system module can contain all the
declarations of a functional module and, in addition, declarations for
rules (\texttt{rl}) and conditional rules (\texttt{crl}).

The executability requirements for equations and memberships in a system
module are the same as those of functional modules, namely, confluence,
termination, and sort-decreasingness. With respect to rules, the satisfaction
of all the conditions in a conditional rewrite rule is attempted sequentially 
from left to right, solving rewrite conditions by means of search; 
for this reason, we can have new variables in such conditions but they
must become instantiated along this process of solving from left to right
(see \cite{maude-book} for details). Furthermore, the strategy followed
by Maude in rewriting with rules is to compute the normal form of a term
with respect to the equations before applying a rule. This strategy is
guaranteed not to miss any rewrites when the rules are \emph{coherent}
with respect to the equations \cite{eq-rl-rwl,maude-book}. In a way
quite analogous to confluence, this coherence requirement means that, given
a term $t$, for each rewrite of it using a rule in $R$ to some term $t'$,
if $u$ is the normal form of $t$ with respect to the equations and
memberships in $E$, then there is a rewrite of $u$ with some rule in
$R$ to a term $u'$ such that $u' =_E t'$ (that is, the equation $t' = u'$
can be deduced from $E$).

\subsection{Theories}\label{subsec:theories}

Theories are used to declare module interfaces, namely the syntactic
and  semantic properties to be satisfied by the actual parameter modules
used in an instantiation. As for modules, Maude supports two different types
of theories: functional theories and system theories, with the same structure
of their module counterparts, but with a different semantics. Functional
theories  are declared with the keywords \verb"fth ... endfth", and
system theories with the keywords \verb"th ... endth". Both of them can
have sorts, subsort relationships, operators, variables, membership axioms,
and equations, and can import other theories or modules. System theories can
also have rules. Although there is no restriction on the operator attributes
that can be used in a theory, there are some subtle restrictions and
issues regarding the mapping of such operators (see Section
\ref{subsec:views}).

Like functional modules, functional theories are membership equational
logic theories, but they do not need to be Church-Rosser and terminating,
and  therefore some or all of their statements may be declared with the
\verb"nonexec" attribute and can only be executed in a controlled way at 
the metalevel.

\subsection{Views}\label{subsec:views}

We use views to specify how a particular target module or theory is claimed
to satisfy a source theory. In general, there may be several ways in which 
such requirements might be satisfied, if at all, by the target module or
theory; that is, there can be many different views, each specifying a
particular interpretation of the source theory in the target. Each
view declaration has an associated set of proof obligations, namely, for
each axiom in the source theory it should be the case that the axiom's
translation by the view holds true in the target. Since the target can
be a module interpreted initially, verifying such proof obligations may
in general require inductive proof techniques. Such proof obligations
are not discharged or checked by the system. 
In the definition of a view we have to indicate its name, the source
theory, the target module or theory, and the mapping of each sort and
operator in the source theory. The source and target of a
view can be any module expression, with the source module expression
evaluating to a theory and the target module expression evaluating to a
module or a theory.

EXPLICAR LOS TIPOS DE MAPPINGS

\subsection{Parameterized modules}\label{subsec:pmod}

System modules and functional modules can be parameterized. A parameterized
system module has syntax

$$
\verb"mod M{" X_1 :: T_1 , \ldots , X_n :: T_n \verb"} is ... endm"
$$

\noindent with $n \geq 1$. Parameterized functional modules have completely
analogous syntax.

The \verb"{"$X_1 :: T_1 , \ldots , X_n :: T_n$\verb"}" part is called the
interface, where each pair $X_i :: T_i$ is a parameter, and each $X_i$ is an
identifier---the parameter name or parameter label---and each $T_i$ is
an expression that yields a theory---the parameter theory. Each parameter
name in an interface must be unique, although there is no uniqueness
restriction on the parameter theories of a module. The parameter theories
of a functional module must be functional theories.

In a parameterized module $M$, all the sorts and statement labels
coming from theories in its interface must be qualified by their names. Thus,
given a parameter $X_i :: T_i$, each sort $S$ in $T_i$ must be
qualified as $X_i\texttt{\$}S$, and each label $l$ of a statement occurring in
$T_i$ must be qualified as $X_i\texttt{\$}l$. In fact, the parameterized module
$M$ is flattened as follows. For each parameter $X_i :: T_i$, 
a renamed copy of the theory $T_i$, called $X_i :: T_i$ is included.
The renaming  maps each sort $S$ to $X_i\texttt{\$}S$, and each label $l$
of a statement occurring in $T_i$ to $X_i\texttt{\$}l$. The renaming has
no effect on importations of modules. Thus, if $T_i$ includes a theory $T'$,
when the renamed
theory $X_i :: T_i$ is created and included into $M$, and the renamed
theory $X_i :: T'$ will also be created and included into $X_i :: T_i$. 
However, the renaming will have no effect on modules imported by either the
$T_i$ or $T'$; for example, if \verb"BOOL" is imported by one of these
theories, it is not renamed, but imported in the same way into $M$.





















\section{Hets}\label{sec:hets}
\setlength{\unitlength}{1740sp}%
%
\begingroup\makeatletter\ifx\SetFigFont\undefined%
\gdef\SetFigFont#1#2#3#4#5{%
  \reset@font\fontsize{#1}{#2pt}%
  \fontfamily{#3}\fontseries{#4}\fontshape{#5}%
  \selectfont}%
\fi\endgroup%
\begin{picture}(12534,8879)(439,-8907)
\thinlines
{\color[rgb]{0,0,0}\put(2458,-5596){\vector( 0,-1){630}}
}%
{\color[rgb]{0,0,0}\put(2458,-4799){\vector( 0,-1){270}}
}%
{\color[rgb]{0,0,0}\put(2458,-4269){\line( 0,-1){260}}
}%
{\color[rgb]{0,0,0}\put(2458,-3645){\vector( 0,-1){270}}
}%
{\color[rgb]{0,0,0}\put(2458,-3115){\line( 0,-1){260}}
}%
{\color[rgb]{0,0,0}\put(2458,-2491){\vector( 0,-1){270}}
}%
{\color[rgb]{0,0,0}\put(2458,-1961){\line( 0,-1){260}}
}%
{\color[rgb]{0,0,0}\put(10773,-5551){\vector( 0,-1){900}}
}%
{\color[rgb]{0,0,0}\put(10773,-4799){\vector( 0,-1){270}}
}%
{\color[rgb]{0,0,0}\put(10773,-4269){\line( 0,-1){260}}
}%
{\color[rgb]{0,0,0}\put(10773,-3645){\vector( 0,-1){270}}
}%
{\color[rgb]{0,0,0}\put(10773,-3115){\line( 0,-1){260}}
}%
{\color[rgb]{0,0,0}\put(10773,-2491){\vector( 0,-1){270}}
}%
{\color[rgb]{0,0,0}\put(10773,-1961){\line( 0,-1){260}}
}%
{\color[rgb]{0,0,0}\put(6436,-3391){\line(-2, 3){595.385}}
}%
{\color[rgb]{0,0,0}\put(6571,-3391){\line( 0, 1){1530}}
}%
{\color[rgb]{0,0,0}\put(6706,-3391){\line( 2, 3){595.385}}
}%
{\color[rgb]{0,0,0}\put(7039,-3523){\line( 3, 2){488.077}}
}%
{\color[rgb]{0,0,0}\put(6226,-3548){\line(-3, 2){488.077}}
}%
{\color[rgb]{0,0,0}\put(5791,-4046){\line( 3, 2){405}}
}%
{\color[rgb]{0,0,0}\put(7253,-4056){\line(-3, 2){405}}
}%
{\color[rgb]{0,0,0}\put(6795,-4557){\line( 5, 2){411.207}}
}%
{\color[rgb]{0,0,0}\put(6283,-4558){\line(-5, 2){411.207}}
}%
{\color[rgb]{0,0,0}\put(5691,-5146){\line( 3, 2){405}}
}%
{\color[rgb]{0,0,0}\put(7403,-5146){\line(-3, 2){405}}
}%
{\color[rgb]{0,0,0}\put(6571,-5056){\line( 0, 1){180}}
}%
{\color[rgb]{0,0,0}\put(10773,-2755){\oval(3785,5632)}
%\put(9106,-164){\oval(210,210)[tl]}
%\put(12541,-5446){\oval(210,210)[br]}
%\put(12541,-164){\oval(210,210)[tr]}
%\put(9106,-5551){\line( 1, 0){3435}}
%\put(9106,-59){\line( 1, 0){3435}}
%\put(9001,-5446){\line( 0, 1){5282}}
%\put(12646,-5446){\line( 0, 1){5282}}
}%
{\color[rgb]{0,0,0}\put(2458,-2755){\oval(3785,5632)}
%{\color[rgb]{0,0,0}\put(781,-5491){\oval(210,210)[bl]}
%\put(781,-145){\oval(210,210)[tl]}
%\put(4306,-5491){\oval(210,210)[br]}
%\put(4306,-145){\oval(210,210)[tr]}
%\put(781,-5596){\line( 1, 0){3525}}
%\put(781,-40){\line( 1, 0){3525}}
%\put(676,-5491){\line( 0, 1){5346}}
%\put(4411,-5491){\line( 0, 1){5346}}
}%
{\color[rgb]{0,0,0}\put(6623,-2755){\oval(3785,5632)}
%{\color[rgb]{0,0,0}\put(4966,-5491){\oval(210,210)[bl]}
%\put(4966,-166){\oval(210,210)[tl]}
%\put(8430,-5491){\oval(210,210)[br]}
%\put(8430,-166){\oval(210,210)[tr]}
%\put(4966,-5596){\line( 1, 0){3464}}
%\put(4966,-61){\line( 1, 0){3464}}
%\put(4861,-5491){\line( 0, 1){5325}}
%\put(8535,-5491){\line( 0, 1){5325}}
}%
{\color[rgb]{0,0,0}\put(4361,-2851){\line( 1, 0){350}}
}%
{\color[rgb]{0,0,0}\put(8521,-2851){\vector( 1, 0){350}}
}%
{\color[rgb]{0,0,0}\put(751,-7276){\oval(330,330)[bl]}
\put(751,-6526){\oval(330,330)[tl]}
\put(2941,-7276){\oval(330,330)[br]}
\put(2941,-6526){\oval(330,330)[tr]}
\put(751,-7441){\line( 1, 0){2190}}
\put(751,-6361){\line( 1, 0){2190}}
\put(586,-7276){\line( 0, 1){750}}
\put(3106,-7276){\line( 0, 1){750}}
}%
{\color[rgb]{0,0,0}\put(586,-8760){\oval(270,270)[bl]}
\put(586,-6361){\oval(270,270)[tl]}
\put(5851,-8760){\oval(270,270)[br]}
\put(5851,-6361){\oval(270,270)[tr]}
\put(586,-8895){\line( 1, 0){5265}}
\put(586,-6226){\line( 1, 0){5265}}
\put(451,-8760){\line( 0, 1){2399}}
\put(5986,-8760){\line( 0, 1){2399}}
}%
{\color[rgb]{0,0,0}\put(1726,-8551){\oval(300,300)[bl]}
\put(1726,-7906){\oval(300,300)[tl]}
\put(4411,-8551){\oval(300,300)[br]}
\put(4411,-7906){\oval(300,300)[tr]}
\put(1726,-8701){\line( 1, 0){2685}}
\put(1726,-7756){\line( 1, 0){2685}}
\put(1576,-8551){\line( 0, 1){645}}
\put(4561,-8551){\line( 0, 1){645}}
}%
{\color[rgb]{0,0,0}\put(8537,-8101){\oval(300,300)[bl]}
\put(8537,-6646){\oval(300,300)[tl]}
\put(12811,-8101){\oval(300,300)[br]}
\put(12811,-6646){\oval(300,300)[tr]}
\put(8537,-8251){\line( 1, 0){4274}}
\put(8537,-6496){\line( 1, 0){4274}}
\put(8387,-8101){\line( 0, 1){1455}}
\put(12961,-8101){\line( 0, 1){1455}}
}%
{\color[rgb]{0,0,0}\put(3481,-7290){\oval(300,300)[bl]}
\put(3481,-6531){\oval(300,300)[tl]}
\put(5656,-7290){\oval(300,300)[br]}
\put(5656,-6531){\oval(300,300)[tr]}
\put(3481,-7440){\line( 1, 0){2175}}
\put(3481,-6381){\line( 1, 0){2175}}
\put(3331,-7290){\line( 0, 1){759}}
\put(5806,-7290){\line( 0, 1){759}}
}%
{\color[rgb]{0,0,0}\put(5986,-7441){\vector( 1, 0){2385}}
}%
\put(2135,-1861){\makebox(0,0)[lb]{\smash{\SetFigFont{7}{8.4}{\sfdefault}{\mddefault}{\updefault}{\color[rgb]{0,0,0}Text}%
}}}
\put(2106,-2438){\makebox(0,0)[lb]{\smash{\SetFigFont{7}{8.4}{\sfdefault}{\mddefault}{\updefault}{\color[rgb]{0,0,0}\emph{Parser}}%
}}}
\put(1441,-3015){\makebox(0,0)[lb]{\smash{\SetFigFont{7}{8.4}{\sfdefault}{\mddefault}{\updefault}{\color[rgb]{0,0,0}Abstract syntax}%
}}}
\put(1616,-3592){\makebox(0,0)[lb]{\smash{\SetFigFont{7}{8.4}{\sfdefault}{\mddefault}{\updefault}{\color[rgb]{0,0,0}\emph{Static analysis}}%
}}}
\put(1131,-4169){\makebox(0,0)[lb]{\smash{\SetFigFont{7}{8.4}{\sfdefault}{\mddefault}{\updefault}{\color[rgb]{0,0,0}(Signature, Sentences)}%
}}}
\put(1896,-4746){\makebox(0,0)[lb]{\smash{\SetFigFont{7}{8.4}{\sfdefault}{\mddefault}{\updefault}{\color[rgb]{0,0,0}\emph{Interfaces}}%
}}}
\put(1641,-5323){\makebox(0,0)[lb]{\smash{\SetFigFont{7}{8.4}{\sfdefault}{\mddefault}{\updefault}{\color[rgb]{0,0,0}XML, ATerms}%
}}}
\put(6301,-3706){\makebox(0,0)[lb]{\smash{\SetFigFont{7}{8.4}{\sfdefault}{\mddefault}{\updefault}{\color[rgb]{0,0,0}CASL}%
}}}
\put(5356,-2356){\makebox(0,0)[lb]{\smash{\SetFigFont{7}{8.4}{\sfdefault}{\mddefault}{\updefault}{\color[rgb]{0,0,0}CoCASL}%
}}}
\put(6976,-2356){\makebox(0,0)[lb]{\smash{\SetFigFont{7}{8.4}{\sfdefault}{\mddefault}{\updefault}{\color[rgb]{0,0,0}CASL-LTL}%
}}}
\put(6121,-1726){\makebox(0,0)[lb]{\smash{\SetFigFont{7}{8.4}{\sfdefault}{\mddefault}{\updefault}{\color[rgb]{0,0,0}CSP-CASL}%
}}}
\put(7201,-3121){\makebox(0,0)[lb]{\smash{\SetFigFont{7}{8.4}{\sfdefault}{\mddefault}{\updefault}{\color[rgb]{0,0,0}SB-CASL}%
}}}
\put(5041,-3166){\makebox(0,0)[lb]{\smash{\SetFigFont{7}{8.4}{\sfdefault}{\mddefault}{\updefault}{\color[rgb]{0,0,0}HasCASL}%
}}}
\put(5086,-4336){\makebox(0,0)[lb]{\smash{\SetFigFont{7}{8.4}{\sfdefault}{\mddefault}{\updefault}{\color[rgb]{0,0,0}SubFOL$^=$}%
}}}
\put(7246,-4336){\makebox(0,0)[lb]{\smash{\SetFigFont{7}{8.4}{\sfdefault}{\mddefault}{\updefault}{\color[rgb]{0,0,0}PFOL$^=$}%
}}}
\put(6346,-4786){\makebox(0,0)[lb]{\smash{\SetFigFont{7}{8.4}{\sfdefault}{\mddefault}{\updefault}{\color[rgb]{0,0,0}FOL$^=$}%
}}}
\put(6346,-5371){\makebox(0,0)[lb]{\smash{\SetFigFont{7}{8.4}{\sfdefault}{\mddefault}{\updefault}{\color[rgb]{0,0,0}Horn$^=$}%
}}}
\put(5400,-5371){\makebox(0,0)[lb]{\smash{\SetFigFont{7}{8.4}{\sfdefault}{\mddefault}{\updefault}{\color[rgb]{0,0,0}$\bullet$}%
}}}
\put(7600,-5371){\makebox(0,0)[lb]{\smash{\SetFigFont{7}{8.4}{\sfdefault}{\mddefault}{\updefault}{\color[rgb]{0,0,0}$\bullet$}%
}}}
\put(1006,-421){\makebox(0,0)[lb]{\smash{\SetFigFont{7}{8.4}{\rmdefault}{\bfdefault}{\updefault}{\color[rgb]{0,0,0}Basic specifications}%
}}}
\put(826,-826){\makebox(0,0)[lb]{\smash{\SetFigFont{7}{8.4}{\rmdefault}{\bfdefault}{\updefault}{\color[rgb]{0,0,0}(logic-specific tools for}%
}}}
\put(826,-1231){\makebox(0,0)[lb]{\smash{\SetFigFont{7}{8.4}{\rmdefault}{\bfdefault}{\updefault}{\color[rgb]{0,0,0}CASL and extensions)}%
}}}
\put(5411,-421){\makebox(0,0)[lb]{\smash{\SetFigFont{7}{8.4}{\rmdefault}{\bfdefault}{\updefault}{\color[rgb]{0,0,0}Graph of CASL}%
}}}
\put(9591,-421){\makebox(0,0)[lb]{\smash{\SetFigFont{7}{8.4}{\rmdefault}{\bfdefault}{\updefault}{\color[rgb]{0,0,0}Structured and}%
}}}
\put(9771,-826){\makebox(0,0)[lb]{\smash{\SetFigFont{7}{8.4}{\rmdefault}{\bfdefault}{\updefault}{\color[rgb]{0,0,0}architectural}%
}}}
\put(9796,-1231){\makebox(0,0)[lb]{\smash{\SetFigFont{7}{8.4}{\rmdefault}{\bfdefault}{\updefault}{\color[rgb]{0,0,0}specifications}%
}}}
\put(10501,-1861){\makebox(0,0)[lb]{\smash{\SetFigFont{7}{8.4}{\sfdefault}{\mddefault}{\updefault}{\color[rgb]{0,0,0}Text}%
}}}
\put(10441,-2428){\makebox(0,0)[lb]{\smash{\SetFigFont{7}{8.4}{\sfdefault}{\mddefault}{\updefault}{\color[rgb]{0,0,0}\emph{Parser}}%
}}}
\put(9886,-3015){\makebox(0,0)[lb]{\smash{\SetFigFont{7}{8.4}{\sfdefault}{\mddefault}{\updefault}{\color[rgb]{0,0,0}Abstract syntax}%
}}}
\put(9936,-3592){\makebox(0,0)[lb]{\smash{\SetFigFont{7}{8.4}{\sfdefault}{\mddefault}{\updefault}{\color[rgb]{0,0,0}\emph{Static analysis}}%
}}}
\put(9641,-4169){\makebox(0,0)[lb]{\smash{\SetFigFont{7}{8.4}{\sfdefault}{\mddefault}{\updefault}{\color[rgb]{0,0,0}Development graph}%
}}}
\put(10216,-4746){\makebox(0,0)[lb]{\smash{\SetFigFont{7}{8.4}{\sfdefault}{\mddefault}{\updefault}{\color[rgb]{0,0,0}\emph{Interfaces}}%
}}}
\put(9946,-5323){\makebox(0,0)[lb]{\smash{\SetFigFont{7}{8.4}{\sfdefault}{\mddefault}{\updefault}{\color[rgb]{0,0,0}XML, ATerms}%
}}}
\put(2476,-8476){\makebox(0,0)[lb]{\smash{\SetFigFont{7}{8.4}{\sfdefault}{\mddefault}{\updefault}{\color[rgb]{0,0,0}(e.g. CCC)}%
}}}
\put(1936,-8161){\makebox(0,0)[lb]{\smash{\SetFigFont{7}{8.4}{\sfdefault}{\mddefault}{\updefault}{\color[rgb]{0,0,0}Consistency checker}%
}}}
\put(846,-7171){\makebox(0,0)[lb]{\smash{\SetFigFont{7}{8.4}{\sfdefault}{\mddefault}{\updefault}{\color[rgb]{0,0,0}(e.g. HOL-CASL)}%
}}}
\put(936,-6766){\makebox(0,0)[lb]{\smash{\SetFigFont{7}{8.4}{\sfdefault}{\mddefault}{\updefault}{\color[rgb]{0,0,0}Theorem prover}%
}}}
\put(8806,-7846){\makebox(0,0)[lb]{\smash{\SetFigFont{7}{8.4}{\sfdefault}{\mddefault}{\updefault}{\color[rgb]{0,0,0}Management of proofs \& change}%
}}}
\put(9091,-7351){\makebox(0,0)[lb]{\smash{\SetFigFont{7}{8.4}{\sfdefault}{\mddefault}{\updefault}{\color[rgb]{0,0,0}Heterogeneous proof engine}%
}}}
\put(10206,-6946){\makebox(0,0)[lb]{\smash{\SetFigFont{7}{8.4}{\rmdefault}{\bfdefault}{\updefault}{\color[rgb]{0,0,0}MAYA}%
}}}
\put(3501,-7216){\makebox(0,0)[lb]{\smash{\SetFigFont{7}{8.4}{\sfdefault}{\mddefault}{\updefault}{\color[rgb]{0,0,0}(e.g. ELAN-CASL)}%
}}}
\put(4051,-6766){\makebox(0,0)[lb]{\smash{\SetFigFont{7}{8.4}{\sfdefault}{\mddefault}{\updefault}{\color[rgb]{0,0,0}Rewriter}%
}}}
\put(5106,-1231){\makebox(0,0)[lb]{\smash{\SetFigFont{7}{8.4}{\rmdefault}{\bfdefault}{\updefault}{\color[rgb]{0,0,0}proposed extensions}%
}}}
\put(5280,-826){\makebox(0,0)[lb]{\smash{\SetFigFont{7}{8.4}{\rmdefault}{\bfdefault}{\updefault}{\color[rgb]{0,0,0}sublanguages and}%
}}}
\end{picture}


\section{Relating the Maude and CASL Logics}\label{sec:comoprh}
\input{inst}

\section{Building Development Graphs}\label{sec:dg}
%!TEX root = main.tex

Images???

We show in this section the details of the generation of development
graphs for Maude specifications.

\subsection{Parameterized modules}

As we have seen in Section \ref{subsec:pmod}, parameterized modules 
receive a list of parameters $P_i$ each one associated to a module
expression $M_i$. Since these module expressions are theories, they
can have sorts that are not instantiated yet, so to avoid clash of
names these sorts are qualified with the parameter name. These
qualifications induce an implicit morphism in the definition links
of the development graph between the theory and the parameterized
module.

\subsection{Module expressions}

Maude module expressions allow to combine and modify the information
contained in Maude modules.

\begin{itemize}
\item The summation expression of the module expressions $\mathit{ME}_1$ and
$\mathit{ME}_2$ generates a new node in the development graph
$(\mathit{ME}_1 + \mathit{ME}_2)$ with
the union of the information in both expressions. A definition link
is also created between the original expressions and the resulting one.
\item The renaming expression $\mathit{ME} * (R)$ creates a morphism with
the information given in $R$ that will be used in the link 
\item The instantiation module expression\footnote{This module expression
cannot be used in Core Maude in all places where the rest of expressions
can be used, for more information see \cite{maude-book}} assigns a view
or a parameter to each of the

Note that an instantiation generates some implicit morphisms and modifies
the ones stated in the views:
\begin{itemize}
\item When the parameter is instantiated with a view
\item As explained in the previous section, parameterized modules can
define sorts of the form \verb"name{X1, ..., Xn}", where \verb"X1, ..., Xn"
are parameter names that can be instantiated or bound by new parameters.
Thus, in addition to the morphism
\end{itemize}

\item When the module expression is a simple identifier the development
graph remains unchanged.
\end{itemize}

\subsection{Importations}

Maude allow three different kinds of module importation:

\begin{itemize}
\item Protecting, that means that \emph{no junk} and \emph{no confusion}
are added to the imported module expression.
\item Extending, that means that allows ``junk'' to be added to imported
module expression, while ``confusion'' is forbidden.
\item Including, that allows both ``junk'' and ``confusion'' to be
added to the imported module expression.
\end{itemize}

Each of these importation modes generates links with different proofs
obligations in the development graph:

\begin{itemize}
\item When a module expression $M_1$ is imported in protecting mode
by a module $M_2$ a new node $M_1'$ with the same signature that $M_1$
is generated and a free link between $M_1$ and $M_1'$
placed, while a definition link between $M_1'$ and $M_2$.
\item Extending mode generates a theorem link with the annotation
\verb"PCons", standing for proof-theoretic conservativity. This
constraint cannot be discharged in the current version of the system.
\item Including mode generates a definition link from the imported module
expression to the module that imports it.
\end{itemize}

The morphism in all these links depends of the type of module expression.
It will be an inclusion if the module expression does not contains a
renaming, while it be a morphism in other case.

\subsection{Views}

Maude views have a theory as source and either a module or a theory
as target. All the sorts and operators declared in the source theory
have to be mapped to sorts and operators in the target

A particular case of mapping between operators is the mapping between
terms, that has the general form \emph{op $e$ to term $t$}
where $e$ is a term consisting of a 
single operator applied to variables�declared either on-the-�y or with
variable declarations in the same view and the target term is any term
with variables, those in the source $e$ in the corresponding sorts
resulting from the mapping.
Since this shortcut allows to map operators with different profiles,



































\section{An example: reversing lists}\label{ex:rev}
\documentclass{entcs} \usepackage{entcsmacro}
\usepackage{graphicx}
\usepackage{mathpartir}
\sloppy
% The following is enclosed to allow easy detection of differences in
% ascii coding.
% Upper-case    A B C D E F G H I J K L M N O P Q R S T U V W X Y Z
% Lower-case    a b c d e f g h i j k l m n o p q r s t u v w x y z
% Digits        0 1 2 3 4 5 6 7 8 9
% Exclamation   !           Double quote "          Hash (number) #
% Dollar        $           Percent      %          Ampersand     &
% Acute accent  '           Left paren   (          Right paren   )
% Asterisk      *           Plus         +          Comma         ,
% Minus         -           Point        .          Solidus       /
% Colon         :           Semicolon    ;          Less than     <
% Equals        =3D           Greater than >          Question mark ?
% At            @           Left bracket [          Backslash     \
% Right bracket ]           Circumflex   ^          Underscore    _
% Grave accent  `           Left brace   {          Vertical bar  |
% Right brace   }           Tilde        ~

% A couple of exemplary definitions:

\newcommand{\Nat}{{\mathbb N}}
\newcommand{\Real}{{\mathbb R}}
\def\lastname{Hausmann and Schr\"oder}
\begin{document}
\begin{frontmatter}
  \title{Optimizing Conditional Logic (Draft)}
  \author[Bremen]{Daniel Hausmann\thanksref{ALL}\thanksref{myemail}}
  \author[Bremen]{Lutz Schr\"oder\thanksref{ALL}\thanksref{coemail}}
  \address[Bremen]{Deutsches Zentrum f\"ur k\"unstliche Intelligenz (DFKI)\\ Universit\"at Bremen, Germany}
  \thanks[ALL]{This work is supported by DFG-project no1234}
  \thanks[myemail]{Email: \href{mailto:Daniel.Hausmann@dfki.de} {\texttt{\normalshape Daniel.Hausmann@dfki.de}}}
  \thanks[coemail]{Email: \href{mailto:Lutz.Schroeder@dfki.de} {\texttt{\normalshape Lutz.Schroeder@dfki.de}}}
\begin{abstract} 
  We describe a specific optimization of the decision algorithm
of conditional logic. The general framework
  of this optimization is the setting of coalgebraic modal logic,
the actual implementation is an extension to
  the generic satisfiability solver for modal logics CoLoSS.
\end{abstract}
\begin{keyword}
  Coalgebraic modal logic, conditional logic,
satisfiability / provability, optimization by memoizing
\end{keyword}
\end{frontmatter}
\section{Introduction}\label{intro}

Introduce here :)

Structure of the paper: First we give a short introduction to
the theory of coalgebraic modal logics, the general abstract
setting in which this paper takes place. Then we describe the
satisfiability (and provability) solving algorithm and its
optimization with regards to conditional logics. Further, we 
analyze the quality of this optimization by relating
the structure of the input formula to the amount of work which may
be saved by using the proposed optimization. Furthermore,
we compare our tool with other model checkers for conditional logic
and finally we discuss to which extent the described
optimization could be used for other modal logics than just
conditional logic.
 
\section{Coalgebraic conditional logic}

Some theory here...

Needed from this section: coalgebraic modal logic etc, notion of the set of
formulae of a modal logic, provability / satisfiability (weak completeness?
see ~\ref{rem:satProv}).
Some (general) words about conditional logic?

Daniel to-do-list (in other sections): Memoizing vs. dynamic programming (cite Vardi).
Complexity of algorithm(s) (cite lutz+co. paper).

\section{The algorithm}

\subsection{The generic sequent calculus}

According to the introduced framework, we now devise a generic
algorithm to decide the provability of formulae. By
instantiating the generic algorithm to a specific modal logic
(simply by using the defining modal rule of this logic), it is
possible to obtain an algorithm to decide provability of
formulae of this specific modal logic.

\begin{definition}
A \emph{Sequent} $\Gamma$ is a set of formulae. If the provability
of a sequent is yet to be decided, it is called an \emph{open sequent}.
If the provability of a sequent has allready been shown or refuted,
the sequent is called a \emph{treated sequent}. A \emph{premise}
$\Lambda$ is a set of open sequents.
\end{definition}

\begin{definition}
\noindent A \emph{rule} $r=(\Lambda,\Gamma)$ consists of a premise
and an open sequent (usually called \emph{conclusion}).
\end{definition}
\noindent The generic sequent calculus is given by a set of rules
$\mathcal{R}_{sc}$ which consists of the finishing and the branching
rules $\mathcal{R}^b_{sc}$ (i.e. rules with no premise or more than
one premise), the linear rules $\mathcal{R}^l_{sc}$ (i.e. rules with
exactly one premise) and the modal rule $\mathcal R^m_{sc}$. The
finishing and the branching rules are presented in Figure~\ref{fig:branching}
(where $\top=\neg\bot$ and $p$ is an atom), the
linear rules are shown in Figure~\ref{fig:linear}. For now, we consider the modal
rule $\mathcal R^m_{sc}$ of the standard modal logic \textbf{K}
(as given by Figure~\ref{fig:modalK}). Note again that
it is possible to treat different modal logics with the same generic
calculus just by appropriately instantiating the modal rule.

\begin{figure}[!h]
  \begin{center}
    \begin{tabular}{| c c c |}
    \hline
      & & \\[-5pt]
      (\textsc {$\neg$F}) \inferrule{ }{\Gamma, \neg\bot} &
      (\textsc {Ax}) \inferrule{ }{\Gamma, p, \neg p} &
      (\textsc {$\wedge$}) \inferrule{\Gamma, A \\ \Gamma, B}{\Gamma, A\wedge B} \\[-5pt]
      & & \\
    \hline
    \end{tabular}
  \end{center}
  \caption{The finishing and the branching sequent rules $\mathcal{R}^b_{sc}$}
  \label{fig:branching}
\end{figure}

\begin{figure}[!h]
  \begin{center}
    \begin{tabular}{| c c |}
    \hline
      & \\[-5pt]
      (\textsc {$\neg\neg$})\inferrule{\Gamma, A}{\Gamma, \neg\neg A} &
      (\textsc {$\neg\wedge$}) \inferrule{\Gamma, \neg A, \neg B}{\Gamma, \neg(A\wedge B)} \\[-5pt]
      & \\
    \hline
    \end{tabular}
  \end{center}
  \caption{The linear sequent rules $\mathcal{R}^l_{sc}$}
  \label{fig:linear}
\end{figure}

\begin{figure}[!h]
  \begin{center}
    \begin{tabular}{| c |}
    \hline
      \\[-5pt]
      (\textsc {\textbf{K}})\inferrule{\neg A_1, \ldots , \neg A_n, A_0}
                      {\Gamma, \neg \Box A_1,\ldots,\neg \Box A_n, \Box A_0 } \\[-5pt]
      \\
    \hline
    \end{tabular}
  \end{center}
  \caption{The modal rule of \textbf{K}}
  \label{fig:modalK}
\end{figure}

Algorithm~\ref{alg:seq} makes use of the sequent rules in the following manner:
In order to show the \emph{provability} of a formula $\phi$, the 
algorithm starts with the sequent $\{\phi\}$ and tries to apply all
of the sequent rules $r\in R_{sc}$ to it
\footnote{The actual haskell
implementation of the algorithm first tries to apply the linear rules
from $\mathcal{R}^l_{sc}$ and afterwards it tries to apply the other
rules from $\mathcal{R}^b_{sc}$ and $\mathcal{R}^m_{sc}$.}.
 % explain what 'applying' means
The result of this one step of application of the rules will be a set
of premises $P$. The proving function is then recursively started on
each sequent of every premise in this set. If there is any premise
whose sequents are all provable, the application of the according rule is
justified and hence the algorithm has succeeded.

\begin{remark}
Due to $\ldots$  %what?
, the \emph{satisfiability} of a formula $\phi$ amounts to the provability
of its negation $\neg \phi$.
\label{rem:satProv}
\end{remark}

\begin{algorithm}[h]
\begin{alg}
\begin{upshape}
  Step 1: Take the input formula $\phi$, construct the initial open
  sequent $\Gamma_0 = \{\phi\}$ from it. Set i=0.\\
  Step 2: Try to apply all rules $r\in R_{sc}$ to any open sequent
  $\Gamma_i$. Let $\Lambda_i$ denote the set of the resulting open
  premises.\\
  Step 3: If $\Lambda_i$ contains the empty premise, mark the open
  sequent $\Gamma_i$ as proven. If $\Lambda_i$ contains no premise at
  all, mark the open sequent $\Gamma_i$ as refuted.\\
  Step 4: Else, add all premises from $\Lambda_i$ to the set of open
  premises and choose any premise from this set. Add all $\Gamma^j_i$
  from this premise to the set of open sequents and continue with
  Step 2.
\end{upshape}
\label{alg:seq}
\end{alg}
\end{algorithm}

\begin{proposition}
\begin{upshape}
Algorithm~\ref{alg:seq} is sound and complete w.r.t. provability. Furthermore,
Algorithm~\ref{alg:seq} has complexity X.
\end{upshape}
\end{proposition}

\begin{proof}
...
\end{proof}

\subsection{The conditional logic instance}

The genericity of the introduced sequent calculus allows us to easiliy
create instantiations of Algorithm~\ref{alg:seq} for a large variety
of modal logics:
By setting $\mathcal R^m_{sc}$ for instance to the rule shown in
Figure~\ref{fig:modalCKCEM}
(where $A_a = A_b$ is shorthand for $A_a\rightarrow A_b\wedge A_b\rightarrow A_a$), we
obtain an algorithm for deciding provability (and satisfiability) of
conditional logic. Note however, that we restrict ourselves to the
examplary conditional logic \textbf{CKCEM} for the remainder of this section
(slightly adapted versions of the optimization will work for other conditional logics).

\begin{figure}[h!]
  \begin{center}
    \begin{tabular}{| c |}
    \hline
      \\[-5pt]
      (\textsc {\textbf{CKCEM}})\inferrule{A_0 = \ldots = A_n \\ B_0,\ldots, B_j,\neg B_{j+1},\ldots,\neg B_n}
                      {\Gamma, (A_0\Rightarrow B_0),\ldots,(A_j\Rightarrow B_j),
                      \neg(A_{j+1}\Rightarrow B_{j+1}),\ldots,\neg(A_n\Rightarrow B_n) } \\[-5pt]
      \\
    \hline
    \end{tabular}
  \end{center}
  \caption{The modal rule $\textbf{CKCEM}$ of conditional logic}
  \label{fig:modalCKCEM}
\end{figure}

In the following, we use the notions of \emph{conditional antecedent} and
\emph{conditional consequent} to refer to the parameters of the modal operator of
conditional logic. 

In order to decide whether there is an
instance of the modal rule of conditional logic which can be applied to
the actual current sequent, it is necessary to create a preliminary premise
for each possible combination of equalities of all the premises of the
modal operators in this sequent. This results in $2^n$ new premises for
a sequent with $n$ top-level modal operators.

\begin{example}
Consider the sequent $\Gamma=\{(A_0\Rightarrow B_0),(A_1\Rightarrow B_1),
(A_2\Rightarrow B_2)\}$. By instantiating the modal rule, several
new premises
such as $\{\{A_0=A_1=A_2\},\{B_0,B_1,B_2\}\}$ or $\{\{A_0=A_2\},\{B_0,B_2\}\}$
may be obtained. Only after the equalities in the first sequent of each
of these premises have been shown (or refuted), it is clear whether the
current instantiation of the modal rule is actually leading to a possibly
provable premise.
\label{ex:cond}
\end{example}

It seems to be a more intelligent approach to first
partition the set of all antecedents of the top-level modal operators in the
current sequent into equivalence classes with respect to logical equality.
This partition does not only allow for a reduction of the number of newly
created open premises in each modal step,
but it also allows us to seperate the two parts of the conditional modal step
(i.e. the first part, showing the necessary equalities, and the second
part, showing that the resulting sequent - consisting only of the appropriate
conditional consequences - is actually provable itself).

\begin{example}
Consider again the sequent from Example~\ref{ex:cond}. By using the examplary
knowledge, that $A_0=A_1$, $A_1\neq A_2$ and $A_0\neq A_2$, it is immediate,
that there are just two reasonable instations of the modal rule, leading
to the two premises $\{\{B_0,B_1\}\}$ and $\{\{B_2\}\}$. For the
first of these two premises, note that it is not necessary to show the
equivalence of $A_0$ and $A_1$ again.
\end{example}


\begin{remark}
In the case of conditional logic, observe the following: Since the
modal antecedents that appear in a formula are not being changed by any
rule of the sequent calculus, it is possible to extract all possibly
relevant antecedents of a formula even \emph{before} the actual sequent
calculus iswe applied. This allows us to first compute the equivalence
classes of all the relevant antecedents and then feed this knowledge into
the actual sequent calculus:
\end{remark}

\section{The optimization}

\begin{definition}
A conditional antecedent of \emph{modal nesting depth} $i$ is a
conditional antecedent which contains at least one antecedent of
modal nesting depth $i-1$ and which contains no antecedent 
of modal nesting depth greater than $i-1$. A
conditional antecedent of nesting depth 0 is an antecedent
that does not contain any further modal operators.
Let $p_i$ denote the set of all conditional antecedents of modal
nesting depth $i$. Further, let $prems(n)$ denote the set of all
conditional antecedents of modal nesting depth at most $n$ (i.e.
$prems(n)=\bigcup_{j=1..n}^{} p_j$).
Finally, let $depth(\phi)$ denote the maximal modal nesting in
the formula $\phi$.
\end{definition}

\begin{example}
In the formula $\phi=((p_0\Rightarrow p_1) \wedge ((p_2\Rightarrow p_3)\Rightarrow p_4))
\Rightarrow (p_5\Rightarrow p_6)$,
the premise $((p_0\Rightarrow p_1) \wedge ((p_2\Rightarrow p_3)\Rightarrow p_4))$ has a
nesting depth of 2 (and not 1), $(p_0\Rightarrow p_1)$ and $(p_2\Rightarrow p_3)$ both
have a nesting depth of 1, and finally $p_0$, $p_2$ and $p_4$ all have a nesting depth of 0.
Furthermore, $prems(1)=\{p_0,p_2,p_4,(p_0\Rightarrow p_1),(p_2\Rightarrow p_3)\}$
and $depth(\phi)=3$.
\end{example}

\begin{definition}
A set $\mathcal{K}$ of treated sequents together with a function
$eval:\mathcal{K}\rightarrow \{\top,\bot\}$ is called \emph{knowledge}.
\end{definition}

We may now construct an optimized algorithm which may allow us
to decide provability (and satisfiability) of formulae more efficiently
in some cases. The optimized algorithm is constructed from two functions
(namely from the so-called \emph{pre-proving} function and the actual proving
function):

\begin{algorithm}[h]
\begin{alg}
\begin{upshape}
  Step 1: Take a formula $\phi$ and knowledge $(\mathcal{K},eval)$ as
  input. Set $\Gamma_0 = \{\phi\}$, $i=0$.\\
  Step 2: Try to apply all rules $r\in \mathcal{RO}_{sc}$
  (supplying $\mathcal{RO}^m_{sc}$ with knowledge $(\mathcal{K},eval)$)
  to any open sequent $\Gamma_i$. Let
  $\Lambda_i$ denote the set of the resulting premises.\\
  Step 3: If $\Lambda_i$ contains the empty premise, mark the open
  sequent $\Gamma_i$ as proven. If $\Lambda_i$ contains no premise at
  all, mark the open sequent $\Gamma_i$ as refuted.\\
  Step 4: Else, add all premises from $\Lambda_i$ to the set of open
  premises and choose any premise from this set. Add all $\Gamma^j_i$
  from this premise to the set of open sequents and continue with
  Step 2.
\label{alg:optSeq}
\end{upshape}
\end{alg}
\end{algorithm}

Algorithm~\ref{alg:optSeq} is very similar to Algorithm~\ref{alg:seq},
however, it uses a modified
set of rules $\mathcal{RO}_{sc}$ and it also allows for the
input of some knowledge $(\mathcal{K},eval)$ in the form of a set of sequents
together with an evaluation function. This knowledge is passed on to
the modified modal rule of conditional logic, which makes appropriate use of it.

$\mathcal{RO}_{sc}$ is obtained from $\mathcal{R}_{sc}$ by replacing
the modal rule from Figure~\ref{fig:modalCKCEM} with the modified modal
rule from Figure~\ref{fig:modalCKCEMm}.

\begin{figure}[h!]
  \begin{center}
    \begin{tabular}{| c |}
    \hline
      \\[-5pt]
      (\textsc {\textbf{CKCEM}$^m$})\inferrule{\bigwedge{}_{i,j=\{1..n\}}{eval(A_i=B_j)=\top}
                                     \\ B_0,\ldots, B_j,\neg B_{j+1},\ldots,\neg B_n}
                      {\Gamma, (A_0\Rightarrow B_0),\ldots,(A_j\Rightarrow B_j),
                      \neg(A_{j+1}\Rightarrow B_{j+1}),\ldots,\neg(A_n\Rightarrow B_n) } \\[-5pt]
      \\
    \hline
    \end{tabular}
  \end{center}
  \caption{The modified modal rule \textbf{CKCEM}$^m$ of conditional logic}
  \label{fig:modalCKCEMm}
\end{figure}

\begin{algorithm}[h]
\begin{alg}
\begin{upshape}
  Step 1: Take a formula $\phi$ as input. Set $i=0$, $\mathcal{K}_0=\emptyset$, $eval_0=\emptyset$.\\
  Step 2: Generate the set $prems_i$ of all conditional antecedents of $\phi$
  of nesting depth at most $i$. If $i<depth(\phi)$ continue
  with Step 3, else set $\mathcal{K}=\mathcal{K}_{i-1}, eval=eval_{i-1}$ and continue with Step 4.\\
  Step 3: Let $eq_i$ denote the set of all equalities $A_a = A_b$ for different
  formulae $A_a,A_b\in prems_i$. Compute
  Algorithm~\ref{alg:optSeq} ($\psi$, $(\mathcal{K}_i,eval_i)$) for all $\psi\in eq_i$.
  Set $\mathcal{K}_{i+1} = eq_i$, set $i = i + 1$. For each equality $\psi\in eq_i$,
  set $eval_{i+1}(\psi)=\top$ if the result of Algorithm~\ref{alg:optSeq} was \verb+True+
  and $eval_{i+1}(\psi)=\bot$ otherwise. Continue with Step 2.\\
  Step 4: Call Algorithm~\ref{alg:optSeq} ($\phi$, $(\mathcal{K},eval)$).
\label{alg:preprove}
\end{upshape}
\end{alg}
\end{algorithm}

Algorithm~\ref{alg:preprove} first computes the knowledge $(\mathcal{K},eval)$ about specific 
subformulae of $\phi$ and then finally checks for provability of
$\phi$ (using this knowledge): In order to show the equivalence of
two conditional antecedents of nesting depth at most $i$, we assume,
that the equalities $\mathcal{K}_{i}$ between modal antecedents of nesting depth less
than $i$ have allready been computed and the result is stored in $eval_i$; hence,
two antecedents are equal, if their equivalence is provable by
Algorithm~\ref{alg:optSeq} using only the knowledge $(\mathcal{K}_{i},eval_i)$.

\subsection{Treatment of needed equivalences only}

Since Algorithm~\ref{alg:preprove} tries to show the logical equivalence of every combination
of any two conditional antecedents that appear in $\phi$, it will have worse completion
time than Algorithm~\ref{alg:seq} on many formulae:

\begin{example}
Consider the formula 
\begin{quote}
$\phi=(((p_0\Rightarrow p_1)\Rightarrow p_2)\Rightarrow p_4)\vee
(((p_5\Rightarrow p_6)\Rightarrow p_7)\Rightarrow p_8)$.
\end{quote}
Algorithm~\ref{alg:preprove} will not only
try to show the necessary equivalences between the pairs
$(((p_0\Rightarrow p_1)\Rightarrow p_2), ((p_5\Rightarrow p_6)\Rightarrow p_7))$,
$((p_0\Rightarrow p_1), (p_5\Rightarrow p_6))$ and $(p_0,p_5)$, but it will
also try to show equivalences between any two conditional antecedents (e.g. $(p_0,
(p_5\Rightarrow p_6))$), even though these equivalences may not be needed
during the execution of Algorithm~\ref{alg:optSeq}.
\end{example}

Based on this observation, it is possible to assign a category to each pair of
antecedents that appear in it:

\begin{definition}
Two conditional antecedents $A$ and $B$ are called \emph{connected (in $\phi$)} if
they both are the antecedent of a top-level modality.

Furthermore, we define sub-antecedents $C$ and $D$ of connected antecedents $A$ and $B$
to be connected to each other, if they appear at the same position in a chain
of conditional antecedents (formally: $A=(\ldots(C\Rightarrow a_n)\ldots \Rightarrow a_0)$,
$B=(\ldots(D\Rightarrow b_n)\ldots \Rightarrow b_0)$).

Two conditional antecedents $A$ and $B$ are called \emph{independent (in $\phi$)} if
they occur at a different modal depth (even though they may have modal nestings the same depth).

The relevance of two antecedents with regard to each other is said to be
\emph{pending} if they occur at the same depth in $\phi$ but each
in the consequent of a different modal operator.
\end{definition}

Since two independent antecedents will never appear in the scope of the
same application of the modal rule, it is in no case necessary to show (or
refute) the logical equivalence of independent
conditional antecedents. Hence it suffices to focus
our attention to the connected and pending conditional antecedents. By pre-proving only
the equivalences of the connected antecedents, we can be assured that we are 
always as efficient as Algorithm~\ref{alg:optSeq}. However, it would still
be necessary to treat the equivalence of those initially pending antecedents which
turn out to be connected during the execution of Algorithm~\ref{alg:optSeq}.
This would not allow us to completely seperate to treatment of equivalences
(achieved by Algorithm~\ref{alg:preprove}) and the actual sequent calculus (realized by
Algorithm~\ref{alg:optSeq}).

If the pending antecedents are also included in the pre-proving, it is obvious, that
any possibly needed equivalence and its truth-value are allready included in $(K,eval)$
when the main proving is induced. On the other hand, we have to be aware that it
may be the case, that we show equivalences of antecedents which later turn out to
be independent.

As result of these considerations, we devise Algorithm~\ref{alg:optPreprove},
an improved version of Algorithm~\ref{alg:preprove}. The only difference is
that before proving any equivalence, Algorithm~\ref{alg:optPreprove} checks
whether the current pair of conditional antecedents is actually connected or if their
relevance to each other is pending; only then it treats the equivalence. Hence
independent pairs of antecedents remain untreated.


\begin{algorithm}[h]
\begin{alg}
\begin{upshape}
  Step 1: Take a formula $\phi$ as input. Set $i=0$, $\mathcal{K}_0=\emptyset$, $eval_0=\emptyset$.\\
  Step 2: Generate the set $prems_i$ of all conditional antecedents of $\phi$
  of nesting depth at most $i$. If $i<depth(\phi)$ continue
  with Step 3, else set $\mathcal{K}=\mathcal{K}_{i-1}, eval=eval_{i-1}$ and continue with Step 4.\\
  Step 3: Let $eq_i$ denote the set of all equalities $A_a = A_b$ for different and not independent
  pairs of formulae $A_a,A_b\in prems_i$. Compute
  Algorithm~\ref{alg:optSeq} ($\psi$, $(\mathcal{K}_i,eval_i)$) for all $\psi\in eq_i$.
  Set $\mathcal{K}_{i+1} = eq_i$, set $i = i + 1$. For each equality $\psi\in eq_i$,
  set $eval_{i+1}(\psi)=\top$ if the result of Algorithm~\ref{alg:optSeq} was \verb+True+
  and $eval_{i+1}(\psi)=\bot$ otherwise. Continue with Step 2.\\
  Step 4: Call Algorithm~\ref{alg:optSeq} ($\phi$, $(\mathcal{K},eval)$).
\label{alg:optPreprove}
\end{upshape}
\end{alg}
\end{algorithm}

\begin{proposition}
\begin{upshape}
Algorithm~\ref{alg:optPreprove} is sound and complete w.r.t. provability.
Algorithm~\ref{alg:optPreprove} has complexity X. Furthermore, Algorithm~\ref{alg:optPreprove}
allows us, to completely seperate the pre-proving of equivalences and
the actual sequent calculus.
\end{upshape}
\end{proposition}
\begin{proof}
...
\end{proof}

\section{Generalized optimization}

As previously mentioned, the demonstrated optimization is not restricted to the
case of conditional
modal logics. 

\begin{definition}
If $\Gamma$ is a sequent, we denote the set of all arguments of
top-level modalities from $\Gamma$ by $arg(\Gamma)$.
A \emph{short sequent} is a sequent which consists of just one formula which
itself is a propositional formula over a fixed maximal number of modal arguments
from $arg(\Gamma)$. In the following, we fix the maximal number of modal arguments
in short sequents to be 2.
\end{definition}

The general method of the optimization then takes the following form: 
Let $S_1,\ldots S_n$ be short sequents and assume that there is
a (w.r.t the considered modal logic) sound instance of the generic rule which
is depicted in Figure~\ref{fig:modalOpt} (where $\mathcal{S}$ is a set of any
sequents).

\begin{figure}[!h]
  \begin{center}
    \begin{tabular}{| c |}
    \hline
      \\[-5pt]
(\textsc {\textbf{Opt}}) \inferrule{ S_1 \\ \ldots \\ S_n \\ \mathcal{S} }
                      { \Gamma } \\[-5pt]
      \\
    \hline
    \end{tabular}
  \end{center}
  \caption{The general rule-scheme to which the optimization may be applied}
  \label{fig:modalOpt}
\end{figure}

We are now able to devise a final version (Algorithm~\ref{alg:genOptPreprove}) of the
optimized algorithm: Instead of considering only equivalences of conditional antecedents
for pre-proving, we now extend our attention to any short sequents over any modal arguments.

\begin{algorithm}[h]
\begin{alg}
\begin{upshape}
  Step 1: Take a formula $\phi$ as input. Set $i=0$, $\mathcal{K}_0=\emptyset$, $eval_0=\emptyset$.\\
  Step 2: Generate the set $args_i$ of all modal arguments of $\phi$
  which have nesting depth at most $i$. If $i<depth(\phi)$ continue
  with Step 3, else set $\mathcal{K}=\mathcal{K}_{i-1}, eval=eval_{i-1}$ and continue with Step 4.\\
  Step 3: Let $seq_i$ denote the set of all short sequents $S_1$ over at most two formulae
  $A_a,A_b\in args_i$. Compute Algorithm~\ref{alg:optSeq} ($\psi$, $(\mathcal{K}_i,eval_i)$) for all
  $\psi\in seq_i$. Set $\mathcal{K}_{i+1} = seq_i$, set $i = i + 1$. For each short sequent
  $\psi\in seq_i$, set $eval_{i+1}(\psi)=\top$ if the result of Algorithm~\ref{alg:optSeq} was
  \verb+True+ and $eval_{i+1}(\psi)=\bot$ otherwise. Continue with Step 2.\\
  Step 4: Call Algorithm~\ref{alg:optSeq} ($\phi$, $(\mathcal{K},eval)$).
\end{upshape}
\label{alg:genOptPreprove}
\end{alg}
\end{algorithm}

Thus it is valid to use this new Algorithm~\ref{alg:genOptPreprove} where the employed ruleset has to be
extended by the generic modified rule given by Figure~\ref{fig:modModalOpt}.

\begin{figure}[!h]
  \begin{center}
    \begin{tabular}{| c |}
    \hline
      \\[-5pt]
      (\textsc {\textbf{Opt}$^m$}) \inferrule{ eval(S_1)=\top \\ \ldots \\ eval(S_n)=\top \\ \mathcal{S} }
                      { \Gamma } \\[-5pt]
      \\
    \hline
    \end{tabular}
  \end{center}
  \caption{The general optimized rule}
  \label{fig:modModalOpt}
\end{figure}

\begin{example}
The following two cases are may be instances of the generic optimization:
\begin{enumerate}
\item (Classical modal Logics / Neighbourhood Semantics) Let $\Gamma = \{\Box A = \Box B\}$,
$n=1$, $S_1=\{A=B\}$ and $\mathcal{S}=\emptyset$. Algorithm~\ref{alg:genOptPreprove}
may be then applied whenever the following congruence rule is sound in the considered
logic:
\begin{quote}
\begin{center}
      (\textsc {\textbf{Opt}$_{Cong}$}) \inferrule{ A=B }
                      { \Box A = \Box B }
  \end{center}
\end{quote}
The according modified version of this rule is as follows:
\begin{quote}
\begin{center}
      (\textsc {\textbf{Opt}$^m_{Cong}$}) \inferrule{ {eval(A=B)=\top} }
                      { \Box A = \Box B }
  \end{center}
\end{quote}

\item (Monotone modal logics) By setting $\Gamma = \{\Box A \rightarrow \Box B\}$,
$n=1$, $S_1=\{A\rightarrow B\}$ and $\mathcal{S}=\emptyset$, we may instantiate
the generic algorithm to the case of modal logics which are monotone w.r.t. their
modal operator. So assume the following rule to be sound in  the considered modal
logic:
\begin{quote}
\begin{center}
      (\textsc {\textbf{Opt}$_{Mon}$}) \inferrule{ A\rightarrow B }
                      { \Box A \rightarrow \Box B }
  \end{center}
\end{quote}
The according modified version of this rule is as follows:
\begin{quote}
\begin{center}
      (\textsc {\textbf{Opt}$^m_{Mon}$}) \inferrule{ {eval(A\rightarrow B)=\top} }
                      { \Box A \rightarrow \Box B }
  \end{center}
\end{quote}
In the case, that \textbf{Opt}$_{Mon}$ is the only modal rule in the
considered logic (i.e. the case of plain monotone modal logic), all the
prove-work which is connected to the modal operator is shifted to the
pre-proving process. Especially matching with the modal rules
$\mathcal{RO}^m_{sc}$ becomes a mere lookup of the value of $eval$.
This means, that all calls of the sequent algorithm Algorithm~\ref{alg:optSeq}
correspond in complexity just to ordinary sat-solving of propositional logic.
Furthermore, Algorithm~\ref{alg:optSeq} will be called $|\phi|$ times. This
observation may be generalized:
\end{enumerate}
\label{ex:neighMon}
\end{example}

\begin{remark}
In the case that all modal rules of the considered logic are instances of
the generic rule (\textbf{Opt}) with $P=\emptyset$ (as seen in Example~\ref{ex:neighMon}),
the optimization does not only allow for a reduction of computing time, but
it also allows us to effectively reduce the sequent calculus to a sat-solving
algorithm.
Furthermore, the optimized algorithm will always be as efficient as the
original one in this case (since every occurence of short sequents over $arg(\Gamma)$ 
which accord to the current instantiation of the rule (\textbf{Opt}) will
have to be shown or refuted even during the course of the original algorithm).
\end{remark}

\section{Implementation}

Write a few words about the Haskell-Implementation here. Nothing interresting apart
from the Web-Interface, and that all of this is a part of CoLoSS.

\subsection{Comparing the proposed algorithms}

In order to show the relevance of the proposed optimizations, we devise three classes
of conditional formulae. Each class has a characteristic general shape, establishing its
difficulty w.r.t. the different algorithms:

Let $p_i$ denote atoms in the following.

\begin{itemize}
\item The formula \verb|bloat(|$i$\verb|)| is a full binary tree of depth $i$ (containing $2^i$ pairwise logically
inqevuilant atoms and $2^i-1$ modal antecedents):
\begin{quote}
\verb|bloat(|$i$\verb|)| = $($\verb|bloat(|$i-1$\verb|)|$)\Rightarrow($\verb|bloat(|$i-1$\verb|)|)\\
\verb|bloat(|$0$\verb|)| = $p_{rand}$
\end{quote}
Formulae from this class should show the problematic perfomance of Algorithm~\ref{alg:preprove} whenever
a formula contains many modal antecedents which appear at different depths. Since Algorithm~\ref{alg:preprove}
does not check whether pairs of modal antecedents are independent or connected, it should perform considerably worse
than Algorithm~\ref{alg:optPreprove} which only attempts to prove the logical equivalence of formulae which are not
independent. A comparison of the different algorithms w.r.t. formulae \verb|bloat(|$i$\verb|)| is depicted in
Figure~\ref{fig:benchBloat}.

\item The formula \verb|conjunct(|$i$\verb|)| is just an $i$-fold conjunction of a specific formula $A$:
\begin{quote}
\verb|conjunct(|$i$\verb|)| = $A_1\wedge\ldots\wedge A_i$\\
$A=(((p_1\vee p_0)\Rightarrow p_2)\vee((p_0\vee p_1)\Rightarrow p_2))\vee\neg(((p_0\vee p_1)\Rightarrow p_2)\vee((p_1\vee p_0)\Rightarrow p_2))$)
\end{quote}
This class contains formulae which contain logically (but not sytactically!) equivalent antecedents.
As $i$ increases, so does the amount of appearance of identical modal antecedents in different positions
of the considered formula. Hence the optimized should perform considerably better than the unoptimized
Algorithm~\ref{alg:seq}. A comparison of the different algorithms w.r.t. formulae \verb|conjunct(|$i$\verb|)| is depicted in
Figure~\ref{fig:benchConjunct}.

\item The formula \verb|explode(|$i$\verb|)| contains equivalent but not syntactically
equal and interchangingly nested modal antecedents of depth at most $i$:
\begin{quote}
\verb|explode(|$i$\verb|)| = $X^i_1\vee\ldots\vee X^i_i$\\
$X^i_1=(A^i_1\Rightarrow(\ldots(A^i_i\Rightarrow (c_1\wedge\ldots\wedge c_i))\ldots))$\\
$X^i_j=(A^i_{j\bmod i}\Rightarrow(\ldots(A^i_{(j+(i-1))\bmod i}\Rightarrow \neg c_j)\ldots))$\\
$A^i_j=p_{j \bmod i}\wedge p_{(j+(i-1)) \bmod i}$
\end{quote}
This class contains complex formulae for which the unoptimized algorithm should not be
efficient any more: Only the combined knowledge about all appearing modal antecedent allows
the proving algorithm to reach all conclusions $c_i$, and only the combined sequent
$(c_1\wedge\ldots\wedge c_i),\neg c_1,\ldots,\neg c_i$ (containing every appearing
conclusion) is provable. The algorithms perform as expected (see Figure~\ref{fig:benchExplode}).
\end{itemize}

\begin{figure}[!h]
  \begin{center}
\begin{tabular}{| l | r | r | r |}
\hline
$i$ & Algorithm~\ref{alg:seq} & Algorithm~\ref{alg:preprove} & Algorithm~\ref{alg:optPreprove}  \\
\hline
 1 & 0,008s & 0,009s & 0,010s\\
 2 & 0,008s & 0,011s & 0,010s\\
 3 & 0,009s & 0,028s & 0,014s\\
 4 & 0,010s & 0,233s & 0,022s\\
 5 & 0,011s & 2,840s & 0,087s\\
 6 & 0,013s & 33,476s & 0,590s\\
 7 & 0,019s & 402,239s & 4,989s\\
 \hline
 \end{tabular}
  \end{center}
  \caption{Results for bloat($i$)}
  \label{fig:benchBloat}
\end{figure}


\begin{figure}[!h]
  \begin{center}
\begin{tabular}{| l | r | r | r |}
\hline
$i$ & Algorithm~\ref{alg:seq} & Algorithm~\ref{alg:preprove} & Algorithm~\ref{alg:optPreprove}  \\
\hline
 1 & 0,012s & 0,013s & 0,012s\\
 2 & 0,021s & 0,015s & 0,014s\\
 3 & 0,681s & 0,021s & 0,021s\\
 4 & 131,496s & 0,048s & 0,048s\\
 5 & $>$600,000s & 0,199s & 0,201s\\
 6 & $>>$600,000s & 1,152s & 1,161s\\
 7 & $>>$600,000s & 8,746s & 8,667s\\
 8 & $>>$600,000s & 74,805s & 75,595s\\
 9 & $>>$600,000s & 730,589s & 742,357s\\
 \hline
 \end{tabular}
  \end{center}
  \caption{Results for conjunct($i$)}
  \label{fig:benchConjunct}
\end{figure}

\begin{figure}[!h]
  \begin{center}
\begin{tabular}{| l | r | r | r |}
\hline
$i$ & Algorithm~\ref{alg:seq} & Algorithm~\ref{alg:preprove} & Algorithm~\ref{alg:optPreprove}  \\
\hline
 1 & 0,009s & 0,008s & 0,009s\\
 2 & 0,011s & 0,010s & 0,010s\\
 3 & 0,028s & 0,012s & 0,014s\\
 4 & 0,268s & 0,018s & 0,018s\\
 5 & 4,555s & 0,025s & 0,027s\\
 6 & 10,785s & 0,035s & 0,039s\\
 7 & $>>$600,000s & 0,054s & 0,060s\\
 8 & $>>$600,000s & 0,079s & 0,089s\\
 9 & $>>$600,000s & 0,122s & 0,140s\\
 \hline
 \end{tabular}
  \end{center}
  \caption{Results for explode($i$)}
  \label{fig:benchExplode}
\end{figure}

The tests reveal that a significant increase of performance may be obtained through
the proposed
optimizations. In general, the implementation of the proposed algorithms (realized
in CoLoSS) has a comparable performance to
other conditional logic provers (such as CondLean). However, the authors of other
provers did not publish their benchmarking sets of formulae, so that a direct
comparison was not possible. Since other provers do not implement the optimization
strategies proposed in this paper, it is reasonable to assume, that the optimized
implementation in CoLoSS outperforms other systems at least for the 
considered examplary formulae \verb|explode(|i\verb|)| and \verb|conjunct(|i\verb|)|.

\section{Summary}  
\begin{problem}
  Finish your paper and get it to your Program Chairman on time!
\end{problem}

\section{Bibliographical references}\label{references}

\begin{thebibliography}{10}\label{bibliography}
\bibitem{cy} Civin, P., and B. Yood, \emph{Involutions on Banach
    algebras}, Pacific J. Math. \textbf{9} (1959), 415--436.
  
\end{thebibliography}

\end{document}


\section{Implementation}\label{sec:implemen}

We describe in this section how the integration described in the
previous sections has been implemented. Section \ref{subsec:abs-syntax}
shows the abstract syntax used to represent Maude modules in Haskell,
Section \ref{subsec:maude-parser} presents how these data structures
are generated in Maude, while Section \ref{sec:imp_dg} explains how
they are translated into a development graph. Finally, Section \ref{sec:imp_free}
outlines how the freeness constraints are implemented.

\subsection{Abstract syntax}\label{subsec:abs-syntax}
%!TEX root = main.tex

In this section we show how the abstract syntax for Maude specifications
is defined in Haskell. This abstract syntax is based in the Maude grammar presented
in \cite[Chapter 24]{maude-book}.

The main datatype of this abstract syntax is \verb"Spec", that
distinguishes between the different specifications available in Maude:
modules, theories and views. Although both modules and theories contain
the same information, their semantics are different and need different
constructors:

{\codesize
\begin{verbatim}
data Spec = SpecMod Module
          | SpecTh Module
          | SpecView View
          deriving (Show, Read, Ord, Eq)
\end{verbatim}
}

A \verb"Module" is composed of the identifier of the module, a list
of parameters, and a list of statements:

{\codesize
\begin{verbatim}
data Module = Module ModId [Parameter] [Statement]
            deriving (Show, Read, Ord, Eq)
\end{verbatim}
}

\noindent while a \verb"View" is composed of a module identifier, the
source and target module expressions, and a list of renamings:

{\codesize
\begin{verbatim}
data View = View ModId ModExp ModExp [Renaming]
            deriving (Show, Read, Ord, Eq)
\end{verbatim}
}

The \verb"Parameter" type contains the identifier of the parameter,
a sort (used as the parameter identifier), and its type (which is a
module expression):

{\codesize
\begin{verbatim}
data Parameter = Parameter Sort ModExp
               deriving (Show, Read, Ord, Eq)
\end{verbatim}
}

A \verb"Statement" can be any of the Maude statements:
importation, sort, subsort, and operator declarations, and
equation, membership axiom, and rule statements:

{\codesize
\begin{verbatim}
data Statement = ImportStmnt Import
               | SortStmnt Sort
               | SubsortStmnt SubsortDecl
               | OpStmnt Operator
               | EqStmnt Equation
               | MbStmnt Membership
               | RlStmnt Rule
               deriving (Show, Read, Ord, Eq)
\end{verbatim}
}

Importations consist of a module expression qualified by the type
of import:

{\codesize
\begin{verbatim}
data Import = Including ModExp
            | Extending ModExp
            | Protecting ModExp
            deriving (Show, Read, Ord, Eq)
\end{verbatim}
}

A subsort declaration keeps single relations between sorts, being
the first one the subsort and the second one the supersort:

{\codesize
\begin{verbatim}
data SubsortDecl = Subsort Sort Sort
                 deriving (Show, Read, Ord, Eq)
\end{verbatim}
}

Operator declarations are composed of the identifier of the
operator, a list of types giving the arity of the operator,
a type for its coarity, and a list of attributes:

{\codesize
\begin{verbatim}
data Operator = Op OpId [Type] Type [Attr]
              deriving (Show, Read, Ord, Eq)
\end{verbatim}
}

Membership statements consist of a term, its sort, a list
of conditions, and a list of statement attributes:

{\codesize
\begin{verbatim}
data Membership = Mb Term Sort [Condition] [StmntAttr]
                deriving (Show, Read, Ord, Eq)
\end{verbatim}
}

Equations and rules share the same elements: the lefthand
and righthand terms of the statement, a list of conditions,
and a list of statement attributes:

{\codesize
\begin{verbatim}
data Equation = Eq Term Term [Condition] [StmntAttr]
              deriving (Show, Read, Ord, Eq)

data Rule = Rl Term Term [Condition] [StmntAttr]
          deriving (Show, Read, Ord, Eq)
\end{verbatim}
}

We distinguish between the following module expressions:

\begin{itemize}
\item A single identifier:

{\codesize
\begin{verbatim}
data ModExp = ModExp ModId
\end{verbatim}
}

\item A summation, that keeps the two module
expressions involved:

{\codesize
\begin{verbatim}
            | SummationModExp ModExp ModExp
\end{verbatim}
}

\item A renaming, that contains the module expression renamed
and the list of renamings:

{\codesize
\begin{verbatim}
            | RenamingModExp ModExp [Renaming]
\end{verbatim}
}

\item An instantiation, composed of the module instantiated
and the list of view identifiers applied:

{\codesize
\begin{verbatim}
            | InstantiationModExp ModExp [ViewId]
            deriving (Show, Read, Ord, Eq)
\end{verbatim}
}

The \verb"Renaming" type distinguishes the different renamings
available in Maude:

\end{itemize}

\begin{itemize}

\item Renaming of sorts, that indicates that the first sort identifier
is changed to the second one:

{\codesize
\begin{verbatim}
data Renaming = SortRenaming Sort Sort
\end{verbatim}
}

\item Renaming of labels, where the first label is renamed to the
second one:

{\codesize
\begin{verbatim}
              | LabelRenaming LabelId LabelId
\end{verbatim}
}

\item Renaming of operators, that can be of three kinds: renaming
of operators without profile, with profile, or a map between terms,
as explained in Section \ref{subsec:views}:

{\codesize
\begin{verbatim}
              | OpRenaming1 OpId ToPartRenaming
              | OpRenaming2 OpId [Type] Type ToPartRenaming
              | TermMap Term Term
              deriving (Show, Read, Ord, Eq)
\end{verbatim}
}

\noindent where \verb"ToPartRenaming" specifies the new operator identifier
and the new attributes:

{\codesize
\begin{verbatim}
data ToPartRenaming = To OpId [Attr]
                    deriving (Show, Read, Ord, Eq)
\end{verbatim}
}

The \verb"Condition" type distinguishes between the different conditions
available in Maude, namely equational conditions, membership conditions,
matching conditions, and rewriting conditions:

{\codesize
\begin{verbatim}
data Condition = EqCond Term Term
               | MbCond Term Sort
               | MatchCond Term Term
               | RwCond Term Term
               deriving (Show, Read, Ord, Eq)
\end{verbatim}
}

We define the type \verb"Qid", a synonym of \verb"Token" that
will be used for identifiers:

{\codesize
\begin{verbatim}
type Qid = Token
\end{verbatim}
}

Terms are always represented in prefix notation. Notice that
the case of an operator applied to a list of terms is slightly different
to the Maude grammar because it also includes the type of the term.
It will be used later in the implementation to rename operators whose
profile has been specified:

{\codesize
\begin{verbatim}
data Term = Const Qid Type
          | Var Qid Type
          | Apply Qid [Term] Type
          deriving (Show, Read, Ord, Eq)
\end{verbatim}
}

Finally, the \verb"Type" distinguishes between sorts and kinds:

{\codesize
\begin{verbatim}
data Type = TypeSort Sort
          | TypeKind Kind
          deriving (Show, Read, Ord, Eq)
\end{verbatim}
}

\end{itemize}


























\subsection{Maude parsing}\label{subsec:maude-parser}
%!TEX root = main.tex

In this section we explain how the Maude specifications introduced in
Hets are parsed in order to obtain the abstract syntax described in
the previous section.

The function \verb"haskellify" receives a module (the first parameter
stands for the original module, while the second one contains the
flattened one) and returns a list of quoted identifiers identifying an
object of type \verb"Spec":

{\codesize
\begin{verbatim}
  op haskellify : Module Module -> QidList .
  ceq haskellify(M, M') = 
      'SpecMod '`( 'Module haskellifyHeader(H) ' ' 
      '`[ haskellifyImports(IL) comma(IL, SS)
          haskellifySorts(SS) comma(IL, SS, SSDS)
          haskellifySubsorts(SSDS) comma(IL, SS, SSDS, ODS)
          haskellifyOpDeclSet(M', ODS) comma(IL, SS, SSDS, ODS, MAS)
          haskellifyMembAxSet(M', MAS) comma(IL, SS, SSDS, ODS, MAS, EqS)
          haskellifyEqSet(M', EqS) '`] '`) '\n '@#$endHetsSpec$#@ '\n
    if fmod H is IL sorts SS . SSDS ODS MAS EqS endfm := M .
\end{verbatim}
}

This function prints the keyword \verb"SpecMod" and uses the \verb"haskellify"
auxiliary functions to print the different parts of the module.
The functions \verb"comma" introduce a comma whenever it is necessary.
Since all the ``haskellify'' functions are very similar, we describe
them by using \verb"haskellifyImports" as example. This function traverses
all the imports in the list and applies the auxiliary function
\verb"haskellifyImport" to each of them:

{\codesize
\begin{verbatim}
  op haskellifyImports : ImportList -> QidList .
  eq haskellifyImports(nil) = nil .
  eq haskellifyImports(I IL) = 'ImportStmnt ' '`( haskellifyImport(I) '`)
                               comma(IL) haskellifyImports(IL) .
\end{verbatim}
}

This auxiliary function distinguishes between the importation modes,
using the appropriate keyword for each of them:

{\codesize
\begin{verbatim}
  op haskellifyImport : Import -> QidList .
  eq haskellifyImport(protecting ME .) = 'Protecting haskellifyME(ME) .
  eq haskellifyImport(including ME .) = 'Including haskellifyME(ME) .
  eq haskellifyImport(extending ME .) = 'Extending haskellifyME(ME) .
\end{verbatim}
}

\noindent where \verb"haskellifyME" is in charge of printing the 
module expression.



\subsection{Logic}\label{sec:logic_imp}
\input{logic-imp}

\subsection{Development Graph}\label{sec:imp_dg}
%!TEX root = main.tex

We describe in this section the main functions used to draw the
development graph for Maude specifications. The main function is
\verb"anaMaudeFile", that receives a record of all the options received
from the command line (of type \verb"HetcatsOpts") and the path of
the Maude file to be parsed and return a pair with the library
name and its environment.

{\codesize
\begin{verbatim}
anaMaudeFile :: HetcatsOpts -> FilePath -> IO (Maybe (LIB_NAME, LibEnv))
anaMaudeFile _ file = do
    dg <- directMaudeParsing file
    let name = Lib_id $ Direct_link "Maude_Module" nullRange
    return $ Just (name, Map.singleton name dg)
\end{verbatim}
}

This environment is computed with the function
\verb"directMaudeParsing", that receives the file path and returns
a development graph. This analysis is performed in different stages:
First, the parser described in Section \ref{subsec:lex-parser}
Then, Maude is executed with \verb"runInteractiveCommand" to use
the parser described in Section \ref{subsec:maude-parser}

{\codesize
\begin{verbatim}
directMaudeParsing :: FilePath -> IO DGraph
directMaudeParsing fp = do
              ns <- parse fp
              let ns' = either (\ _ -> []) id ns
              (hIn, hOut, _, _) <- runInteractiveCommand maudeCmd
              hPutStrLn hIn $ "load " ++ fp
              ms <- traverseNames hIn hOut ns'
              hPutStrLn hIn "in Maude/hets.prj"
              psps <- predefinedSpecs hIn hOut
              sps <- traverseSpecs hIn hOut ms
              hClose hIn
              hClose hOut
              return $ insertSpecs (psps ++ sps) Map.empty Map.empty [] emptyDG
\end{verbatim}
}

We explain briefly the main data structures used during the generation of
the development graph:

\begin{itemize}

\item The type \verb"ParamSort" defines a pair with a symbol representing
a sort and a list of tokens indicating the parameters present in the sort,
so for example the sort \verb"List{X, Y}" generates the pair
\verb"(List{X, Y}, [X,Y])":

{\codesize
\begin{verbatim}
type ParamSort = (Symbol, [Token])
\end{verbatim}
}

\item The information of each node introduced
in the development graph is stored in the tuple \verb"ProcInfo", that
contains the following information:

\begin{itemize}
\item The identifier of the node.
\item The signature of the node.
\item A list of symbols standing for the sorts that are not instantiated.
\item A list of triples with information about the parameters of the
specification, namely the name of the parameter, the name of the theory
and the list of not instantiated sorts from this theory.
\item A list with information about the parameterized sorts.
\end{itemize}

{\codesize
\begin{verbatim}
type ProcInfo = (Node, Sign, Symbols, [(Token, Token, Symbols)], [ParamSort])
\end{verbatim}
}

\item Each \verb"ProcInfo" tuple is associated to its corresponding module
expression in the \verb"TokenInfoMap" map:

{\codesize
\begin{verbatim}
type TokenInfoMap = Map.Map Token ProcInfo
\end{verbatim}
}

\item When a module expression is parsed a \verb"ModExpProc" tuple is
returned, containing the following information:

\begin{itemize}
\item The identifier of the module expression.
\item The \verb"TokenInfoMap" updated with the data in the module
expression.
\item The morphism associated to the module expression.
\item The list of sorts parameterized in this module expression.
\item The development graph thus far.
\end{itemize}

{\codesize
\begin{verbatim}
type ModExpProc = (Token, TokenInfoMap, Morphism, [ParamSort], DGraph)
\end{verbatim}
}

\item We distinguish the type of importation statement with the type
\verb"ImportType":

{\codesize
\begin{verbatim}
data ImportType = Pr | Ex | Inc
\end{verbatim}
}

\item When parsing an importation statement we return a tuple containing:

\begin{itemize}

\item The type of importation.
\item The identifier of the module expression imported.
\item The updated map with the information associated with each
module expression.
\item The morphism associated to the module expression.
\item The list of sorts parameterized in this module expression.
\item The updated development graph.
\end{itemize}

{\codesize
\begin{verbatim}
type ImportProc = (ImportType, Token, TokenInfoMap, Morphism, [ParamSort], DGraph)
\end{verbatim}
}

\item When parsing a list of importation statements we return:

\begin{itemize}
\item The list of parameter information: the name of the parameter,
the name of the theory and the sorts not instantiated.
\item The updated \verb"TokenInfoMap" map.
\item The list of morphisms associated with each parameter.
\item The updated development graph.
\end{itemize}

{\codesize
\begin{verbatim}
type ParamInfo = ([(Token, Token, Symbols)], TokenInfoMap, [Morphism], DGraph)
\end{verbatim}
}

\item Data about views is kept in a separated way from data about theories
and modules. The \verb"ViewMap" map associates to each view identifier a
tuple with:

\begin{itemize}
\item The identifier of the target node of the view.
\item The morphism generated by the view.
\item A Boolean value indicating whether the target is a theory
(\verb"True") or a module (\verb"False").
\end{itemize}

{\codesize
\begin{verbatim}
type ViewMap = Map.Map Token (Node, Token, Morphism, [Renaming], Bool)
\end{verbatim}
}

\item Finally, we describe the tuple used to return the data structures
updated when a specification or a view is introduced in the development
graph. It contains:

\begin{itemize}
\item The updated \verb"TokenInfoMap".
\item The updated \verb"ViewMap".
\item A list of tokens indicating
\item The new development graph.
\end{itemize}

{\codesize
\begin{verbatim}
type InsSpecRes = (TokenInfoMap, ViewMap, [Token], DGraph)
\end{verbatim}
}

\end{itemize}

The function \verb"insertSpecs" traverses the specifications updating the
data structures and the development graph with \verb"insertSpec":

{\codesize
\begin{verbatim}
insertSpecs :: [Spec] -> TokenInfoMap -> ViewMap -> [Token] -> DGraph -> DGraph
insertSpecs [] _ _ _ dg = dg
insertSpecs (s : ss) tim vm ths dg = insertSpecs ss tim' vm' ths' dg'
              where (tim', vm', ths', dg') = insertSpec s tim vm ths dg
\end{verbatim}
}

The behavior of \verb"insertSpec" is different for each type of Maude
specifications. When the introduced specification is a module, the
following actions are performed:

\begin{itemize}
\item The parameters are parsed as follows:

\begin{itemize}
\item The list of parameters declarations is obtained with the auxiliary
function \verb"getParams".
\item These declarations are processed with \verb"processParameters",
that returns a tuple of type \verb"ParamInfo" shown above.
\item Given the parameters names, we traverse the list of sorts to check
if the module defines parameterized sorts with \verb"getSortsParameterizedBy".
\item The links between the theories in the parameters and the current modules
are created with \verb"createEdgesParams".
\end{itemize}

\item The importations are handled as follows:

\begin{itemize}
\item The importation statements are obtained with \verb"getImportsSorts".
Although this function also returns the sorts declared in the module, in
this case they are not needed and its value is ignored.
\item These importations are handled by \verb"processImports", that
returns a list of \verb"ImportProc" containing the information of each
parameter.
\item The definition link generated by the imports are created with
\verb"createEdgesImports".
\end{itemize}

\item The final signature is obtained with \verb"sign_union_morphs"
by merging the signature in the current module with the one obtained
from the morphisms from the parameters and the imports.

\end{itemize}

{\codesize
\begin{verbatim}
insertSpec :: Spec -> TokenInfoMap -> ViewMap -> [Token] -> DGraph -> InsSpecRes
insertSpec (SpecMod sp_mod) tim vm ths dg = (tim4, vm, ths, dg5)
           where ps = getParams sp_mod
                 (pl, tim1, morphs, dg1) = processParameters ps tim dg
                 top_sg = Maude.Sign.fromSpec sp_mod
                 paramSorts = getSortsParameterizedBy (paramNames ps) 
                                                      (Set.toList $ sorts top_sg)
                 (il, _) = getImportsSorts sp_mod
                 ips = processImports tim1 vm dg1 il
                 (tim2, dg2) = last_da ips (tim1, dg1)
                 sg = sign_union_morphs morphs $ sign_union top_sg ips
                 ext_sg = makeExtSign Maude sg
                 nm_sns = map (makeNamed "") $ Maude.Sentence.fromSpec sp_mod
                 sens = toThSens nm_sns
                 gt = G_theory Maude ext_sg startSigId sens startThId
                 tok = HasName.getName sp_mod
                 name = makeName tok
                 (ns, dg3) = insGTheory dg2 name DGBasic gt
                 tim3 = Map.insert tok (getNode ns, sg, [], pl, paramSorts) tim2
                 (tim4, dg4) = createEdgesImports tok ips sg tim3 dg3
                 dg5 = createEdgesParams tok pl morphs sg tim4 dg4
\end{verbatim}
}

When the specification inserted is a theory the process varies slightly:

\begin{itemize}
\item Theories cannot be parameterized in Core Maude, so the parameter
handling is not required.
\item
\end{itemize}

{\codesize
\begin{verbatim}
insertSpec (SpecTh sp_th) tim vm ths dg = (tim3, vm, tok : ths, dg3)
            where (il, ss1) = getImportsSorts sp_th
                  ips = processImports tim vm dg il
                  ss2 = getThSorts ips
                  (tim1, dg1) = last_da ips (tim, dg)
                  sg = sign_union (Maude.Sign.fromSpec sp_th) ips
                  ext_sg = makeExtSign Maude sg
                  nm_sns = map (makeNamed "") $ Maude.Sentence.fromSpec sp_th
                  sens = toThSens nm_sns
                  gt = G_theory Maude ext_sg startSigId sens startThId
                  tok = HasName.getName sp_th
                  name = makeName tok
                  (ns, dg2) = insGTheory dg1 name DGBasic gt
                  tim2 = Map.insert tok (getNode ns, sg, ss1 ++ ss2, [], []) tim1
                  (tim3, dg3) = createEdgesImports tok ips sg tim2 dg2
\end{verbatim}
}



















\subsection{Comorphism}\label{sec:comorphism}
%!TEX root = main.tex

We show in this section the main functions used to implement the comorphism
from Maude to \CASL described in Section \ref{sec:comoprh}.
The main function in this transformation is
\verb"maude2casl", that computes the \CASL signature and sentences
given the Maude signature and sentences:

{\codesize
\begin{verbatim}
maude2casl :: MSign.Sign -> [Named MSentence.Sentence]
              -> (CSign.CASLSign, [Named CAS.CASLFORMULA])
\end{verbatim}
}

This function splits the work into different stages:

\begin{itemize}

\item
The function \verb"rewPredicates" generates the \verb"rew" predicates for
each sort to simulate the rewrite rules in the Maude specification.

\item
The function \verb"rewPredicatesSens" creates the formulas associated to
the \verb"rew" predicates created above, stating that they are reflexive
and transitive.

\item
The \CASL operators are obtained from the Maude operators:

\begin{itemize}
\item
The function \verb"translateOps" splits the Maude operator map
into a tuple of \CASL operators and \CASL associative operators.

\item
Since \CASL does not allow the definition
of polymorphic operators, these operators are removed from the map
with \verb"deleteUniversal" and for each one of these Maude operators we
create a set of \CASL operators with all the possible profiles with
\verb"universalOps".

\end{itemize}

\item \CASL sentences are obtained from the Maude sentences and from
predefined \CASL libraries:

\begin{itemize}
\item In the computation of the \CASL formulas we split Maude sentences in
equations defined without the \verb"owise" attribute, equations defined
with \verb"owise", and the rest of statements with the function
\verb"splitOwiseEqs".
\item The equations defined without the \verb"owise" attribute are
translated as universally quantified equations, as shown in Section
\ref{sec:comoprh}, with \verb"noOwiseSen2Formula".
\item Equations with the \verb"owise" attribute are translated using
a negative existential quantification, as we will show later, with
the function \verb"owiseSen2Formula". This function requires as additional
parameter the definition of the formulas defined without the \verb"owise"
attribute, in order to state that the equations defined with \verb"owise"
are applied when the rest of possible equations cannot.
\item The rest of statements, namely memberships and rules, are translated
with the function \verb"mb_rl2formula".
\item There are some built-in operators in Maude that are not defined by
means of equations. To allow the user to reason about them we provide
some libraries with the definitions of these operators as \CASL formulas,
obtained with \verb"loadLibraries".
\end{itemize}

\item Finally, the \CASL symbols are created:

\begin{itemize}
\item The kinds are translated to symbols with \verb"kinds2syms".
\item The operators are translated with \verb"ops2symbols".
\item The symbol predicates are obtained with \verb"preds2syms".
\end{itemize}

\end{itemize}

{\codesize
\begin{verbatim}
maude2casl msign nsens = (csign { CSign.sortSet = cs,
                            CSign.sortRel = sbs',
                            CSign.opMap = cops',
                            CSign.assocOps = assoc_ops,
                            CSign.predMap = preds,
                            CSign.declaredSymbols = syms }, new_sens)
   where csign = CSign.emptySign ()
         ss = MSign.sorts msign
         ss' = Set.map sym2id ss
         mk = kindMapId $ MSign.kindRel msign
         sbs = MSign.subsorts msign
         sbs' = maudeSbs2caslSbs sbs mk
         cs = Set.union ss' $ kindsFromMap mk
         preds = rewPredicates cs
         rs = rewPredicatesSens cs
         ops = deleteUniversal $ MSign.ops msign
         ksyms = kinds2syms cs
         (cops, assoc_ops, _) = translateOps mk ops
         cops' = universalOps cs cops $ booleanImported ops
         rs' = rewPredicatesCongSens cops'
         pred_forms = loadLibraries (MSign.sorts msign) ops
         ops_syms = ops2symbols cops'
         (no_owise_sens, owise_sens, mbs_rls_sens) = splitOwiseEqs nsens
         no_owise_forms = map (noOwiseSen2Formula mk) no_owise_sens
         owise_forms = map (owiseSen2Formula mk no_owise_forms) owise_sens
         mb_rl_forms = map (mb_rl2formula mk) mbs_rls_sens
         preds_syms = preds2syms preds
         syms = Set.union ksyms $ Set.union ops_syms preds_syms
         new_sens = concat [rs, rs', no_owise_forms, owise_forms,
                            mb_rl_forms, pred_forms]
\end{verbatim}
}

The \verb"rew" predicates are declared with the function
\verb"rewPredicates", that traverses the set of sorts applying
the function \verb"rewPredicate":

{\codesize
\begin{verbatim}
rewPredicates :: Set.Set Id -> Map.Map Id (Set.Set CSign.PredType)
rewPredicates = Set.fold rewPredicate Map.empty
\end{verbatim}
}

This function defines a binary predicate using as name the constant
\verb"rewID" and the sort as type of the arguments:

{\codesize
\begin{verbatim}
rewPredicate :: Id -> Map.Map Id (Set.Set CSign.PredType)
                -> Map.Map Id (Set.Set CSign.PredType)
rewPredicate sort m = Map.insertWith (Set.union) rewID ar m
   where ar = Set.singleton $ CSign.PredType [sort, sort]
\end{verbatim}
}

Once these predicates has been declared, we have to introduce
formulas to state their properties. The function \verb"rewPredicatesSens"
accomplishes this task by traversing the set of sorts and applying
\verb"rewPredicateSens":

{\codesize
\begin{verbatim}
rewPredicatesSens :: Set.Set Id -> [Named CAS.CASLFORMULA]
rewPredicatesSens = Set.fold rewPredicateSens []
\end{verbatim}
}

This function generates the formulas for each sort:

{\codesize
\begin{verbatim}
rewPredicateSens :: Id -> [Named CAS.CASLFORMULA] -> [Named CAS.CASLFORMULA]
rewPredicateSens sort acc = ref : trans : acc
        where ref = reflSen sort
              trans = transSen sort
\end{verbatim}
}

We describe the formula for the reflexivity, being the transitivity
analogous. A new variable of the required sort is created with
the auxiliary function \verb"newVar", then the qualified predicate
name is created with the \verb"rewID" constant and applied to the
variable. Finally, the formula is named with the prefix \verb"rew_refl_"
followed by the name of the sort:

{\codesize
\begin{verbatim}
reflSen :: Id -> Named CAS.CASLFORMULA
reflSen sort = makeNamed name $ quantifyUniversally form
        where v = newVar sort
              pred_type = CAS.Pred_type [sort, sort] nullRange
              pn = CAS.Qual_pred_name rewID pred_type nullRange
              form = CAS.Predication pn [v, v] nullRange
              name = "rew_refl_" ++ show sort
\end{verbatim}
}

The function \verb"translateOps" traverses the map of Maude operators,
applying to each of them the function \verb"translateOpDeclSet":

{\codesize
\begin{verbatim}
translateOps :: IdMap -> MSign.OpMap -> OpTransTuple
translateOps im = Map.fold (translateOpDeclSet im) (Map.empty, Map.empty, Set.empty)
\end{verbatim}
}

Since the values in the Maude operator map are sets of operator declarations
the auxiliary function \verb"translateOpDeclSet" has to traverse these sets, applying
\verb"translateOpDecl" to each operator declaration:

{\codesize
\begin{verbatim}
translateOpDeclSet :: IdMap -> MSign.OpDeclSet -> OpTransTuple -> OpTransTuple
translateOpDeclSet im ods tpl = Set.fold (translateOpDecl im) tpl ods
\end{verbatim}
}

The function \verb"translateOpDecl" receives an operator declaration,
that consists of all the operators declared with the same profile at
the kind level. The function traverses these operators, transforming
them into \CASL operators with the function \verb"ops2pred" and returning
a tuple containing the operators, the associative operators, and the
constructors:

{\codesize
\begin{verbatim}
translateOpDecl :: IdMap -> MSign.OpDecl -> OpTransTuple -> OpTransTuple
translateOpDecl im (syms, ats) (ops, assoc_ops, cs) = case tl of
                      [] -> (ops', assoc_ops', cs')
                      _ -> translateOpDecl im (syms', ats) (ops', assoc_ops', cs')
      where sym = head $ Set.toList syms
            tl = tail $ Set.toList syms
            syms' = Set.fromList tl
            (cop_id, ot, _) = fromJust $ maudeSym2CASLOp im sym
            cop_type = Set.singleton ot
            ops' = Map.insertWith (Set.union) cop_id cop_type ops
            assoc_ops' = if any MAS.assoc ats
                         then Map.insertWith (Set.union) cop_id cop_type assoc_ops
                         else assoc_ops
            cs' = if any MAS.ctor ats
                  then Set.insert (Component cop_id ot) cs
                  else cs
\end{verbatim}
}

As said above, Maude equations that are not defined with the \verb"owise"
attribute are translated to \CASL with \verb"noOwiseSen2Formula". This
function extracts the current equation from the named sentence, translates
it with \verb"noOwiseEq2Formula" and creates a new named sentence
with the resulting formula:

{\codesize
\begin{verbatim}
noOwiseSen2Formula ::  IdMap -> Named MSentence.Sentence -> Named CAS.CASLFORMULA
noOwiseSen2Formula im s = s'
       where MSentence.Equation eq = sentence s
             sen' = noOwiseEq2Formula im eq
             s' = s { sentence = sen' }
\end{verbatim}
}

The function \verb"noOwiseEq2Formula" distinguishes whether the equation
is conditional or not. In both cases, the Maude terms in the equation
are translated into \CASL terms with \verb"maudeTerm2caslTerm", and a
strong equation is used to create a formula. If the equation has no
conditions this formula is universally quantified and returned as result,
while if it has conditions each of them generates a formula and their
conjunction, computed with \verb"conds2formula", will be used as premise
of the equality formula:

{\codesize
\begin{verbatim}
noOwiseEq2Formula :: IdMap -> MAS.Equation -> CAS.CASLFORMULA
noOwiseEq2Formula im (MAS.Eq t t' [] _) = quantifyUniversally form
      where ct = maudeTerm2caslTerm im t
            ct' = maudeTerm2caslTerm im t'
            form = CAS.Strong_equation ct ct' nullRange
noOwiseEq2Formula im (MAS.Eq t t' conds@(_:_) _) = quantifyUniversally form
      where ct = maudeTerm2caslTerm im t
            ct' = maudeTerm2caslTerm im t'
            conds_form = conds2formula im conds
            concl_form = CAS.Strong_equation ct ct' nullRange
            form = createImpForm conds_form concl_form
\end{verbatim}
}

\verb"maudeTerm2caslTerm" is defined for each Maude term:

\begin{itemize}

\item Variables are translated into qualified \CASL variables, and their
sort is changed to the corresponding kind:

{\codesize
\begin{verbatim}
maudeTerm2caslTerm :: IdMap -> MAS.Term -> CAS.CASLTERM
maudeTerm2caslTerm im (MAS.Var q ty) = CAS.Qual_var q ty' nullRange
        where ty' = maudeType2caslSort ty im
\end{verbatim}
}

\item Constants are translated into the applications of the constant
name to the empty list of arguments, using again the kind of the
given sort:

{\codesize
\begin{verbatim}
maudeTerm2caslTerm im (MAS.Const q ty) = CAS.Application op [] nullRange
        where name = token2id q
              ty' = maudeType2caslSort ty im
              op_type = CAS.Op_type CAS.Total [] ty' nullRange
              op = CAS.Qual_op_name name op_type nullRange
\end{verbatim}
}

\item The application of an operator to a list of terms is translated
into another application, translating recursively the arguments into
valid \CASL terms:

{\codesize
\begin{verbatim}
maudeTerm2caslTerm im (MAS.Apply q ts ty) = CAS.Application op tts nullRange
        where name = token2id q
              tts = map (maudeTerm2caslTerm im) ts
              ty' = maudeType2caslSort ty im
              types_tts = getTypes tts
              op_type = CAS.Op_type CAS.Total types_tts ty' nullRange
              op = CAS.Qual_op_name name op_type nullRange
\end{verbatim}
}

\end{itemize}

The conditions are translated into a conjunction with \verb"conds2formula",
that traverses the conditions applying \verb"cond2formula" to each of them,
and then creates the conjunction of the obtained formulas:

{\codesize
\begin{verbatim}
conds2formula :: IdMap -> [MAS.Condition] -> CAS.CASLFORMULA
conds2formula im conds = CAS.Conjunction forms nullRange
        where forms = map (cond2formula im) conds
\end{verbatim}
}

\begin{itemize}

\item Both equality and matching conditions are translated into
strong equations:

{\codesize
\begin{verbatim}
cond2formula :: IdMap -> MAS.Condition -> CAS.CASLFORMULA
cond2formula im (MAS.EqCond t t') = CAS.Strong_equation ct ct' nullRange
       where ct = maudeTerm2caslTerm im t
             ct' = maudeTerm2caslTerm im t'
cond2formula im (MAS.MatchCond t t') = CAS.Strong_equation ct ct' nullRange
       where ct = maudeTerm2caslTerm im t
             ct' = maudeTerm2caslTerm im t'
\end{verbatim}
}

\item Membership conditions are translated into \CASL memberships by translating
the term and the sort:

{\codesize
\begin{verbatim}
cond2formula im (MAS.MbCond t s) = CAS.Membership ct s' nullRange
      where ct = maudeTerm2caslTerm im t
            s' = token2id $ getName s
\end{verbatim}
}

\item Rewrite conditions are translated into formulas by using both terms
as arguments of the corresponding \verb"rew" predicate: 

{\codesize
\begin{verbatim}
cond2formula im (MAS.RwCond t t') = CAS.Predication pred_name [ct, ct'] nullRange
       where ct = maudeTerm2caslTerm im t
             ct' = maudeTerm2caslTerm im t'
             ty = token2id $ getName $ MAS.getTermType t
             kind = Map.findWithDefault (errorId "rw cond to formula") ty im
             pred_type = CAS.Pred_type [kind, kind] nullRange
             pred_name = CAS.Qual_pred_name rewID pred_type nullRange
\end{verbatim}
}

\end{itemize}

The equations defined with the \verb"owise" attribute are translated
with \verb"owiseSen2Formula", that traverses them and applies
\verb"owiseEq2Formula" to the inner equation:

{\codesize
\begin{verbatim}
owiseSen2Formula ::  IdMap -> [Named CAS.CASLFORMULA]
                     -> Named MSentence.Sentence -> Named CAS.CASLFORMULA
owiseSen2Formula im owise_forms s = s'
       where MSentence.Equation eq = sentence s
             sen' = owiseEq2Formula im owise_forms eq
             s' = s { sentence = sen' }
\end{verbatim}
}

This function receives all the formulas defined without the \verb"owise"
attribute and, for each formula with the same operator in the lefthand
side that the current equation (obtained with \verb"getLeftApp"), it
generates with \verb"existencialNegationOtherEqs" a negative existential
quantification stating that the arguments do not match or the condition
does not hold that is used as premise of the equation:

{\codesize
\begin{verbatim}
owiseEq2Formula :: IdMap -> [Named CAS.CASLFORMULA] -> MAS.Equation
                   -> CAS.CASLFORMULA
owiseEq2Formula im no_owise_form eq = form
      where (eq_form, vars) = noQuantification $ noOwiseEq2Formula im eq
            (op, ts, _) = fromJust $ getLeftApp eq_form
            ex_form = existencialNegationOtherEqs op ts no_owise_form
            imp_form = createImpForm ex_form eq_form
            form = CAS.Quantification CAS.Universal vars imp_form nullRange
\end{verbatim}
}

%The rest of Maude sentences are translated in a similar way to the one
%shown for the conditions above.
%The rest of the sentences generated in the comorphism are obtained
%from external libraries with the function \verb"readLib". We describe
%below how sentences defining the behavior of the natural numbers are
%loaded: once the library is obtained, we transform the theory sentences
%into named sentences with \verb"toNamedList" and then we ``coerce''
%them with \verb"coerceSens" to indicate that they are \CASL sentences.
%Finally, the sentence about the generators is filtered and the result
%returned:

%{\codesize
%\begin{verbatim}
%loadNaturalNatSens :: [Named CAS.CASLFORMULA]
%loadNaturalNatSens = 
%       let lib = head $ unsafePerformIO $ readLib "Maude/MaudeNumbers.casl"
%       in case lib of
%           G_theory lid _ _ thSens _ -> let sens = toNamedList thSens
%                                        in do
%                                            sens' <- coerceSens lid CASL "" sens
%                                            filter (not . ctorCons) sens'
%\end{verbatim}
%}

The translation from sorts, operators and predicates to symbols
works in a similar way to the transformations shown above, so we only
describe  how the predicate symbols are obtained. The function
\verb"preds2syms" traverses the map of predicates and inserts each
obtained symbol into the set with \verb"pred2sym":

{\codesize
\begin{verbatim}
preds2syms :: Map.Map Id (Set.Set CSign.PredType) -> Set.Set CSign.Symbol
preds2syms = Map.foldWithKey pred2sym Set.empty
\end{verbatim}
}

This function traverses the set of predicate types and creates the
symbol corresponding to each one with \verb"createSym4id":

{\codesize
\begin{verbatim}
pred2sym :: Id -> Set.Set CSign.PredType -> Set.Set CSign.Symbol -> Set.Set CSign.Symbol
pred2sym pn spt acc = Set.fold (createSym4id pn) acc spt
\end{verbatim}
}

\verb"createSym4id" generates the symbol and inserts it into the
accumulated set:

{\codesize
\begin{verbatim}
createSym4id :: Id -> CSign.PredType -> Set.Set CSign.Symbol -> Set.Set CSign.Symbol
createSym4id pn pt acc = Set.insert sym acc
      where sym = CSign.Symbol pn $ CSign.PredAsItemType pt
\end{verbatim}
}























\subsection{Freeness constraints}\label{sec:imp_free}
\input{freeness}

\section{Concluding remarks and future work}\label{sec:conclusions}
%!TEX root = main.tex

We have presented how Maude has been integrated into
\Hets, a parsing, static analysis, and proof management tool that
combines various tools for different specification languages. To
achieve this integration, we consider preordered algebra semantics for
Maude and define an institution comorphism from Maude to \CASL.  This
integration allows to prove properties of Maude specifications like
those expressed in Maude views. We have also implemented a
normalization of the development graphs that allows us to prove
freeness constraints. We have used this transformation to connect
Maude to Isabelle \cite{Isabelle02}, a Higher Order Logic prover, and
have demonstrated a small example proof about reversal of lists.
Moreover, this encoding is suited for proofs of e.g.\ extensionality
of sets, which require first-order logic, going beyond the abilities
of existing Maude provers like ITP.

Since interactive proofs are often not easy to conduct, future work
will make proving more efficient by adopting automated induction
strategies like rippling~\cite{DBLP:conf/tphol/DixonF04}.  We also
have the idea to use the automatic first-order prover SPASS for
induction proofs by integrating special induction strategies directly
into \Hets.

We have also studied the possible comorphisms from \CASL to Maude. We
distinguish whether the formulas in the source theory are confluent and
terminating or not. In the first case, that we plan to check with the
Maude termination~\cite{MTT08} and confluence checker \cite{ChurchRoss10},
we map formulas to equations,
whose execution in Maude is more efficient, while in the second case
we map formulas to rules.

Finally, we also plan to relate \Hets' Modal Logic and Maude models in order to use
the Maude model checker \cite[Chapter 13]{maude-book} for linear temporal
logic.


\textbf{Acknowledgments} We wish to thank Francisco Dur\'an for discussions 
regarding freeness in Maude, Martin K\"uhl for cooperation
on the implementation of the system presented here,
and Maksym Bortin for help with the Isabelle proofs.
This work has been
supported by the German Federal Ministry of Education and Research
(Project 01 IW 07002 FormalSafe), the German Research Council (DFG)
under grant KO-2428/9-1, the Comunidad de Madrid project \emph{PROMETIDOS}
(S2009/TIC--1465), and the MICINN Spanish project
\emph{DESAFIOS10} (TIN2009-14599-C03-01).


%strat and frozen attributes. Co-algebraic constructions.

{\small
\bibliographystyle{abbrv}
\bibliography{alberto}
}

\end{document}
