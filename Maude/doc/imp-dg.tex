%!TEX root = main.tex

We describe in this section the main functions used to draw the
development graph for Maude specifications. The main function is
\verb"anaMaudeFile", that receives a record of all the options received
from the command line (of type \verb"HetcatsOpts") and the path of
the Maude file to be parsed and return a pair with the library
name and its environment.

{\codesize
\begin{verbatim}
anaMaudeFile :: HetcatsOpts -> FilePath -> IO (Maybe (LIB_NAME, LibEnv))
anaMaudeFile _ file = do
    dg <- directMaudeParsing file
    let name = Lib_id $ Direct_link "Maude_Module" nullRange
    return $ Just (name, Map.singleton name dg)
\end{verbatim}
}

This environment is computed with the function
\verb"directMaudeParsing", that receives the file path and returns
a development graph. This analysis is performed in different stages:
First, the parser described in Section \ref{subsec:lex-parser}
Then, Maude is executed with \verb"runInteractiveCommand" to use
the parser described in Section \ref{subsec:maude-parser}

{\codesize
\begin{verbatim}
directMaudeParsing :: FilePath -> IO DGraph
directMaudeParsing fp = do
              ns <- parse fp
              let ns' = either (\ _ -> []) id ns
              (hIn, hOut, _, _) <- runInteractiveCommand maudeCmd
              hPutStrLn hIn $ "load " ++ fp
              ms <- traverseNames hIn hOut ns'
              hPutStrLn hIn "in Maude/hets.prj"
              psps <- predefinedSpecs hIn hOut
              sps <- traverseSpecs hIn hOut ms
              hClose hIn
              hClose hOut
              return $ insertSpecs (psps ++ sps) Map.empty Map.empty [] emptyDG
\end{verbatim}
}








