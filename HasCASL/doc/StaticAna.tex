\documentclass{article}
\usepackage[T1]{fontenc}

\parindent 0pt
\parskip 5pt

\begin{document}

\title{The HasCASL language and its analysis}

\author{C. Maeder}

\maketitle

\section{Preliminaries}

see the documentation for HetCATS and CASL

\section{HasCASL Parser}
The HasCASL parser reuses the token (in \texttt{HToken.hs}) and
\texttt{itemList} parser (in \texttt{ParseItem.hs}) from CASL as well as ids
and annotations from the top-level HetCATS directory.

The concrete syntax of HasCASL is given as plain text in
\texttt{Concrete-Syntax.txt}. The abstract syntax is given in
\texttt{As.hs}. The document \texttt{Abstract-Syntax.txt} is out-dated and was
intended as a human readable abstract syntax.

The data types in \texttt{As.hs} usually contain alternatives (Haskell
constructors) for different representation before and after the static
analysis, because an application usually requires a mixfix analysis that
depends on a global environment. Parsing alone only depends on its input
(and not on any global information).

\section{Static Analysis}

HasCASL entities are -- in hierarchical
order -- kinds, types and terms, where predicate types and formulae are
special cases of types and terms respectively.

Kinds are constructed from classes, types from type
constructors and terms from values, i.e. operations or predicates.

Classes, type constructors and values may be declared by giving them a name.
A type constructor must at least be declared with a kind and a value with a
type. (An implicit kind will always be the universe class \texttt{Type}.)

Usually declarations are extended with further properties or even a unique
definition.

In opposition to the hierarchy a class can be defined as a downset of a type
and a subtype can be defined via a predicate.

\subsection{Classes}

The major representation of a class is given by an intersection class over
class names only. The universe class \texttt{Type} is simply an empty
intersection. Nested intersection classes are flattened already by the
parser. A trivial intersection class is a single class name.

\subsubsection*{Class declarations}
\begin{itemize}
\item A class name can be declared as a subclass of an intersection class. If
  this intersection class is trivial then the superclass name is also regarded
  as a new or repeated declaration.
  
  A cyclic subclass relation is prohibited.
\item An undeclared class name may be defined as a synonym (or alias) for an
  intersection class. For the purpose of comparing classes, synonyms are
  expanded.
\item An undeclared class name may be defined as the downset of a type,
  comprising all subtypes of the given type. (The variable name for the
  subtype is ignored.)
  
  The downset of a type is a subclass of the downset of a
  supertype.
\item No superclasses can be declared for downsets or intersection classes

\end{itemize}

\subsubsection*{Remarks}

The class name of a downsets within an intersection class is not further
expanded, although one may have different classe names for the same downset.
Furthermore, type variables may be declared with an anonymous downset.

Class names are token ids, that is, simple ids with an optional compound list,
because types within downsets may also be compound ids.

Class declarations may be repeated, sub- and superclass declarations are
accumulated as long as they are non-cyclic.

Class definitions may be repeated, if they remain equal, otherwise it is an
error. Two downsets should be equal if the corresponding types are equal. (For
type equality type aliases must be expanded.) Currently class equality for
downsets is ignored.

All redeclarations issue a warning. 

The optional list of basic-items and the ``\texttt{instance}''-tag for a class
are not handled yet.

\subsection{Kinds}

There are two builtin kind constructors ($\to$ and $\times$), but these only
control the notation of type constructor applications, for mixfix type ids
only the overall arity of the kind matters. 

The arguments of a kind may again be kinds (to support constructor classes
later on) or classes with (optional) co/contra variance inducing further
subtype (not subclass) relations.

\subsection{Types}

Builtin type constructors are 4 different function arrows (partiality combined
with continuity), products (in $\times$ notation) and the lazy
prefix type constructor ``$?$''.

Type constructors names are declared with a kind. These names may be mixfix
ids, but in order to avoid a mixfix analysis for types, they should be
simple ids. Instead the parser resolves types as follows:

\begin{itemize}
\item function arrows have lowest infix priority and associate to the right
\item next come products with (generalized) infix $\times$ notation
\item next comes lazy type construction
\end{itemize}

Everything else to the right of a $?$ is regarded as mixfix type. Simple cases
of such types are the following:

\begin{itemize}
\item A single token is taken as nullary type constructor name
\item empty braces or brackets are taken as an id (with two tokens) as above
\item braces or brackets with one argument refer to an unary outfix type
  constructor $\{\_\_\}$ or $[\_\_]$.
\item braces or brackets with more arguments are considered illegal.
\item parenthesis may only contains a single argument. More arguments may only
  occur to the left of a type constructor token.
\item The empty tuple is illegal. Instead the builtin type 
  \texttt{Unit} should be used.
\end{itemize}

Mixfix type expressions with several components are simply regarded as
unparenthesized prefix applications provided that the first component is a
token (taken as type constructor id). 

Square brackets with several arguments could be taken as compound lists for a
preceding token. (Singleton compound lists are not supported, but regarded as
type construction, ie. for the list type.)

Type names cannot be overloaded! (The analysis of types and terms could
become more consistent if mixfix analysis and overload resolution were equally
treated for both types and terms. With a mixfix analysis -- that is
a rougher kind of overload resolution -- the parsing of types could
be simplified. 

Type names may be redeclared, but must then have the same kind structure. Only
the classes in argument or result positions may be different. This flavor of
overloading corresponds (on the level of terms) to the overloading groups of
operations (in CASL) that only differ in sorts that are in the
subsort relation.

\subsubsection*{General declarations}

Generally, there are two flavors of declarations for type constructors and
operations: value and pattern notation. In a value declaration a mixfix id is
separated from its kind or type by a colon whereas in a pattern declaration
arguments and mixfix components are intermingled and only the (optional)
result (class, kind or type) is written to the right of a colon. 

If a result is omitted a default is substituted. No kind defaults to the
\texttt{Universe} class and no result type for an operation indicates a
predicate declaration (although the builtin \texttt{Pred} type constructor is
recommended). The keyword \texttt{pred} is only kept for compatibility with
CASL and not recommended.

For simplicity only the value notation -- that does have no parenthesis on the
(id) left hand side (but possible places) -- is supported. 

The pattern notation may be seen as a generalization of the value notation,
because the result may be of higher order, but even then the mixing of places
and parenthesized arguments should be forbidden (places may at most follow the
last parenthesized argument).

The number of places must be equal or less to the number of arguments, where
arguments of products are also counted. However, if a place is given for a
first product component, then places must follow for all components of that
product. (Within applications further arguments are written as for
simple ids.)

\subsubsection*{Type declarations}

\begin{itemize}
\item There are simple value and pattern declarations. 
\item Type patterns (with the implicit kind \texttt{Universe})
  may be declared to be subtypes of a given type. Only a simple type name as
  supertype is regarded to be an additional declaration (for compatibility
  with CASL)
  
  This does (confusingly) not mean that the result class is an anonymous
  downset (as within variable declarations)!
\item Type patterns may be declared to be isomorphic, that is
  they are not equal and this corresponds to a trivial cyclic subtype
  relation.
\item A type pattern may defined to be a predicate subtype defined via a
  formula. For a repeated definition the equality of the formulae is
  undecidable and not thoroughly checked (but left to a proof tool).  
\item A type pattern may be a type alias. When comparing types these aliases
  are replaced with there definition (with proper argument substitution). 
\end{itemize}

\subsection{Variable declarations}

By variable declarations either type or value variables can be declared.
Type variables must be simple ids and there kind can either be an anonymous
downset (and using $<$ instead of a colon) or an intersection class (where
class names can also refer to a downset).

A wrongly spelled (or non-existing) class name causes the analysis to assume
that not a type but value variable is declared (most likely causing a type
error). Keeping class and type names disjoint is not furhter enforced.


\section{Test}

The subdirectory \texttt{test} contains \texttt{*.hascasl} files with basic
HasCASL specifications. The binary \texttt{../hacapa} called with the option
\texttt{analysis} and an input file produces an environment and diagnostic
messages on standard output. 

The test script can be called by \texttt{make check}. It compares results with
corresponding \texttt{*.output} files. The message ``passed'' indicates a
successful comparison. Initial \texttt{*.output} files are created when they
don't exist, but non-matching \texttt{*.output} files may also be overwritten
by \texttt{make output}.

\end{document}


%%% Local Variables: 
%%% mode: latex
%%% TeX-master: t
%%% End: 

