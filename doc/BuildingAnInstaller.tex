
\documentclass[a4paper]{article}
\usepackage{xspace}
%%%%%%%%%%%%%%%
\newcommand{\NeededHDSpaceMB}{500}

%%%%%%%%%%%%%%%%%%%%%%%%%%%%%%%%%%%%%%
\newcommand{\Hets}{\textmd{\textsc{Hets}}\xspace}
\newcommand{\HetsInstallDir}{hets-install}

\title{How to build an installer for \Hets}
\author{Klaus L\"uttich}
\date{last update: August 2006}

\begin{document}
\maketitle
\begin{abstract}
  This document describes the generation of automatic platform
  dependent and independent installers for \Hets with IzPack. IzPack
  allows the description of installers with XML and was extended by
  Heng Jiang supervised by Klaus L�ttich. This description is twofold:
  $(i)$ it gives a quick overview and the interface for the generator;
  and $(ii)$ it gives a detailed listing of steps performed during the
  genration of installers.
\end{abstract}

\section{Generation of Installers for \Hets with IzPack}

IzPack compiles installer descriptions in XML into \texttt{jar} files,
which can be executed with a Java Runtime Environment (JRE version 1.4
or newer). The XML files are in HetCATS/pack together with a Makefile
which eases the generation. The generation process needs
\NeededHDSpaceMB~MB of hard disk space including the generated
installers. An empty directory should be created for the generation of
installers which will be called \texttt{\HetsInstallDir} throughout this
document. The platform dependent installers for the distribution via
WWW end up in the directory hets-install and are called 
\texttt{install\_hets\_{\it arch}\_{\it os}.jar} where \textit{arch} is
the architecture (x86,sparc,ppc) and \textit{os} is the operating
system (linux,solaris,macosx). Note, not all combinations are available.

A typical genration process involves these steps:
\begin{enumerate}
\item create a directory: \texttt{mkdir \HetsInstallDir}
\item copy XML files from the HetCATS working directory to the new
  directory: \texttt{cp -r HetCATS/pack \HetsInstallDir}
\item 
\end{enumerate}
\clearpage

\section{Detailed List of Steps Performend for the Generation of Installers}

\begin{itemize}
\item use HetCATS/Makefile.installer for the generation of platform dependent
  Hets installers
\item a new directory (main) is created with platform dependent
  subdirectories and a platform independent directory tree
\item the installers will end up in the main directory with specific
  suffixes showing their platform
  \begin{itemize}
  \item hets-installer-ppc-mac.jar
  \item hets-installer-x86-linux.jar
  \item hets-installer-sparc-solaris.jar
  \end{itemize}
\end{itemize}

Makfile.installer:
\begin{itemize}
\item check out a fresh CASL-lib working copy;
  and make sure that only .casl and .het files are in the CASL-lib
  tree;
  we might include other files as well
\item check out a fresh HetCATS working copy and perform all steps to
  build a source distribution except for the generation of the
  tar-archive
\item a fresh programmatica source is needed as well
\item object files should not apper in the installer
\item hets binaries are obtained from the www-server as daily hets
  files
\item SPASS binaries are obtained from the SPASS-web-site
\item scripts from the utils directory
  \begin{itemize}
  \item getDailyHets.sh
  \end{itemize}
\item GMP.framework for Mac OS X from hets homepage
\end{itemize}
\clearpage
Directory-layout of installed directory: (dirs end with a slash)\\
Hets/
\begin{itemize}
\item bin/
  \begin{itemize}
  \item hets --- the shell script derived from HetCATS/hets.in with
    properly set paths
  \item isatool --- link to isabelle's isatool
  \item isabelle --- link to isabelle
  \item uDrawGraph --- link to uDrawGraph startup script
  \item SPASS --- a link to the SPASS binary
  \item --- a link to the OWL-DL parser script
  \item getDailyHets.sh --- properly configured to download to Hets/lib/hets
  \end{itemize}
\item lib/
  \begin{itemize}
  \item Isabelle/ --- if installed via IzPack; properly configured
  \item spass-XX/ --- binary and source code of SPASS
  \item hets/ 
    \begin{itemize}
    \item hets-XX --- the real binary of hets with version number as extension,
      e.g. \verb,hets-0.52,
    \item --- OEL-DL parser startup script
    \item --- jar-file(s) for OWL-DL parser
    \end{itemize}
  \item CASL-lib/ --- the CASL-lib libraries
  \item src/
    \begin{itemize}
    \item HetCATS/
    \item uni/
    \item programmatica/
    \end{itemize}
  \end{itemize}
\end{itemize}
\end{document}
