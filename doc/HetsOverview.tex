\documentclass{article}

\parindent 0pt
\parskip 5pt

\begin{document}

\section{Preliminaries}

Make sure a new version (currently 5.04.2) of \texttt{ghc}
(www.haskell.org) is properly installed.  Also
\texttt{/usr/local/bin/perl, /usr/local/bin/bash} and \texttt{GNU
  make} are required.


\subsection{Accessing the CVS repository}

Read \texttt{http://www.informatik.uni-bremen.de/\~{}ger/cvs/CVS.html}
and send the encrypted password and your (or another) username to
George Russell (\texttt{ger@tzi.de}). Under Solaris you can use
\texttt{/home/till/bin/encrypt} to get the password. After George's
reply you can log in. 

For \texttt{cvs login} the environment variable \texttt{CVSROOT} must be set:

\begin{verbatim}
CVSROOT=:pserver:$USER@cvs-agbkb.informatik.uni-bremen.de:/repository
export CVSROOT
\end{verbatim}

Go to or make a directory with sufficient space, i.e. under
\texttt{/home/cofi/\$USER}, and then retrieve the source tree:

\begin{verbatim}
cvs checkout HetCATS
\end{verbatim}

With \texttt{cvs up} you'll always retrieve the latest versions,
unless you modified a file. A modified file can be put into the
repository by \texttt{cvs commit MyFile}. Before committing something
you may check differences by \texttt{cvs diff MyFile}. Make sure the
committed file is usable for others and enter a meaningful log
message. A short log message can be given via the \texttt{-m} option:

\begin{verbatim}
cvs commit -m "corrected typing of 'eror' to 'error' " MyFile
\end{verbatim}

A completely new file must be added before committing by \texttt{cvs
  add MyFile}.  No longer used files can be removed by \texttt{cvs
  remove MyFile}.  Moving a file from one directory to another is
best done by adding and removing.

\texttt{cvs log MyFile} gives an overview of all revisions. An old
version, i.e. the initial revision 1.1, may be viewed using the
\texttt{-p} and \texttt{-r} options. (If you forget \texttt{-p}, then the
current revision will change.)

\begin{verbatim}
cvs up -p -r1.1 MyFile | less
\end{verbatim}

When you no longer work with the repository, you may release it:

\begin{verbatim}
cvs release HetCATS
\end{verbatim}

\subsection{Using hugs}

It is also possible to use \texttt{hugs} (version of Nov. 2002) instead of
\texttt{ghc}.  The environment variable \texttt{HUGSFLAGS} (one line) can be
given (in your \texttt{\~{}/.bashrc} file) as follows:

\begin{verbatim}
export HUGSFLAGS="-98 +o -h20M\
  -P.:$HCAT:$HCAT/hugs:$HCAT/hetcats:{Hugs}/libraries:\
  -E\"emacsclient +%d %s\""
\end{verbatim}

The variable \texttt{HCAT} must be set to your own \texttt{HetCATS} directory.

\subsection{Using ghci}

When using \texttt{ghci} it is a good idea to create a file
\texttt{\~{}/.ghci} with the following contents:

\begin{verbatim}
:set -fglasgow-exts -Wall 
:set -i..:../ghc:../hetcats
\end{verbatim}

Instead of \texttt{..} the actual \texttt{HetCATS} directory should be
included (if you don't start ghci in a subdirectory).

\section{Getting started}

\subsection{Check out}
The latest versions of all sources should be checked out in the top
directory \texttt{HetCATS/} using:

\texttt{cvs up -dP}

Use option \texttt{-d} to also get new directories and
use option \texttt{-P} to prune (get rid of) empty directories.

\subsection{(Pre-)Compile}

\texttt{cd HetCATS} and call \texttt{make} (on Linux) or
\texttt{gmake} (on Solaris). To fully recompile everything call
\texttt{gmake distclean} before. Files with extensions
\texttt{*.\{d,o,hi\}} are intermediate compilation results.

One important effect is that \texttt{*.der.hs} files will be
``drifted'', i.e.  from \texttt{*.der.hs} files actual haskell source
files (\texttt{*.hs}) are created that include extra code for
instances (of class \texttt{PosItem}) derived by DrIFT (in
\texttt{HetCATS/utils/}).

\subsection{Directory structure of \texttt{HetCATS}}

The subdirectories \texttt{utils} (for DrIFT), \texttt{hetcats} (for
the final application), \texttt{pretty}, \texttt{parsec}, \texttt{fgl}
have been mentioned above.

\begin{description}
\item[CASL] Christian 
\item[Common] all
\item[CspCASL] Markus
\item[GUI] 
\item[HasCASL] Christian
\item[Haskell] Christian
\item[Logic] 
\item[Lottery] Markus
\item[Modal] Markus
\item[Static] 
\item[Syntax] 
\item[doc] for documentations like this one
\item[docs] created by haddock (or \texttt{make doc})
\item[ghc] include this directory if you use ghc
\item[haterm-1.0/src] the adapted ATerm library (Klaus)
\item[hetcats] converts sml-CATS ATerms (Klaus)
\item[hugs] include this directory if you use hugs
\item[mini] Till
\item[test] for test cases (Pascal)
\item[utils] DrIFT, haddock, ag (attribute grammar system)
\end{description}

The directories \texttt{hugs} and \texttt{ghc} currently only implement the function \texttt{unsafeCoerce} differently. 

The top-level directory only contains \texttt{hets.hs}. Further documentation
should appear in \texttt{doc} of subdirectories.

\end{document}

