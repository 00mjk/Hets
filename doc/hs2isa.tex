\documentclass{llncs}
%\documentclass[a4paper,11pt]{article}
%\textwidth  15.93cm
%\textheight 24cm
%\oddsidemargin  0.0cm
%\evensidemargin 0.0cm
%\topmargin  0.0cm

\begin{document}

\title{Translating Haskell to Isabelle}
\author{ Paolo Torrini\inst{1} \and
  Christoph Lueth\inst{2}
  Christian Maeder\inst{2}
  Till Mossakowski\inst{2}
}
\institute{Verimag Grenoble, {\tt Paolo.Torrini@imag.fr} 
\and DFKI Lab Bremen, {\tt \{Christoph.Lueth,Christian.Maeder,Till.Mossakowski\}@dfki.de} }
% \date{\today}

\maketitle

\begin{abstract}
  We present partial translations of Haskell programs to Isabelle that have
  been implemented as part of the Heterogenous Tool Set. The the target logic
  is Isabelle/HOLCF, and the translation is based on a shallow embedding
  approach.
  % The AWE package has been used to support a translation of monadic
  %operators based on theory morphisms.
\end{abstract}

\sloppy

\section{Introduction}
\label{intro}

Automating the translation from programming languages to the languagee of a
generic prover may provide useful support for the formal development and the
verification of programs. It has been argued that functional languages can
make the task of proving assertions about programs written in them easier,
owing to the relative simplicity of their semantics
\cite{Thompson92,Thompson95}. The idea of translating Haskell programs, came
to us, more specifically, from an interest in the use of functional languages
for the specification of reactive systems. Haskell is a strongly typed, purely
functional language with lazy evaluation, polymorphic types extended with type
constructor classes, and a syntax for side effects and pseudo-imperative code
based on monadic operators \cite{HaskellRep}.  Several languages based on
Haskell have been proposed for application to robotics \cite{phh99,Hudak2003}.
In such languages, monadic constructors are extensively used to deal with
side-effects. Isabelle is a generic theorem-prover implemented in SML
supporting several logics --- in particular,  
Isabelle/HOL is the implementation in Isabelle of classical higher-order logic
based on simply typed lambda calculus extended with axiomatic type classes. It
provides support for reasoning about programming functions, both in terms of
rich libraries and efficient automation. 
Isabelle/HOLCF \cite{holcf} \cite{Paulson94isa,holcf} is Isabelle/HOL
conservatively extended with the Logic of Computable Functions --- a
formalisation of domain theory.

We have implemented as functions of Hets translations of Haskell to
Isabelle/HOLCF following an approach based on shallow embedding,
mapping Haskell types to Isabelle ones, therefore taking full
advantage of Isabelle built-in type-checking. Hets
\cite{MossaTh,HetsUG,Hets} is an Haskell-based application designed to
support heterogeneous specification and the formal development of
programs. It has an interface with Isabelle, and relies on
Programatica \cite{Prog04} for parsing and static analysis of Haskell
programs.  Programatica already includes a translation to
Isabelle/HOLCF which, in contrast to ours, is based on an object-level
modelling of the type system \cite{Huff}.

The paper is organised as follows: Section~\ref{sec:semantics} discusses
the semantic background of the translation, while the subsequent sections
are devoted to the translation of types, datatypes, classes and function
definitions, respectively. Sect.~\ref{sec:ex} shows some example proofs,
and Sect.~\ref{sec:conclusion} concludes the paper.

%Section 2 gives some background, section 3 introduces the system, section 4
%gives the sublanguages of Haskell that can be translated, in section 5 we
%define the two translations.
%, in section 6 we sketch a denotational semantics
%associated with the translation to Isabelle/HOLCF, in section 7 we show how translation
%of monads is carried out with AWE.

\section{Semantic Background of the Translation}
\label{sec:semantics}

We firstly describe the subset of Haskell that we cover.
Our translation to Isabelle/HOLCF covers at present Booleans, integers, basic
constructors (function, product, list, \emph{maybe}), equality,
single-parameter type classes (with some limitations), \emph{case} and
\emph{if} expressions, \emph{let} expressions without patterns, mutually
recursive data-types and functions. 
%It relies on existing formalisations of
%lazy lists and \emph{maybe}. 
It keeps into account partiality and laziness by
following, for the main lines, the denotational semantics of lazy evaluation
given in \cite{winskel}. There are several limitations: \emph{Prelude} syntax
is covered only partially; list comprension, \emph{where} expressions and
\emph{let} with patterns are not covered; further built-in types and type
classes are not covered; imports are not allowed; constructor type classes are
not covered at all --- and so for monadic types beyond list and \emph{maybe}.
Of all these limitations, the only logically deep ones are those related to
classes --- all the other ones are just a matter of implementation.

For the translation, we have followed the informal description of the
operational semantics given in the Haskell report \cite{HaskellRep},
and also consulted the complete static semantics for Haskell 98
\cite{journals/jfp/Faxen02}, as well as the (fragmentary) dynamic
semantics \cite{oai:CiteSeerPSU:71374}.  However, it should be noted
that there is no denotational semantics of Haskell! This also has been
confirmed in the answers to a query that one of the authors has sent
to the Haskell mailing list.  There is a static semantics for Haskell
Hence, our translation to Isabelle/HOLCF can be seen as the first 
denotational semantics given to a large subset of Haskell 98.
This also means that there is no notion of correctness of this
translation, because it just \emph{defines} the denotational semantics.
Of course, an interesting question is the coincidence of 
denotational and operational semantics. However, this is beyond the
scope of the paper.

%The base translation to Isabelle-HOL is more limited, insofar as it covers
%only total primitive recursive functions. A better semantics with respect to
%partiality could be obtained by lifting the type of function values with
%\emph{option}, but this has not been pursued here. Still, \emph{option} has
%been used to translate \emph{maybe}. On the other hand, laziness appears very
%hard to be captured with Isabelle-HOL. It also seems hard to overcome the
%limitation to primitive recursion. Other limitations are similar to those
%mentioned for the translation to Isabelle/HOLCF --- with the notable exception of
%monads.



%\section{Translations in Hets}

%The Haskell-to-Isabelle translation in Hets requires GHC, Programatica,
%Isabelle and AWE. The application is run by a command that takes as arguments
%a target logic and an Haskell program, given as a GHC source file. The latter
%gets analysed and translated, the result of a successful run being an Isabelle
%theory file in the target logic.

%The Hets internal representation of Haskell is similar to that of
%Programatica, whereas the internal representation of Isabelle is based on the
%ML definition of the Isabelle base logic, extended in order to allow for a
%simpler representation of Isabelle-HOL and Isabelle/HOLCF.  Haskell programs and
%Isabelle theories are internally represented as Hets theories --- each of them
%formed by a signature and a set of sentences, according to the theoretical
%framework described in \cite{MossaTh}.  Each translation, defined as
%composition of a signature translation with a translation of all sentences, is
%essentially a morphism from theories in the internal representation of the
%source language to theories in the representation of the target language.



\section{Translation of Types}
\label{sec:types}


In Isabelle/HOL types are interpreted as sets (class \emph{type});
functions are total and may not be computable. A non-primitive
recursive function may require discharging proof obligations already
at the stage of definition --- in fact, a specific relation has to be
given for a function to be proved total. In Isabelle/HOLCF each type
is interpreted as a pointed complete partially ordered set (class
\emph{pcpo}) i.e. a set with a partial order which has suprema of
$\omega$-chains and has a bottom. Isabelle's formalisation, based on
axiomatic type classes \cite{Wenzel}, makes it possible to deal with
complete partial orders in quite an abstract way.  Functions are
generally partial and computability can be expressed in terms of
continuity. Recursion can be expressed in terms of least fixed-point
operator, and so, in contrast with Isabelle/HOL, function definition
does not depend on proofs.  Nevertheless, proving theorems in
Isabelle/HOLCF may turn out to be comparatively hard.  After being
spared the need to discharge proof obligations at the definition
stage, one has to bear with assumptions on function continuity
throughout the proofs. A standard strategy is then to define as much
as possible in Isabelle/HOL, using Isabelle/HOLCF type constructors to
lift types only when this is necessary.

The provision of pcpos, domains and continuous functions by
Isabelle/HOLCF eases the translation of Haskell types and functions a
lot.  However, special care is needed when trying to understand the
Haskell semantics. If one reads the section of the Haskell report
dealing with types and classes, one could come to the conclusion that
a function space in Haskell can be mapped to the space of the
continuous functions in Isabelle/HOLCF --- this would correspond to a
purely lazy language. However, Haskell is a mixed eager and lazy
language, and it provides a function $seq$ that enforces eager
evaluation. This function is introduced in part 6 of the Haskell
report, ``Predefined Types and Classes'', in section 6.2.  We quote
from there:
\begin{quote}
  However, the provision of $seq$ has important semantic consequences,
  because it is available at every type. As a consequence, $\bot$ is
  not the same as $\lambda x \rightarrow \bot$, since $seq$ can be
  used to distinguish them.
\end{quote}
In order to enforce this distinction, each function space needs to be
lifted. The same holds for products.  We define these liftings in a
specific Isabelle theory \emph{HsHOLCF} (included in the Hets
distribution) as follows:

\begin{verbatim} 
defaultsort pcpo

domain ('a,'b) lprod = lpair (lazy lfst :: 'a) (lazy lsnd :: 'b)

domain 'a Lift = Lift (lazy 'a)

types  ('a, 'b) "-->"    = "('a -> 'b) Lift" (infixr 0)

constdefs
  liftedApp :: "('a --> 'b) => ('a => 'b)" ("_$$_" [999,1000] 999)
                                                (* application *)
  "liftedApp f x == case f of
                     Lift $ g => g $ x"
constdefs
  liftedLam :: "('a => 'b) => ('a --> 'b)" (binder "Lam " 10)
                                                (* abstraction *)
  "liftedLam f == Lift $ (LAM x . f x)"
\end{verbatim}

Our translation of Haskell types to Isabelle types is defined
recursively.  It is based on a translation of names for avoidance of
name clashes that is not specified here. We write $\alpha'$ for both
the recursive translation of item $\alpha$ and the renaming according
to the name translation.  The translation of types is given by the
following rules:


\noindent Types
$$\begin{array}{ll}
  a & \Longrightarrow \ 'a::\{pcpo\} \\
  () & \Longrightarrow \ unit \ \mbox{lift} \\
  Bool & \Longrightarrow \ bool \ \mbox{lift} \\
  Integer & \Longrightarrow \ int \ \mbox{lift} \\
  \tau_{1} \to \tau_{2} & \Longrightarrow
  \ \tau'_{1} \ \verb@-->@ \ \tau'_{2})  \\
  (\tau_{1},\tau_{2}) & \Longrightarrow
  \ (\tau'_{1} ~lprod~ \tau'_{2}) \\
  {[\tau]} \ & \Longrightarrow \ \tau' \ llist \\
  T \ \tau_1 \ \ldots \ \tau_n & \Longrightarrow
  \ \tau'_1 \ \ldots \ \tau'_n \ T' \\
  & \quad \mbox{with} \ T \ \mbox{either \ datatype \ or \ defined \ type} \\
\end{array}$$

Built-in types are translated to the lifting of the corresponding HOL
type.  The Isabelle/HOLCF type constructor \emph{lift} is used to lift
types to flat domains. The type constructor $llist$ is dicussed in the
next section.


\section{Translation of Datatypes}

As explained in the Haskell report \cite{HaskellRep}, section 4.2.3,
the following four Haskell declarations

\begin{verbatim}
 data D1 = D1 Int
 data D2 = D2 !Int
 type S = Int
 newtype N = N Int
\end{verbatim}

have four different semantics. Indeed, the correct translation to
Isabelle/HOLCF is as follows:

\begin{verbatim}
domain D1 = D1 (lazy D1_1::"Int lift")
domain D2 = D2 (D2_1::"Int lift")
types S = "Int lift"
pcpodef N = "{x:: Int lift . True}"
by auto
\end{verbatim}
\noindent In Isabelle/HOLCF, the keyword \emph{domain} defines a
(possibly recursive) domain as solution of the corresponding domain
equation.  The keyword \emph{lazy} ensures that the constructor $D1$
is non-strict, i.e.\ $D1 \ \bot \ \neq \ \bot$.  The keyword
\emph{pcpodef} can be used to define sub-pcpos of existing pcpos;
here, we use it just to introduce an isomorphic copy of an existing
pcpo --- this is the semantic of Haskell \emph{newtype} definitions.

%Note that the constructor for N is Abs_N, the selector Rep_N. 

Lists are translated to the domain \emph{llist}, defined as follows 
in our prelude theory \emph{HsHOLCF}:

\begin{verbatim}
domain 'a llist = nil | ## (lazy HD :: 'a) (lazy TL :: 'a llist) 
\end{verbatim}
\noindent allowing for partial list as well as for infinite ones \cite{holcf}.\\

For each datatype, we have to lift the constructors from the
HOLCF continuous function spaces to our lifted function spaces:

\begin{verbatim}
constdefs
  cont2lifted2 :: "('a -> 'b -> 'c) => ('a --> 'b --> 'c)"
  "cont2lifted2 f == Lam x . Lam y. f $ x $ y"

  llCons :: "'a --> 'a llist --> 'a llist"
  "llCons == cont2lifted2 (op ##)"
\end{verbatim}



The general scheme for translation of mutually recursive lazy Haskell
datatypes to Isabelle/HOLCF domains is as follows:
\begin{tabbing}
$ data \ \phi_1 \ = \ C_{11} \ x_1 \ldots x_i
  \ | \ \ldots \ | \ C_{1p} \ y_1 \ldots y_j $ \\
$ \ldots $ \\
$ data \ \phi_n \ = \ C_{n1} \ w_1 \ldots w_h
  \ | \ \ldots \ | \ C_{1q} \ z_1 \ldots z_k \Longrightarrow $ \\
$  \qquad $\=$ domain$\=$ \ \phi'_1 \ = \  $\=$ C'_{11} \ (lazy~ ::\ x'_1) \ \ldots \ (lazy~ :: \ 
  x'_i) \ |$\\
\>\>\>$ \ \ldots \ | \ $\\
\>\>\>$ C'_{1p} \ 
  (lazy~ :: \ y'_1) \ \ldots \ (lazy~ :: \ y'_j) $\\
\>$  and \ \ldots $\\
\>$  and \ $\>$\phi'_n \ = \ C'_{n1} \ (lazy~ :: \ w'_1) \ \ldots \ (lazy~ :: \ 
  w'_h) \ | \ $\\
\>\>\>$\ldots \ | $\\
\>\>\>$\  C'_{nq} \ 
  (lazy~ :: \ z'_1) \ \ldots \ (lazy~ :: \ z'_k) $\\
\end{tabbing}
Mutually recursive datatypes relies on specific Isabelle syntax
(keyword \emph{and}).\footnote {Due to a bug in Isabelle/HOLCF 2005,
  declaration of mutually recursive lazy domains does not work.
  However, this will be fixed in Isabelle 2007; the fix currently is
  available in the development shnapshot.} Order of declarations is
taken care of. In case of strict arguments (indicated with ! in
Haskell), the keyword \emph{lazy} is omitted.

The translation scheme for type synonyms simply is
$$  type \ \tau \ = \ \tau_1 \qquad \qquad \Longrightarrow
  \ types \ \tau' \ = \ \tau'_1 
$$


\section{Translation of Kinds and Type Classes}

Type classes in Isabelle and Haskell associate a set of functions to a
type class identifier (these are also called the \emph{methods} of the
class). In Isabelle, type classes are typically further specified
using a set of axioms; for example, the class
$\mathit{linorder}$ of total orders is specified using the
usual total order axioms. Of course, such axiomatizations are not
possible in Haskell. Indeed, in Haskell, there is no check that the
class $Ord$ actually consists of total orders only, and hence it would
be inappropriate to translate it to $\mathit{linorder}$ in Isabelle.
Instead, we translate to a newly declared Isabelle class $Ord$.
The only thing that we assume is that it is a subclass of $pcpo$,
because all Haskell types are translated to pcpos.
Hence, tpye classes are translated to Isabelle/HOLCF as subclasses of
\emph{pcpo} with empty axiomatization.
Methods declarations associated with
Haskell classes are translated to independent function declarations with
appropriate class annotation on type variables. 



Class instance declarations declare that a particular type belongs to
a class. In Isabelle, instance declarations generate proof
obligations, namely that the methods for the type at hand indeed
satisfy the axioms of the class. Since our translations only
generates classes without axioms (beyond those of $pcpo$), proofs are trivial and proof obligation
may be automatically discharged.


A Haskell class instance declaration that declares type $T$
to belong to class $C$ may define the behaviour of the
class methods for $T$. The same is possible with
normal Isabelle constant definitions, if the type of the
constant (function) is specialised to $T$ in the definition.
With this, we avoid an explicit handling of dictionaries, as
described in the static semantics of Haskell \cite{journals/jfp/Faxen02}.


%Not all the problems have been solved with
%respect to arities that may conflict in Isabelle, although they correspond to
%compatible Haskell instantiations.  

A restriction of Isabelle is that it does neither allow for
multi-parameter classes, nor for type constructor ones. Therefore, the
same restrictions apply to our translation.



\noindent Classes
$$\begin{array}{ll}
  K & \Longrightarrow \ K' 
\end{array}$$

\noindent Type schemas
$$\begin{array}{ll} 
  (\{K \ v\} \ \cup \ ctx) \ \Rightarrow \ \tau & \Longrightarrow 
  \ (ctx \Rightarrow \tau)'  
           \ [(v'::s) / (v'::({K'} \cup s))]  \\
  \{\} \ \Rightarrow \ \tau & \Longrightarrow \ \tau'
\end{array}$$\\

Haskell type variables are translated to variables of class
\emph{pcpo}.  Each type is associated to a sort in Isabelle (in
Haskell, the same concept is called ``kind''), defined by the set of
the classes of which it is member.  
\medskip

\noindent Class declarations
$$\begin{array}{l} class \ K \ where \ (Dec_1; \ldots; Dec_n \rceil) \ 
  \Longrightarrow \ class \ K' \ \subseteq
  \ pcpo; \ Dec'_1; \ \ldots; \ Dec'_n \\
\end{array}$$
        
\noindent Class instance definitions
$$\begin{array}{l}
instance \ ctx \ \Rightarrow \ K_T \ (T \ v_1 \ \ldots \ v_n) \ where \\ 
  \qquad \qquad \qquad (f_1::\tau_1 = t_1; \ldots; \ f_n::\tau_n = t_n)
  \ \Longrightarrow \\
  \qquad instance \\
  \qquad \qquad \tau' :: \ K_{T}' \ (\{pcpo\} \ \cup \ \{K':(K \ v_1)
  \in ctx\}, \ \ldots, \\
  \qquad \qquad \qquad \quad \quad 
         \{pcpo\} \ \cup \ \{K':(K \ v_n) \in ctx\}) \\
  \qquad \qquad \mbox{with \ proof \ obligation}; \\
  \qquad defs \ f'_{\tau 1} :: (ctx \Rightarrow \tau_1)' = t'_1; \ \ldots; \
  f'_{\tau n} :: (ctx \Rightarrow \tau_n)' = t'_n
\end{array}$$\\





\section{Translation of Function Definitions and Terms}

Terms of built-in type are translated using Isabelle/HOLCF-defined lifting function
\emph{Def}.  The bottom element $\bot$ is used for undefined terms.
Isabelle/HOLCF-defined $flift1 :: \ ('a \Rightarrow 'b::pcpo) \Rightarrow ('a \ lift
\to 'b)$ and $flift2 :: \ ('a \Rightarrow 'b) \Rightarrow ('a \ lift \to 'b \ 
lift)$ are used to lift operators, as well as the following, defined in
\emph{HsHOLCF}.\\

\begin{verbatim}
fliftbin :: ('a => 'b => \ 'c) => ('a lift -> 'b lift -> 'c lift) 
fliftbin f == flift1 (%x. flift2 (f x))
\end{verbatim}
\noindent Boolean values are translated to values of \emph{bool lift} 
(\emph{tr} in Isabelle/HOLCF) i.e.  \emph{TT}, \emph{FF} and $\bot$, and
Boolean connectives to the corresponding Isabelle/HOLCF operators.
Isabelle/HOLCF-defined \emph{If then else fi} and \emph{case} syntax are used
to translate conditional and case expressions, respectively.  There are
restrictions, however, on case epressions, due to limitations in the
translation of patterns; in particular, only simple patterns are allowed (no
nested ones). On the other hand, Isabelle sensitiveness to the order of
patterns in case expressions is dealt with. Multiple function definitions are
translated as definitions based on case expressions. In function definitions
as well as in case expressions, both wildcards --- not available in Isabelle
--- and incomplete patterns --- not allowed --- are dealt with by elimination,
$\bot$ being used as default value in the latters. Only let expressions
without patterns on the left are dealt with; guarded expressions are
translated as conditional ones; where expressions and list comprehension are
not covered.



$$\begin{array}{l}
  f :: \phi \qquad \qquad \qquad \quad \Longrightarrow
  \ consts \ f' \ :: \ \phi' \\
 f \ \overline{x} \ p_1 \ \overline{x}_1 \ = \ t_1; \ 
  \ldots;
  \ f \ \overline{x} \ p_n \ \overline{x}_n \ = \ t_n \ \Longrightarrow \\
  \qquad (f \ \overline{x} \ = \ case \ y \ of \ (p_1 \to (\backslash
  \overline{x}_1 \to \ t_1); \ 
  \ldots; \ p_n (\to \backslash \overline{x}_n \to \ t_n)))' \\
  f \ \overline{x} \ = \ t \qquad \qquad \Longrightarrow \ defs \ f' :: \phi'
  \ 
  = \ Lam \ \overline{x}'. \ t' \\
  \qquad \mbox{with} \ f::\phi \ \mbox{not \ occurring \ in} \ t \\
  (f_1 \ \overline{v_1} \ = \ t_1; \ \ldots;
  \ f_n \ \overline{v_n} \ = \ t_n) \ \Longrightarrow \\
  \qquad fixrec \ f'_1 :: \phi'_1 \ =  
    \ (Lam \ \overline{v_1}'. \ t'_1) \ and \\
  \qquad \quad \ldots \\
  \qquad \quad and \ f'_n :: \phi'_n \ =
  \ (Lam \ \overline{v_n}'. \ t'_n) \\
  \qquad \mbox{with} \ f_1::\phi_1, \ldots, f_n::\phi_n \ 
  \mbox{mutually \ recursive} \\
\end{array}$$\\


Function declarations use Isabelle keyword \emph{consts}. 
%
Non-recursive definitions are translated to standard definitions using
Isabelle keyword \emph{defs}. Recursive definitions rely on
Isabelle/HOLCF package \emph{fixrec} which provides nice syntax for
fixed point definitions, including mutual recursion. Lambda
abstraction is translated as continuous abstraction for lifted
function spaces (\emph{Lam}), function application as continuous
application (the \emph{\$\$} operator), see Sect.~\ref{sec:types}
above.

\section{Example Proofs}
\label{sec:ex}

The Haskell declarations
\begin{verbatim}
comp ::  (b -> c) -> (a -> b) -> a -> c
comp f g x = f (g x)

map1 :: (a -> b) -> [a] -> [b]
map1 f [] = []
map1 f (x:xs) = f x : map1 f xs
\end{verbatim}

\noindent 
are translated to

\begin{verbatim}
consts
X_comp :: "('b --> 'c) --> ('a --> 'b) --> 'a --> 'c"
map1 :: "('a --> 'b) --> 'a llist --> 'b llist"

defs
X_comp_def :
"(X_comp :: ('b --> 'c) --> ('a --> 'b) --> 'a --> 'c) ==
 Lam f. Lam g. Lam x. f $$ (g $$ x)"

fixrec "(map1 :: ('a --> 'b) --> 'a llist --> 'b llist) =
         (Lam qX1.  (Lam qX2. case qX2 of
             lNil => lNil |
             op ### $ pX1 $ pX2 =>
                 llCons $$ (qX1 $$ pX1) $$ (map1 $$ qX1 $$ pX2)))"
\end{verbatim}


\noindent 
The fixrec definition leads to a bunch of theorems, one for simplification.
However, the latter makes the Isabelle simplifier loop, and hence needs
to be replaced with a more suitable simplification theorem:
\begin{verbatim}
lemmas map1_simps [simp del]
lemma map1_simp : "map1 $$ (f::('a --> 'b)) $$ (x::'a llist) =
    (case x of lNil => lNil | 
               op ### $ pX1 $ pX2 => 
                    llCons $$ (f$$pX1)$$ (map1$$f$$pX2))"
  apply (subst map1_simps)
  by auto
\end{verbatim}

\noindent 
Due to the recursion, also this theorem cannot be fed into the simplifier.
However, using substitution, it helps in proving the following
expected lemmas about the behaviour of $map1$:

\begin{verbatim}
lemma map1_UU [simp] : "map1 $$ f $$ UU = UU"
  apply (subst map1_simp) 
  by simp

lemma map1_lNil[simp] : " map1 $$ f $$ lNil = lNil"
  apply (subst map1_simp) 
  by simp

lemma map1_cons1[simp] : "map1 $$ f $$ (op ### $ x $ (xs::'a llist)) =
                          op ### $ (f $$ x) $ (map1 $$ f $$ xs)"
  apply (subst map1_simp)
  apply (simp add: cont2lifted2_def)
done
\end{verbatim}

\noindent With these, it is now easy to prove, via induction, that map
distributes over composition:
\begin{verbatim}
theorem compMap : 
   "X_comp $$ (map1 $$ f) $$ (map1 $$ g) $$ x = 
    map1 $$ (X_comp $$ f $$ g) $$ x"
  apply (rule llist.ind)
  back
  apply (auto simp add: X_comp_def)
done
\end{verbatim}

\section{Conclusion and Related Work}
\label{sec:conclusion}

We provide a shallow embedding of Haskell to Isabelle/HOLCF,
which can be used for proving properties of Haskell programs,
but also is the first denotational semantics that has been
given to Haskell.

The main advantage of our shallow approach is to get as much as
possible out of the automation currently available in Isabelle,
especially with respect to type checking. Isabelle/HOLCF in particular
provides with an expressive semantics covering lazy evaluation, as
well as with a smart syntax --- also thanks to the \emph{fixrec}
package. It is interesting to note that Haskell functions and product
types have to be lifted due to the mixture of eager and lazy evaluation
that Haskell exhibits due to the presence of the $seq$ function.
Type classes in Haskell are similar enough to Isabelle's type classes
such that explicit handling of dictionaries can be avoided.

The main disadvantage of our approach is the lack of type
constructor classes. Anyway, it is possible to get around the
obstacle, at least partially, by relying on an axiomatic
characterisation of monads and on a proof-reuse strategy that actually
minimises the need for interactive proofs.

Concerning related work, although there have been translations of
functional languages to first-order systems --- those to FOL of
Miranda \cite{Thompson95,Thompson89,Thompson95b} and Haskell
\cite{Thompson92}, both based on large-step operational semantics;
that of Haskell to Agda implementation of Martin-Loef type theory in
\cite{Abel} --- still, higher-order logic may be quite helpful in
order to deal with features such as currying and polymorphism.
Moreover, higher-order approaches may rely on denotational semantics
--- as for examples, \cite{Huff} translating Haskell to HOLCF, and
\cite{Pollack} translating ML to HOL --- allowing for program
representation closer to specification as well as for proofs
comparatively more abstract and general.

The translation of Haskell to Isabelle/HOLCF proposed in \cite{Huff}
uses deep embedding to deal with types.  Haskell types are translated
to terms, relying on a domain-theoretic modelling of the type system
at the object level, allowing explicitly for a clear semantics, and
providing for an implementation that can capture most features,
including type constructor classes.  In contrast, we provide in the
case of Isabelle/HOLCF with a translation that follows the lines of a
denotational semantics under the assumption that type constructors and
type application in Haskell can be mapped to corresponding
constructors and built-in application in Isabelle without loss from
the point of view of behavioural equivalence between programs --- in
particular, translating Haskell datatypes to Isabelle ones.  Our
solution gives in general less expressiveness than the deeper approach
--- however, when we can get it to deal with cases of interest, it
might make proofs easier.

%On the other hand, the main novelty in our work is
%to rely on theory morphisms and on their implementation for Isabelle in the
%package AWE \cite{AWE}, in order to deal with special cases of monadic
%operator.
% Currently Hets provides with an extension of the
%base translation to Isabelle/HOL which uses AWE and covers state monad types
%inclusive of \emph{do} notation.  Due to present limitations of AWE, this
%solution is available only for Isabelle/HOL at the moment, although in
%principle it could work for Isabelle/HOLCF as well.


Future work should use this framework for proving properties of
Haskell programs.  Currently Hets already provides some experimental
translation of constructor classes and monads, also covering \emph{do}
notation, using theory morphisms as provided by the package AWE
\cite{AWE}. However, there are mainly syntactic problems with name
clashes that currently prevent a proper integration with
Isabelle/HOLCF. These problems should be solved in the near future.

For monadic programs, we are also planning to use the monad-based
dynamic Hoare and dynamic logic that already have been formalised in
Isabelle \cite{Walter05}. Our translation tool from Haskell to
Isabelle is part of the Heterogeneous Tool Set Hets and can be
downloaded from \texttt{http://www.dfki.de/sks/hets}. More details
about the translations can be found in \cite{Tlmm}.


%This translation, due to the character of the P-logic
%typing system, could be more easily carried out relying on some form
%of universal quantification on type variables, or else further relying
%on parametrisation.

\subsection*{Acknowledgment}
This work has been supported by the {\em Deutsche
  Forschungsgemeinschaft} under grants KR \mbox{1191/5-2} and \mbox{KR
  1191/7-2 }. We thank Brian Huffmann for help with the Isabelle/HOLCF
package and Erwin R. Catesbeiana for pointing out an inconsistency.

\bibliographystyle{alpha} \bibliography{case}

\end{document}

\subsection{HOL}

The translation $\omega_s \ :: \ H_s \ \to \ HOL$ from programs in $H_s$
to theories in Isabelle/HOL extended with AWE can be defined with the
following set of rules.\\

\noindent Types
$$\begin{array}{ll}
  () & \Longrightarrow \ unit \\
  a & \Longrightarrow \ 'a::\{type\} \\
  Bool & \Longrightarrow \ boolean \\
  Integer & \Longrightarrow \ int \\
  \tau_{1} \to \tau_{2} & \Longrightarrow
  \ \tau'_{1} \ \Rightarrow \ \tau'_{2} \\
  (\tau_{1},\tau_{2}) & \Longrightarrow
  \ (\tau'_{1} * \tau'_{2}) \\
  {[\tau]} \ & \Longrightarrow \ \tau' \ list \\
  Maybe \ \tau & \Longrightarrow \ \tau' \ option \\
  T \ \tau_1 \ \ldots \ \tau_n & \Longrightarrow
  \ \tau'_1 \ \ldots \ \tau'_n \ T' \\
  & \quad \mbox{with} \ T \ \mbox{either \ datatype \ or \ defined \ type} \\
\end{array}$$

\noindent Classes
$$\begin{array}{ll}
  Eq & \Longrightarrow \ Eq \\
  K & \Longrightarrow \ K' 
\end{array}$$

\noindent Type schemas
$$\begin{array}{ll} 
  (\{K \ v\} \ \cup \ ctx) \ \Rightarrow \ \tau & \Longrightarrow 
  \ (ctx \Rightarrow \tau)'  
           \ [(v'::s) / (v'::({K'} \cup s))]   \\
  \{\} \ \Rightarrow \ \tau & \Longrightarrow \ \tau'
\end{array}$$\\

Here we highlight the main differences the translation to Isabelle/HOLCF and this,
semantically rather more approximative one to Isabelle/HOL (thereafter simply
HOL).  Function type, product and list are used to translate the corresponding
Haskell constructors. Option types are used to translate \emph{Maybe}. Haskell
datatypes are translated to HOL datatypes. Type variables are of class
\emph{type}.

Standard lambda-abstraction ($\lambda $) and function application are used
here, instead of continuous ones. Non-recursive definitions are treated in an
analogous way as in the translation to Isabelle/HOLCF. However, partial functions and
particularly case expressions with incomplete patterns are not allowed.

In HOL one has to pay attention to the distinction between \emph{primitive
  recursive} functions (introduced by the keyword \emph{primrec}) and
generally recursive ones.  Termination is guaranteed for each primitive
recursive function by the fact that recursion is based on the datatype
structure of one of the parameters. In contrast, termination is no trivial
matter for recursion in general. A strictly decreasing measure needs to be
association with the parameters. This cannot be dealt with automatically in
general. Threferore here we restrict translation to primitive recursive
functions.

Mutual primitive recursion is allowed for under additional restrictions ---
more precisely, given a set $F$ of functions: 1) all the functions in $F$ are
recursive in the first argument; 2) all recursive arguments in $F$ are of the
same type modulo variable renaming; 3) each type variable occurring in the
type of a function in $F$ already occurs in the type of the first argument.
The third conditions is a way to ensure that we do not end up with type
variables which occurs on the right hand-side but not on the left hand-side of
a definition. In fact, given mutually recursive functions $f_{1}, \ldots,
f_{n}$ of type $A \rightarrow B_{1}, \ldots, A \rightarrow B_{n}$ after
variable renaming, they are translated to projections of a new function of
type $A \rightarrow (B_{1} * \ldots * B_{n})$ which is defined for cases of
$A$ with products of the corresponding values of $f_{1}, \ldots, f_{n}$. The
expression $nth_n \ t$ used in the translation rule is simply an informal
abbreviation for the HOL function, defined in terms of $fst$ and $snd$, which
extracts the $n$-th projection from a tuple no shorter than $n$.\\

\noindent Terms
$$\begin{array}{ll}
  x::\tau & \Longrightarrow \ x'::\tau' \\
  () & \Longrightarrow \ () \\
  True & \Longrightarrow \ True \\
  False & \Longrightarrow \ False \\
  \&\& & \Longrightarrow \ \& \\
  || & \Longrightarrow \ | \\
  not & \Longrightarrow \ Not \\
  c & \Longrightarrow \ c \\
  t \in \{+,-,*,div,mod,<,>\} & \Longrightarrow \ t \\
  negate \ x & \Longrightarrow \ - x   \\
  {[]} & \Longrightarrow \ {[]} \\
  t:ts & \Longrightarrow \ t'\#ts' \\
  head & \Longrightarrow \ hd \\
  tail & \Longrightarrow \ tl \\
  == & \Longrightarrow \ hEq \\
  /= & \Longrightarrow \ hNEq \\
  Just & \Longrightarrow \ Some \\
  Nothing & \Longrightarrow \ None \\
  return & \Longrightarrow \ return \\
  bind & \Longrightarrow \ mbind \\

  C & \Longrightarrow \ C' \\
  f & \Longrightarrow \ f' \\
  \backslash \overline{x} \ \to \ t & \Longrightarrow 
  \ \lambda \ \overline{x}'. \ t' \\
  t_1 \ t_2 & \Longrightarrow \ t'_1 \ t'_2 \\
  (t_1,t_2) & \Longrightarrow \ (t'_1, \ t'_2) \\
  fst & \Longrightarrow \ fst \\
  snd & \Longrightarrow \ snd \\
  let \ (x_1 = t_1; & \\ 
  \qquad \dots; & \\
  \qquad x_n = t_n) \quad in \ t & \Longrightarrow   
     \ let \ (x'_1 = t'_{1}; \ \dots; \ x'_n = t'_{n}) \ in \ t' \\
  if \ t \ then \ t_1 \ else \ t_2 & \Longrightarrow 
     \ if \ t' \ then \ t'_1 \ else \ t'_2 \\
  case \ x \ of \ (p_1 \to t_1; \\
    \qquad \qquad \ldots; & \\ 
    \qquad \qquad p_n \to t_n) & \Longrightarrow  
   \ case \ x' \ p'_1 \Rightarrow t'_1 \ | \ \ldots \ | \ p'_{n} 
       \Rightarrow t'_n \\
   & \qquad \mbox{if} \ p'_1, \ldots, p'_n \neq \_ \ 
              \mbox{is \ a \ complete \ match} \\
   & case \ x' \ p'_1 \Rightarrow t'_1 \ | \ \ldots \ | \ p'_{n-1}
          \Rightarrow t'_{n-1} \\ 
   & \qquad \qquad \qquad | \ q_{1} \Rightarrow t'_n
          \ | \ \ldots \ | \ q_{k} \Rightarrow t'_n \\
   & \qquad \mbox{if} \ p_n = \_, \ \mbox{with} \ p'_1, \ldots,
         p'_{n-1}, q_1, \ldots, q_k \\
   & \qquad \mbox{complete \ match} 
\end{array}$$\\

\noindent Declarations
$$\begin{array}{l} class \ K \ where \ (Dec_1; \ldots; Dec_n \rceil) \ 
  \Longrightarrow \ class \ K' \ \subseteq 
         \ type; \ Dec'_1; \ \ldots; \ Dec'_n \\
  f :: \phi \qquad \qquad \qquad \quad \Longrightarrow
  \ consts \ f' \ :: \ \phi' \\
  type \ \tau \ = \ \tau_1 \qquad \qquad \Longrightarrow
  \ type \ \tau \ = \ \tau'_1 \\
  (data \ \phi_1 \ = \ C_{11} \ x_1 \ldots x_i
  \ | \ \ldots \ | \ C_{1p} \ y_1 \ldots y_j; \\
  \ldots; \\
  data \ \phi_n \ = \ C_{n1} \ w_1 \ldots w_h
  \ | \ \ldots \ | \ C_{1q} \ z_1 \ldots z_k) \ \Longrightarrow \\
  \qquad datatype \ \phi'_1 \ = \ C'_{11} \ x'_1 \ \ldots \ x'_i \ | \ \ldots
  \ | \ C'_{1p}
  \ y'_1 \ \ldots \ y'_j \\
  \qquad and \ \ldots \\
  \qquad and \ \phi'_n \ = \ C'_{n1} \ w'_1 \ \ldots \ w'_h \ | \ \ldots \ | \ 
  C'_{nq}
  \ z'_1 \ \ldots \ z'_k \\
  \qquad \mbox{where} \ \phi_1, \ \phi_n \ \mbox{are \ mutually \ recursive \ 
    datatype}
\end{array}$$

\noindent Definitions
$$\begin{array}{l} f \ \overline{x} \ p_1 \ \overline{x}_1 \ = \ t_1; \ 
  \ldots;
  \ f \ \overline{x} \ p_n \ \overline{x}_n \ = \ t_n \ \Longrightarrow \\
  \qquad (f \ \overline{x} \ = \ case \ y \ of \ (p_1 \to (\backslash
  \overline{x}_1 \to \ t_1); \ 
  \ldots; \ p_n (\to \backslash \overline{x}_n \to \ t_n)))' \\
  f \ \overline{x} \ = \ t \qquad \qquad \Longrightarrow \ defs \ f' :: \phi'
  \ 
  == \ \lambda \ \overline{x}'. \ t' \\
  \qquad \mbox{with} \ f::\phi \ \mbox{not \ occurring \ in} \ t \\

  f_1 \ \ y_1 \ \overline{x_1} \ =
  \ t_1; \ \ldots; \ f_n \ y_n \ 
  \overline{x_n} \ = \ t_n \ \Longrightarrow \\
  \qquad decl \ f_{new} :: 
     (\sigma_1(ctx_1) \ \cup \ \ldots \ \cup \ \sigma_n(ctx_n)
  \ \Rightarrow \\
  \qquad \qquad \qquad \qquad \sigma_1(\tau_{1a}) \ \to \
  (\sigma_1(\tau_1), \ldots, \sigma_n(\tau_n)))' \\
  \qquad primrec \ f_{new} \ sp_1 \ = 
        \ (\lambda \ \overline{x_1}'. \ t'_1 [y'_1/sp_1], \ldots, \ \lambda \ 
  \overline{x_n}'. t'_n [y'_n/sp_1]); \\ 
  \qquad \qquad \qquad \ldots; \\
  \qquad \qquad \qquad 
         f_{new} \ sp_k \ = \ (\lambda \ \overline{x_1}'. \ t'_1 
  [y'_1/sp_k], \ldots,
  \ \lambda \ \overline{x_n}'. \ t'_n [y'_n/sp_k]); \\
  \qquad defs \ f_1 \ x \ == \ nth_1 \ (f_{new} \ x); \ \ldots; 
     \ f_n \ x \ == \ nth_n \ (f_{new} \ x) \\
  \qquad \mbox{with} \ f_1:: (ctx_1 \Rightarrow \tau_{1a} \to \tau_1), 
    \ \ldots, \ f_n::(ctx_n \Rightarrow \tau_{na} \to \tau_n) \\ 
  \qquad \mbox{mutually \ recursive} \\

  instance \ ctx \ \Rightarrow \ K_T \ (T \ v_1 \ \ldots \ v_n) \ where \\ 
  \qquad \qquad \qquad (f_1::\tau_1 = t_1; \ldots; \ f_n::\tau_n = t_n)
  \ \Longrightarrow \\
  \qquad instance \\
  \qquad \qquad \tau' :: \ K_{T}' \ (\{pcpo\} \ \cup \ \{K':(K \ v_1)
  \in ctx\}, \\
  \qquad \qquad \qquad \quad \quad 
       \ldots, \  \{pcpo\} \ \cup \ \{K':(K \ v_n) \in ctx\}) \\
  \qquad \qquad \mbox{with \ proof \ obligation}; \\
  \qquad defs \quad 
        f'_1 :: (ctx \Rightarrow \tau_1)' == t'_1; \ \ldots; \ f'_n ::
     (ctx \Rightarrow \tau_n)' == t'_n\\

  instance \ Monad \ \tau \ where \ 
         (def_{eta}; def_{bind}) \ \Longrightarrow \\
  \qquad defs \quad def'_{eta}; \ def'_{bind}; \\
  \qquad t\_instantiate \ Monad \ mapping \ m\_\tau' \\
  \qquad \qquad \mbox{with \ construction \ and \ proof \ obligations} \\
  \qquad \mbox{where} \ m_\tau' \ \mbox{is \ defined \ as \ 
       theory \ morphism \ associating} \\
  \qquad MonadType.M, \ MonadOpEta.eta, 
         \ MonadOpBind.bind \\ 
  \qquad \mbox{to} \ tau', \ def'_{eta}, \ def'_{bind} \ 
        \mbox{respectively;} 
\end{array}$$\\

Type classes are translated to subclasses of \emph{type}. An axiomatisation of
Haskell equality for total functions can be found in \emph{HsHOL}.

$consts$
$$\begin{array}{ll}
  heq :: & 'a \ \to \ 'a \ \to \ bool \\
  hneq :: & 'a \ \to \ 'a \ \to \ bool \\
\end{array}$$

$axclass \ Eq \ < \ type$ 

$eqAx: heq \ p \ q \ = \ Not \ (hneq \ p \ q)$\\

Given the restriction to total functions, equality on built-in types can be
defined as HOL equality.

\section{Semantics (for HOLCF)}

Denotational semantics con be given as basis for the translation to Isabelle/HOLCF.
Essentially, the claim here is that the expressions on the left hand-side of
the following tables represent the denotational meaning of the Haskell
expressions on the right hand-side, as well as of the Isabelle/HOLCF expressions to
which they are translated. The language on the left hand-side is still based
on Isabelle/HOLCF where type have been extended with abstraction ($\Lambda$) and fixed
point ($\mu$) in order to represent the denotational meaning of domain
declarations.\\

$$\begin{array}{ll}
  \lceil a \rceil & = \ 'a::pcpo \\
  \lceil () \rceil & = \ unit \ lift \\
  \lceil Bool \rceil & = \ bool \ lift \\
  \lceil Integer \rceil & = \ int \ lift \\
  \lceil \to \rceil & = \ \to \\
  \lceil (,) \rceil & = \ * \\
  \lceil [] \rceil & = \ seq \\
  \lceil Maybe \rceil & = \ maybe \\
  \lceil T_1 \ T_2 \rceil & = \ \lceil T_1 \rceil \ \lceil T_2 \rceil \\
%  \lceil TC \rceil & = & \mu X. (\Lambda \ v_1, \ldots, v_n.
%  \ \lceil \tau_1 \rceil + \ldots + \lceil \tau_{k} \rceil)[X/TC] \\
%  \mbox{when \ there \ is \ a \ declaration} \ data \ TC \ v_1 \ \ldots \ v_n
%  \ = \ C_1::\tau_1 | \ldots |
%  C_k::\tau_k \\
  \lceil TC_{i} \rceil & = \ let \ F \ = \ \mu \ (X_1*\ldots*X_k). \\ 
  & \quad ((\Lambda \ v_{11}, \ldots, v_{1m}.  
    \ \lceil \tau_{11} \rceil + \ldots + \lceil \tau_{1p} \rceil), 
    \ \ldots, \\
  & \quad (\Lambda \ v_{k1}, \ldots, v_{kn}. \ \ldots, 
    \ \lceil \tau_{k1} \rceil + \ldots + \lceil \tau_{kq} \rceil))
    [X_1/TC_1,\ldots,X_k/TC_k] \\
  & in \ nth_i(F)  \\
  \quad \mbox{with} \ 0 < i \leq k, & \mbox{when} \ data \ TC_1 \ v_{11} \ \ldots \ v_{1m} \ = \ C_{11}::\tau_{11}
      | \ldots | C_{1p}::\tau_{1p}; \\ 
  & \quad \ldots; \ data \ TC_k \ v_{k1} \ \ldots \ v_{kn} \ = 
        \ C_{k1}::\tau_{k1} | \ldots |
  C_{kq}::\tau_{kq} \\
  & \mbox{are \ mutually \ recursive \ declarations} \\
\end{array}$$\\

The representation of term denotation is similar to what we get from the
translation, except that for functions we give the representation of the
meaning of \emph{fixrec} definitions ($FIX$ is the Isabelle/HOLCF fixed point
operator).\\

$$\begin{array}{ll}
  \lceil x::a \rceil & = \ x'::\lceil a \rceil \\
  \lceil c \rceil & = \ c' \\
  \lceil \backslash x \ \to \ f \rceil & = \ LAM \ x_{t}. \ \lceil f \rceil \\
  \lceil (a,b) \rceil & = \ (\lceil a \rceil, \lceil b \rceil) \\
  \lceil t_1 \ t_2 \rceil & = \ \lceil t_1 \rceil \ \cdot \ 
  \lceil t_2 \rceil \\
  \lceil let \ x_{1} \ \dots \ x_{n} \ in \ exp \rceil & = \ let \ \lceil
  x_{1} \rceil \ \dots \ \lceil x_{n} \rceil \ in \ \lceil exp \rceil \\
  \lceil f_i \rceil & = \ let \ g \ = 
    FIX \ (x_1,\ldots,x_n). \ 
  (\lceil t_1 \rceil, \ldots, \lceil t_{n} \rceil [f_1/x_1,\ldots,f_n/x_n] \\
   & in \ nth_i (g) \\
   \quad \mbox{with} \ 0 < i \leq n,
   & \mbox{where} \ f_1 = t_1, \ f_n = t_n \ \mbox{are \ mutually \ recursive
  \ definitions}
% \lceil TyCons \ a_{1} \ldots a_{n} \rceil & = & \lceil a_{1} \rceil
% \ldots \lceil a_{n} \rceil \ TyCons_{t}
\end{array}$$


\section{Monads with AWE}

A monad is a type constructor with two operations that can be specified
axiomatically --- \emph{eta} (injective) and \emph{bind} (associative, with
\emph{eta} as left and right unit) \cite{moggi89}.  Isabelle does not have
type constructor classes, therefore monads cannot be translated directly. The
indirect solution that we are pursuing, is to translate monadic types as types
that satisfy the monadic axioms. This solution can be expressed in terms of
theory morphisms --- maps between theories, associating signatures to
signatures and axioms to theorems in ways that preserve operations and
arities, entailing the definition of maps between theorems. Theory morphisms
allow for theorems to be moved between theories by translating their proof
terms, making it possible to implement parametrisation at the theory level
(see \cite{AWE} for details).  A \emph{parameterised theory} $\mathit{Th}$ has
a sub-theory $\mathit{Th}_P$ which is the parameter --- this may contain
axioms, constants and type declarations. Building a theory morphism from
$\mathit{Th}_P$ to a theory $I$ provides the instantiation of the parameter
with $I$, and makes it possible to translate the proofs made in the abstract
setting of $\mathit{Th}$ to the concrete setting of $I$ --- the result being
an extension of $I$.  AWE is an extension of Isabelle that can assist in the
construction of theory morphisms \cite{AWE}.

A notion of monad \cite{AWE2} can be built in AWE by defining, on an abstract
level, a hierarchy of theories culminating in \emph{Monad}, based on the
declaration of a unary type constructor $M$ (in \emph{MonadType}) with the two
monad operations (contained in \emph{MonadOpEta} and \emph{MonadOpBind},
respectively) and the relevant axioms (in \emph{MonadAxms}). To show that a
specific type constructor forms a monad, we have to construct a theory
morphism from \emph{MonadAxms} to the specific theory; this involves giving
specific definitions of the operators, as well as discharging proof
obligations associated with specific instances of
the axioms. The following gives an example.  \\

\noindent $data \ LS \ a  \ = \ N \ | \ C \ a \ (LS \ a)$

\noindent $instance \ Monad \ LS \ where$ 
$$\begin{array}{ll}
  return \ x & = \ C \ x \ N   \\
  x \ >>= \ f & = \ case \ x \ of \\
  & N \ \to \ N  \\
  & C \ a \ b \ \to \ cnc \ (f \ a) \ (b \ >>= \ f) 
\end{array}$$

$cnc :: \  LS \ a \ \to \ LS \ a \ \to \ LS \ a $
$$\begin{array}{ll}
cnc \ x \ y & = \ case \ x \ of \\
  & N \ \to \ y \\
  & C \ w \ z \ \to \ cnc \ z \ (C \ w \ y)
\end{array}$$

\noindent These definitions are plainly translated to HOL, as follows. Note
that these are not overloaded definitions. \\

\noindent $datatype \ 'a \ LS \ = \ N \ | \ C \ 'a \ ('a \ LS)$

\noindent $consts $
$$\begin{array}{ll}
  return\_LS & :: \ 'a \ \Rightarrow 'a \ LS \\
  mbind\_LS & :: \ 'a \ LS \ \Rightarrow \ ('a \ \Rightarrow \ 'b \ LS) \ 
  \Rightarrow \ 'b \ LS \\
  cnc & :: \ 'a \ LS \ \Rightarrow \ 'a \ LS \ \Rightarrow \ 'a \ LS \\
\end{array}$$

\noindent $defs $
$$\begin{array}{l}
return\_LS\_def : \ return\_LS:: \ ('a \ LS \ \Rightarrow 'a) \ == \
                \lambda  x. \ C \ x \ N 
\end{array}$$

\noindent $primrec $
$$\begin{array}{ll}
mbind\_LS \ N & = \ \lambda f. \ N \\
mbind\_LS \ (C \ pX1 \ pX2) & = \ \lambda f. \ cnc \ (f \ pX1) \ (mbind\_LS \ pX2 \ f)
\end{array}$$

\noindent $primrec $
$$\begin{array}{ll}
cnc \  N & = \ \lambda  b. \ b \\
cnc \ (C \ pX1 \ pX2) & = \ \lambda  b. \ cnc \ pX2 \ (C \ pX1 \ b)
\end{array}$$

\noindent In order to build up the instantiation of \textit{LS} as a monad,
here it is defined the morphism $m\_LS$ from \emph{MonadType} to the
instantiating theory \emph{Tx}, by associating $M$ to $LS$.\\

$thymorph \ m\_LS : \ MonadType \ \longrightarrow \ Tx $
$$\begin{array}{l}
  maps \ [('a \ MonadType.M \ \mapsto \ 'a \ Tx.LS)] \\
  renames: \ [(MonadOpEta.eta \mapsto return\_LS),
                (MonadOpBind.bind \mapsto mbind\_LS)] \\
\end{array}$$

\noindent
Renaming is used in order to avoid name clashes in case of more than one
monads --- here again, note the difference with overloading. Morphism
\emph{m\_LS} is then used to instantiate the parameterised theory
\emph{MonadOps}.\\

\noindent $t\_instantiate \ MonadOps \ mapping \ m\_LS$ \\

\noindent This instantiation gives the declaration of the
instantiated methods, which may now be defined.\\

\noindent $defs $
$$\begin{array}{l}
LS\_eta\_def: \ LS\_eta \ == \ return\_LS \\
LS\_bind\_def: \ LS\_bind \ == \ mbind\_LS 
\end{array}$$

\noindent In order to construct a mapping from \emph{MonadAxms} to
\emph{Tx}, the user needs to prove the monad axioms as HOL lemmas (in this
case, by straightforward simplification). The translation will print out
\emph{sorry} instead.

$$\begin{array}{ll}
lemma \ LS\_lunit : & LS\_bind \ (LS\_eta \ x) \ t \ = \ t \ x \\
lemma \ LS\_runit : & LS\_bind \ (t:: \ 'a \ LS) \ LS\_eta \ = \ t \\
lemma \ LS\_assoc : & LS\_bind \ (LS\_bind \ (s:: \ 'a \ LS) \ t) \ u \ = \\
                    & LS\_bind \ s \ (\lambda  x. \ LS\_bind \ (t \ x) \ u) \\
lemma \ LS\_eta\_inj : & LS\_eta \ x \ = \ LS\_eta \ y \ \Longrightarrow \ x \ = \ y 
\end{array}$$

\noindent Now, the morphism from \emph{MonadAxms} to
\emph{Tx} can be built, and then used to instantiate \emph{Monad}. This gives
automatically access to the theorems proven in \emph{Monad} and, modulo
renaming, the monadic syntax which is defined there. \\

$thymorph \ mon\_LS : \ MonadAxms \ \longrightarrow \ Tx $
$$\begin{array}{ll}
  maps & [('a \ MonadType.M \ \mapsto \ 'a \ Tx.LS)] \\
   &  [(MonadOpEta.eta  \ \mapsto \ Tx.LS\_eta), \\
   &   (MonadOpBind.bind \ \mapsto \ Tx.LS\_bind)]
\end{array}$$
$t\_instantiate \ Monad \ mapping \ mon\_LS$\\
$renames: \ [ \ldots ]$\\

\noindent The \emph{Monad} theory allows for the characterisation of
single parameter operators. In order to cover other monadic operators, a
possibility is to build similar theories for type constructors of fixed arity.
An approach altogether similar to the one shown for HOL could be used, in
principle, for Isabelle/HOLCF as well.



\begin{comment}
Here is a simple example. \\

\noindent $ fn3 \ :: \ AC \ a \ b \to (a \to a) \to AC \ a \ b $
$$\begin{array}{ll}
fn3 \ k \ f & = \ case \ k \ of \\
  & Ct \ x \to Ct \ (f \ x) \\
  & Rt \ x \ y \to Rt \ (fn4 \ x) \ y \\
\end{array}$$
   
\noindent $ fn4 \ :: \ AC \ a \ b \to AC \ a \ b $
$$\begin{array}{ll}
fn4 \ k & = \ case \ k \ of \\ 
  & Rt \ x \ y \to Rt \ (fn3 \ x \ (\backslash z \to z)) \ y \\
   & Ct \ x \to Ct \ x \\
\end{array}$$

\noindent It translates to HOL as follows.\\

\noindent $consts$
$$\begin{array}{ll} fn3 \ :: & ('a, 'b) \ AC \Rightarrow ('a \Rightarrow
  'a) \Rightarrow 
  \ ('a, 'b) \ AC \\
  fn4 \ :: & ('a, 'b) \ AC \Rightarrow 
  \ ('a, 'b) \ AC \\
  fn3\_Xfn4\_X :: & ('a, 'b) \ AC \Rightarrow 
  \ (('a \Rightarrow 'a) \Rightarrow \\
  & ('a, 'b) \ AC) * 
  \ (('a \Rightarrow 'a) \Rightarrow 
  \ ('a, 'b) \ AC) \\
\end{array}$$

\noindent $defs$
$$\begin{array}{lll}
fn3\_def : & fn3 \ == \ \lambda  k \ f. \ fst \ (( fn3\_Xfn4\_X ::  \\ 
  & \ ('a, 'b) \ AC \Rightarrow (('a \Rightarrow 'a) \\
  & \Rightarrow ('a, 'b) \ AC) * ((unit \Rightarrow unit) \Rightarrow \\ 
  & ('a, 'b) \ AC) ) \ k) \ f \\
  fn4\_def : & fn4 \ == \ \lambda  k \ f. \ snd \ (( fn3\_Xfn4\_X ::  \\ 
  & ('a, 'b) \ AC \Rightarrow \\
  & (('a \Rightarrow 'a) \Rightarrow ('a, 'b) \ AC) * \\ 
  & (('a \Rightarrow 'a) \Rightarrow ('a, 'b) \ AC) ) \ k) \ f \\
\end{array}$$


\noindent $primrec$ 
$$\begin{array}{lll}
fn3\_Xfn4\_X & (Ct \ pX1) \ = & (\lambda  f. \ Ct \ (f \ pX1), \\
                           &        &  \lambda  f. \ Ct \ pX1) \\
fn3\_Xfn4\_X & (Rt \ pX1 \ pX2) & =  \\
  & (\lambda  f. \ Rt \ (snd & (fn3\_Xfn4\_X \ pX1) \ f) \ pX2, \\
  &  \lambda  f. \ Rt \ (fst & (fn3\_Xfn4\_X \ pX1) \ f) \ pX2) \\
\end{array}$$
\end{comment}


In the following, we give a definition of the sub-language of Haskell
$H_c$ that is covered by the translation to Isabelle/HOLCF.\\

\noindent Types
$$\begin{array}{lll}
  \tau \ = () \\
  & Bool \\
  & Integer & \\
  & v & \mbox{type \ variable}\\ 
  & \tau_1 \to \tau_2 & \\
  & (\tau_1,\tau_2) & \\
%  & Either \ tau_1 \ tau_2 \\
  & [\tau] & \\
  & Maybe \ \tau & \\
  & T \ \tau_{1} \ \ldots \tau_{n} &  \mbox{either \ datatype \ or \
  defined \ type} 
%  & T \ \tau_{1} \ \ldots \ \tau_{n} \qquad \mbox{defined \ type}
\end{array}$$

\noindent Type classes
$$\begin{array}{lll}
  K \ = & Eq & \mbox{with \ default \ methods} \ ==, \ /= \\
  & K & \mbox{defined \ type \ class} \\
\end{array}$$

\noindent Contexts
$$\begin{array}{lll}
  ctx \ = & \{K \ v\} \ \cup \ ctx & \\
          & \{\} & \mbox{empty \ context} \\
\end{array}$$

\noindent Type schemas
$$\begin{array}{ll}
  \phi \ = & ctx \Rightarrow \tau \\
  & \qquad \mbox{where \ all \ variables \ in} \ ctx \ \mbox{are \ in} \ \tau
\end{array}$$

\noindent Simple patterns
$$\begin{array}{lll} 
  sp \ = & \_ & \mbox{wildcard} \\
    & v & \mbox{variable \ of \ datatype}\\
    & C \ \overline{v} & \mbox{case \ of \ datatype}
\end{array}$$

\noindent Terms
$$\begin{array}{lll} t \ = 
  & t \in \{True,False,\&\&,||,not\} & \mbox{on \ Boolean} \\
  & c \in \aleph & \mbox{Integer \ constant} \\
  & t \in \{+,*,-,div,mod,negate,<,>\} & \mbox{on \ Integer} \\
  & t \in \{:,head,tail\} & \mbox{on list} \\
  & t \in \{==,\ /=\} & \mbox{on \ equality \ types} \\
  & t \in \{Just,Nothing\} & \mbox{on \ \emph{maybe} \ types} \\
  & t \in \{fst,snd\} & \mbox{on \ pairs} \\
  & (,) & \\
  & x & \mbox{variable} \\ 
  & f & \mbox{function \ symbol} \\
  & C & \mbox{data \ constructor} \\  
  & if \ t \ then \ t_1 \ else \ t_2 \\
  & case \ x \ of \ (sp_{1} \to t_1; \ \ldots; \ sp_{n} \to t_n) \\
  & let \ (x_1 = t_1; \ \ldots; \ x_n = t_n) \ in \ t 
\end{array}$$

\noindent Declarations
$$\begin{array}{ll}
  Decl \ = & type \ T \ \overline{v} \ = \ \tau \\
  & data \ ctx \ \Rightarrow \ T \ \overline{v} \ =
  \ C_{1} \ \overline{x}_1 \ | \ \ldots \ | \ C_{k} \ \overline{x}_k \\
%  & \qquad \mbox{with} \ {\overline{x}}_1, {\overline{x}}_k \ \subseteq \ 
%  \overline{v} \\
  & ctx \ \Rightarrow \ f \ :: \ \tau \\
  & class \ K \ 
  where \ (f_{1}:: \tau_{1}; \ldots; f_{n}::\tau_{n}) \\
  & \qquad \mbox{where} \ \tau_{1}, \tau_{n} \ 
  \mbox{have \ only \ one \ type \  variable} 
\end{array}$$

\noindent Definitions
$$\begin{array}{ll}
  Def \ = & f \ \overline{v} \ = \ t \\
  & f \ \overline{v}_{1} \ sp_1 \ {\overline{w}}_{1} \ = \ t_1; \ \ldots;
  \ f \ \overline{v}_{n} \ sp_n \ {\overline{w}}_{n} \ = \ t_n \\
  & instance \ ctx \Rightarrow K \ \tau \ where \ 
  (f_1 = t_1; \ldots; \ f_n = t_n) \\
  & \qquad \mbox{where} \ f_1, \ldots, f_n \ \mbox{are \ methods \ of} \ K
\end{array}$$\\


\subsection{HOL}

In contrast, the following gives a definition of the Haskell sub-language
$H_s$ that is covered by the translation to HOL.\\

\noindent Types, Type classes, Contexts, Type schemas, Simple patterns,
Declarations 

 \emph{as before}\\

\noindent Type constructor classes 

 $Monad$\\

\noindent Terms
$$\begin{array}{lll} t \ = 
  & () & \\
  & t \in \{True,False,\&\&,||,not\} & \mbox{on \ Boolean} \\
  & c \in \aleph & \mbox{Integer \ constant} \\
  & t \in \{+,*,-,div,mod,negate,<,>\} & \mbox{on \ Integer} \\
  & t \in \{:,head,tail\} & \mbox{on list} \\
  & t \in \{==,\ /=\} & \mbox{on \ equality \ types} \\
  & t \in \{Just,Nothing\} & \mbox{on \ Maybe} \\
  & t \in \{return,bind\} & \mbox{on \ monadic \ types} \\
  & t \in \{fst,snd\} & \mbox{on \ pairs} \\
  & (,) & \\
  & x & \mbox{variable} \\ 
  & f & \mbox{function \ symbol} \\
  & C & \mbox{data \ constructor} \\
  & if \ t \ then \ t_1 \ else \ t_2 \\
  & case \ x \ of \ (sp_{1} \to t_1; \ \ldots; \ sp_{n} \to t_n) \\
  & \qquad \mbox{with} \ sp_1, \ldots, sp_n \ \mbox{complete \ match} \\
  & let \ (x_1 = t_1; \ \ldots; \ x_n = t_n) \ in \ t 
\end{array}$$

\noindent Definitions
$$\begin{array}{ll}
  Def \ = & f \ \overline{v} \ = \ t \\
  & \qquad \mbox{where} \ f \ \mbox{is \ totally \ defined \ and \ primitive
    \ recursive} \\
  & f \ \overline{v}_1 \ sp_1 \ \overline{w}_1 \ = \ t_1; \ \ldots;
  \ f \ \overline{v}_n \ sp_n \ \overline{w}_n \ = \ t_n \\
  & \qquad \mbox{where} \ f \ \mbox{is \ totally \ defined \ and \ primitive
    \ recursive} \\
  & f_1 \ v_1::\tau \ \overline{w_1} \ = \ t_1; \ ldots; \ 
  f_n \ v_n::\tau \ \overline{w_n} \ = \ t_n \\
  & \qquad \mbox{where} \ f_1::\phi_1, \ldots, f_n::\phi_n \ \mbox{are \ 
    totally \ defined, \ mutually} \\
  & \qquad \mbox{primitive \ recursive \ in \ the \ first \ argument, \ and \ 
    forall} \\
  & \qquad 0 < i \leq n \ \mbox{there \ exists \ type 
                                     \ variable \ renaming} \ 
  \sigma_i \ \mbox{such} \\
  & \qquad \mbox{that} \ \tau_1 = \sigma_i(\tau_i) \ \mbox{and \ all \ the \ 
    variables \ in} \ \phi_i \ \mbox{appear \ in} \ \tau_i \\
  & instance \ ctx \Rightarrow K \ \tau \ where \ 
  (f_1 = t_1; \ldots; \ f_n = t_n) \\
  & \qquad \mbox{where} \ f_1, \ldots, f_n \ \mbox{are \ totally \ defined, \ 
    primitive \ recursive} \\
  & \qquad \mbox{methods \ of} \ K
\end{array}$$\\



\noindent Terms
$$\begin{array}{ll}
  x::\tau & \Longrightarrow \ x'::\tau' \\
  () & \Longrightarrow \ Def \ () \\
  True & \Longrightarrow \ TT \\
  False & \Longrightarrow \ FF \\
  \&\& & \Longrightarrow \ trand \\
  || & \Longrightarrow \ tror \\
  not & \Longrightarrow \ neg \\
  c & \Longrightarrow \ Def \ c \\
  t \in \{+,-,*,div,mod,<,>\} & \Longrightarrow \ fliftbin \ t \\
  negate & \Longrightarrow \ flift2 \ -   \\
  {[]} & \Longrightarrow \ nil \\
  t:ts & \Longrightarrow \ t'\#\#ts' \\
  head & \Longrightarrow \ HD \\
  tail & \Longrightarrow \ TL \\
  == & \Longrightarrow \ hEq \\
  /= & \Longrightarrow \ hNEq \\
  Just & \Longrightarrow \ return \\
  Nothing & \Longrightarrow \ fail \\

  C & \Longrightarrow \ C' \\
  f & \Longrightarrow \ f' \\
  \backslash \overline{x} \ \to \ t & \Longrightarrow 
  \ LAM \ \overline{x}'. \ t' \\
  t_1 \ t_2 & \Longrightarrow \ t'_1 \ \cdot \ t'_2 \\
  (t_1,t_2) & \Longrightarrow \ (t'_1, \ t'_2) \\
  fst & \Longrightarrow \ cfst \\
  snd & \Longrightarrow \ csnd \\
  let \ (x_1 = t_1; & \\ 
  \qquad \dots; & \\
  \qquad x_n = t_n) \quad in \ t & \Longrightarrow   
     \ let \ (x'_1 = t'_{1}; \ \dots; \ x'_n = t'_{n}) \ in \ t' \\
  if \ t \ then \ t_1 \ else \ t_2 & \Longrightarrow 
     \ If \ t' \ then \ t'_1 \ else \ t'_2 \ fi \\
  case \ x \ of \ (p_1 \to t_1; & \\
    \qquad \qquad \ldots; & \\ 
    \qquad \qquad p_n \to t_n) & \Longrightarrow  
   \ case \ x' \ p'_1 \Rightarrow t'_1 \ | \ \ldots \ | \ p'_{n} 
       \Rightarrow t'_n \\
   & \qquad \mbox{if} \ p'_1, \ldots, p'_n \neq \_ \ 
              \mbox{is \ a \ complete \ match} \\
   & case \ x' \ p'_1 \Rightarrow t'_1 \ | \ \ldots \ | \ p'_{n-1}
          \Rightarrow t'_{n-1} \\ 
   & \qquad \qquad \qquad | \ q_{1} \Rightarrow t'_n
          \ | \ \ldots \ | \ q_{k} \Rightarrow t'_n \\
   & \qquad \mbox{if} \ p_n = \_, \ \mbox{with} \ p'_1, \ldots,
         p'_{n-1}, q_1, \ldots, q_k \\
   & \qquad \mbox{complete \ match} \\
   & case \ x' \ p'_{1} \Rightarrow t'_1 \ | \ \ldots \ | \ p'_n
  \Rightarrow t'_n \\
   & \qquad \qquad \qquad | \ q_{1} \Rightarrow \bot \ 
           | \ \ldots q_{k} \Rightarrow \bot \\
   & \qquad \mbox{with} \ p'_1, \ldots, p'_{n}, q_1, \ldots, q_k \ 
      \mbox{complete \ match} 
\end{array}$$\\
