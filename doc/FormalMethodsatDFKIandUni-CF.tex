% FormalMethodsatDFKIandUni-CF.tex
\begin{hcarentry}[section]{Formal Methods at DFKI and University Bremen and University Magdeburg}
\report{Christian Maeder}%05/13
\participants{Mihai Codescu, Christoph L\"uth, Till Mossakowski}
\status{active development}
\makeheader

The activities of our groups center on formal methods, covering a
variety of formal languages and also translations and heterogeneous
combinations of these.

We are using the Glasgow Haskell Compiler and many of its extensions
to develop the Heterogeneous tool set (Hets).  Hets is a parsing,
static analysis and proof management tool incorporating various
provers and different specification languages, thus providing a tool
for heterogeneous specifications. Logic translations are first-class
citizens.

The languages supported by Hets include the CASL family, such as the
Common Algebraic Specification Language (CASL) itself (which provides
many-sorted first-order logic with partiality, subsorting and
induction), HasCASL, CoCASL, CspCASL, and an extended modal logic
based on CASL. Other languages supported include propositional logic,
QBF, Isabelle, Maude, VSE, TPTP, THF, FPL (logic of functional
programs), LF type theory and still Haskell (via Programatica). More
recently, ontology languages like OWL, RDF, Common Logic, and DOL (the
Distributed Ontology Language) have been integrated.

Hets can speak to the following provers:
\begin{compactitem}
\item
minisat, zChaff (SAT solvers),
\item
SPASS, Vampire, Darwin, KRHyper and MathServe (automated first-order theorem provers),
\item
Pellet and Fact++ (description logic tableau provers),
\item
Leo-II and Satallax (automated higher-order theorem provers),
\item
Isabelle (an interactive higher-order theorem prover),
\item
CSPCASL-prover (an Isabelle-based prover for CspCASL),
\item VSE (an interactive prover for dynamic logic).
\end{compactitem}

The user interface of the Hets implementation (about 200K lines of
Haskell code) is based on some Haskell sources such as bindings to
uDrawGraph (formerly Davinci) and Tcl/TK that we maintain and also
gtk2hs~\cref{gtk2hs}.  Additionally we have a command line interface
and a prototypcial web interface based on warp~\cref{warp} with a
RESTful API.

HasCASL is a general-purpose higher-order language which is in particular
suited for the specification and development of functional programs; Hets also
contains a translation from an executable HasCASL subset to Haskell. There is
a prototypical translation of a subset of Haskell to Isabelle/HOL.

\FurtherReading
\begin{compactitem}
\item Group activities overview:

\url{http://www.informatik.uni-bremen.de/agbkb/forschung/formal\_methods/}

\item CASL specification language:

\url{http://www.cofi.info}

\item distributed ontology language DOL:

\url{http://www.ontoiop.org}

\item Heterogeneous tool set:

\url{http://hets.dfki.de}\\
\url{http://www.informatik.uni-bremen.de/htk/}\\
\url{http://www.informatik.uni-bremen.de/uDrawGraph/}
\end{compactitem}
\end{hcarentry}
