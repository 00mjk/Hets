
\documentclass{article}

\usepackage[latin1]{inputenc}
\usepackage{url,hyperref}

\title{How to build the hets web interface}
\author{Klaus L�ttich}

\begin{document}
\maketitle

\section{Quick start guide}
This part gives only a very rough overview of what you need.
\subsection{Installation of WASH}
Prerequisites: ghc $>=$ 6.4, GNU make, darcs $>=$ 1.0.3

\begin{enumerate}
\item get WashNGo via darcs:\\
  \texttt{darcs get http://darcs.complete.org/local-branches/WashNGo/}
\item copy additional files to the WashNGo repository
\item patch your darcs repository with the darcs patch (or the plain
  patch)\\
  from inside the WashNGo repository directory:\\
  \verb|darcs apply UniqueRenamePatch|
\item inside the WashNGo repository run \verb|make -f Makfile.hets all|
\end{enumerate}
\subsection{Compilation of hets.cgi}
If you want different locations for the generated files and html-links
you have to change \verb|HetCATS/GUI/hets_cgi.hs| and then just run
\verb|make cgi|.

\section{Detailed instructions}

\subsection{Installation of WASH}
All the additionally needed files and patches are in the archive
\verb|HetCATS/GUI/wash.tgz|. Install darcs
(\url{http://www.abridgegame.org/darcs/}).
\begin{enumerate}
\item get WashNGo via darcs:\\
  \texttt{darcs get
  http://darcs.complete.org/local-branches/WashNGo/}\\
  You get a directory \verb|WashNGo| which is referenced as the
  \emph{repository directory} from now on.
\item copy all files from the directory
  \verb|HetCATS/GUI/wash_addon/|to the repository directory.
\item patch your darcs repository with the darcs patch\\
  from inside the repository directory:\\
  \verb|darcs apply UniqueRenamePatch|
\item inside the WashNGo repository run
  \verb|make -f Makfile.hets all|\\   
  and you end up with \verb|$HOME/wash-pkg/|% $ where the file
  \verb|package.conf|, all \verb|*.hi|-files and the needed library
  file are installed to.
\end{enumerate}

\subsection{Configuration and Compilation of hets.cgi}

All the configuration of the web interface is done via String
constants in \\ \verb|HetCATS/GUI/hets_cgi.hs|. They are all documented
in this source file.

After checking and/or changing the configuration constants run
\verb|make cgi|. If Programmatica is in place even Haskell support is
compiled into hets.cgi.

\paragraph{Note for Uni-Bremen people} The user id with which hets.cgi
is running must have permission to write to \verb|/home/www/cofi/hets-tmp/|!
\end{document}
