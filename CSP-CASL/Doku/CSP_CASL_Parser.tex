\documentclass[11pt, twoside]{article}

\usepackage[latin1]{inputenc}
\usepackage{xspace}
\usepackage{german}
\usepackage{moreverb}

%%%%%%%% Seitengestaltung
\oddsidemargin 6pt
\evensidemargin 6pt
\marginparwidth 48pt
\marginparsep 10pt 
\topmargin -18pt
\headheight 12pt
\headsep 25pt
%\footheight 12pt
\footskip 30pt 
\textheight 625pt
\textwidth 431pt
\columnsep 10pt
\columnseprule 0pt 

\usepackage{fancyheadings}
\usepackage{xspace}

%%%%%%%%%%%% Ergaenzungen fuer pagestyle{fancy}
\pagestyle{fancy}
\renewcommand{\baselinestretch}{1.3}
\addtolength{\headheight}{2pt}
\parindent1.5em 
\headsep0.8cm
\headheight0.68cm

\author{Daniel Pratsch $<$coldsky@tzi.de$>$\\
         Markus Roggenbach $<$roba@tzi.de$>$} 
\title{CSP-CASL Parser -- Systemdokumentation}
\date{\today}

\begin{document}

\maketitle

\tableofcontents

\section{Systemvoraussetzungen}

Der CSP-CASL Parser wird relativ zu HetCATS programmiert. F�r die
Systemumgebunt hei�t das: Das CSP-CASL Verzeichnis ist direktes
Unterverzeichnis von HetCATS.

\begin{appendix}

%\section{CoreTypes}\label{app:core_types}
%\listinginput{1}{../tool/src/core_types.sml}

%\bibliographystyle{alpha}
%\bibliography{doc.bib}

\end{appendix}

\end{document}

