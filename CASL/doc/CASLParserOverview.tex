\documentclass{article}

\parindent 0pt
\parskip 5pt

\begin{document}

\title{CASL basic items}

\author{C. Maeder}

\maketitle

\section{Preliminaries}

see the documentation for HetCATS


\section{Compiling the CASL parser}

\texttt{cd HetCATS/CASL} and call \texttt{./ghc-call} (a script that contains
all necessary ghc options and parameters). An alternative is to call
\texttt{gmake capa} in the top directory.

This creates a binary called \texttt{capa} that can parse (and pretty
print) ``Basic Specification with Subsorts'' of the CASL Summary
version 1.0.1 \\ (www.brics.dk/Projects/CoFI/Documents/CASL/Summary/).

Structured specifications (see \texttt{Static/hetpa.hs}) must be parsed by the heterogenous parser
.

Note: if module \texttt{AS\_Basic\_CASL} can not be found, then
\texttt{AS\_Basic\_CASL.hs} was not ``drifted'' (i.e., derived from
\texttt{AS\_Basic\_CASL.der.hs}). 

Also \texttt{Common/AS\_Annotation.hs} must have been drifted.

\section{Testing}

A test call might be: 

\texttt{./capa BasicSpec.casl}

The binary \texttt{capa} is also used to test various parsers. Test cases are
given as \texttt{*.casl} files.

Calling \texttt{./runcheck.sh ./capa} performs many tests and compares the
results with corresponding \texttt{*.output} files.

Alternatively a test can be run using \texttt{runhugs}. For this the
file \texttt{capa.lhs} is executable and the test can be run
by \texttt{./runcheck.sh ./capa.lhs}.

\texttt{Wrong*.casl} files contain wrong CASL code and should produce
error messages. Apart from extra tests, correct (\texttt{Bla.casl})
and wrong files (\texttt{WrongBla.casl}) are tested. If a \texttt{diff}
with the expected output fails, then also the number of
``\texttt{error}'' occurrences (in the produced output) is counted. For a
successful comparison a ``\texttt{passed}'' is emitted.

Calling ``\texttt{./runcheck.sh ./capa set}'' will update the
\texttt{*.output} files (and a subsequent \texttt{./runcheck.sh ./capa} should
always pass). 

Alternatively, all checks can be preformed by \texttt{make check}. Output
files will be overwritten by \texttt{make output}. With \texttt{cvs up} or
\texttt{cvs diff} changes can be compared to checked-in versions of the output
files.

\section{Source files in \texttt{HetCATS/CASL}}

\begin{description}
\item[AS\_Basic\_CASL.der.hs] defines the abstract syntax tree for CASL
\item[Formula.hs] parser \texttt{term} and \texttt{formula}
\item[ItemList.hs] parses items (separated by semicolons) generically
  (\texttt{itemList}) and annotations between or following these
  items. 
\item[Latin.hs] used by \texttt{Static.hs}
\item[LiteralFuns.hs] used by \texttt{Print\_AS\_Basic.hs}
\item[Logic\_CASL.hs] contains the instance for the class \texttt{Logic.Logic}
\item[MixfixParser.hs] uses precedence and associativity to resolve
  mixfix terms (and formulae). Also \%list, \%string, \%number and
  \%floating annotations are resolved 
\item[OpItem.hs] parsers \texttt{opItem(s)} and \texttt{predItem(s)}
\item[Parse\_AS\_Basic.hs] supplies the top-level parsers \texttt{basicSpec,
  basicItems, dotFormulae} and \texttt{sigItems}
\item[Print\_AS\_Basic.hs] pretty prints data types of
  \texttt{AS\_Basic\_CASL.hs}
\item[RunMixfixParser.hs] is an additional driver to test the mixfix
  analysis
\item[RunParsers.hs] contains the driver called in Main for test parsers and
  is reused in \texttt{HetCATS/HasCASL}. (It declares an existential type.)
\item[RunStaticAna.hs] is an additional driver to test the static 
  analysis (call \texttt{capa analysis <file>})  
\item[Sign.hs] is the analysed abstract syntax and the signature for an
  instance of the class \texttt{Logic}.
\item[SortItem.hs] parser \texttt{sortItem(s)} (requires \texttt{Formula.hs}
  for subsort definitions)
\item[Static.hs] implements the static analysis
\item[Sublogics.hs] used by \texttt{Logic\_CASL.hs}
\item[SymbolParser.hs] parses symbols and symbol maps that are not needed for
  basic specs but for heterogeneous structured specifications
  (\texttt{Logic\_CASL.hs}). These parsers do not deal with annotations,
  currently
\item[TypeItem.hs] parsers \texttt{datatype} and \texttt{typeItems}
\item[capa.lhs] is the main module that simply lists the test parsers
  and is executable for hugs
\end{description}

\section{Remarks}

The following files have been moved to the \texttt{Common} directory.

\begin{description}
\item[Token.hs] generic parser for mixfix ids and some keyword parsers
  (reused in \texttt{HetCATS/HasCASL}). The haskell data types
  \texttt{Token} and \texttt{Id} are defined in \texttt{Common/Id.hs}!
\item[Lexer.hs] various scanners and extensions of the Parsec library 
\item[Keywords.hs] CASL keywords as named identifiers (to be used
  for parsing and printing and thus ensuring consistent spellings)
\end{description}

The above files rely on the data types in \texttt{Id.hs} and
\texttt{AS\_Annotation.der.hs} in the \texttt{Common} directory. For pretty
printing the class in \texttt{PrettyPrint.hs} is required.
\texttt{PrettyPrint.hs} in turn relies on
\texttt{GlobalAnnotations.hs} and \texttt{LaTeX\_funs.hs} and
\texttt{LaTeX\_maps.hs} (to consider \%display annotations).

\texttt{Common/Lib/Pretty.lhs} is an adapted copy of\\
\texttt{//research.microsoft.com/\~{}simonpj/downloads/pretty-printer/pretty.html}
that is also included as Haskell (\texttt{text}) library. 

\texttt{PrettyPrint.hs} supplies an instance for \texttt{Id}. The
instance for annotations is given in \texttt{Print\_AS\_Annotation.hs}.
\texttt{CASL/Print\_AS\_Basic} further imports
\texttt{Common.PPUtils}. (Pretty printing requires ``glasgow extensions''.)

\texttt{GlobalAnnotations.hs} contains the \texttt{PrecedenceGraph} and
imports \texttt{Map} and \texttt{Graph}. \texttt{Graph} has its own
implementation of maps, namely \texttt{SimpleMap.hs}.  

\texttt{Graph} came
from \texttt{www.cs.orst.edu/\~{}erwig/fgl/}.  

\texttt{Map} and \texttt{Set}
come from \texttt{www.cs.uu.nl/\~{}daan/ddata.html}.

With \texttt{GlobalAnnotations.hs} also
\texttt{GlobalAnnotationsFunctions.hs} is sometimes needed. The later
depends on \texttt{GraphUtils.hs}. 

\texttt{CASL/MixfixParser.hs} uses \texttt{Result.hs} to combine
diagnostic messages (and \texttt{Set} to combine states).

The actually parser combinators come from 
\texttt{www.cs.uu.nl/\~{}daan/parsec.html} that are part of the
Haskell library but also have been slightly extended (by
\texttt{consumeNothing}) and included in the \texttt{Common/Lib}
subdirectory. \texttt{Parsec.hs} simply re-exports parts from
\texttt{Prim.hs}, \texttt{Combinator.hs},
\texttt{Char.hs}, \texttt{Error.hs} and
\texttt{Pos.hs}. 

\texttt{Parsec.Pos} is reused in \texttt{Id.hs}.

\texttt{Anno\_Parser.hs} uses the mixfix ids from \texttt{Token.hs}.
(\texttt{CaslLanguage.hs} that is based on \texttt{Parsec.Token} and
\texttt{Parsec.Language} supplies an alternative parser for ids.)
\texttt{Anno\_Parser.hs} further uses \texttt{Parsec.Perm} (that relies on
glasgow extensions).

The \texttt{MixfixParser} only operates on \texttt{FORMULA} and
\texttt{TERM} from \texttt{AS\_Basic\_CASL} and not on
types in \texttt{Sign.hs}. 

Type checking (considering subtypes and overloading) of terms and
formulae (plus their conversions to the types in \texttt{Sign.hs}) is
still missing.

\section{Scripts}

\texttt{runcheck.sh} includes \texttt{checkFunctions.sh}.
\texttt{iterate.sh} checks differences between the working copy and
the most recent cvs version.

\end{document}

